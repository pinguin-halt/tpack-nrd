\section{Introduction}\label{introduction}

Lanskap pendidikan IPA abad ke-21 menuntut guru memiliki kompetensi
multifaset yang melampaui penguasaan konten disiplin ilmu semata. Guru
IPA kini diharapkan merancang pembelajaran yang secara bermakna
mengintegrasikan teknologi digital, mendorong penalaran interdisipliner,
dan mengatasi tantangan keberlanjutan yang mendesak \citep{kelley2016aconceptual,unesco2017item}. Ekspektasi ini sangat tinggi bagi calon guru IPA. Dalam
persiapan profesionalnya, mereka perlu dibekali tidak hanya pemahaman
teoretis tentang berbagai kerangka pedagogik, tetapi juga keterampilan
praktis untuk menggabungkan beragam kerangka tersebut menjadi RPP yang
runtut dan saling terhubung. Namun, program pendidikan guru yang ada
sering kali masih memperlakukan aspek-aspek ini (Technological
Pedagogical Content Knowledge (TPACK), integrasi
Science-Technology-Engineering-Mathematics (STEM), dan Education for
Sustainable Development (ESD)) sebagai bagian kurikulum yang terpisah.
Akibatnya, calon guru seringkali harus menghadapi sendiri kompleksitas
merancang RPP yang integratif.

Kemampuan merancang RPP integrative (RPP yang secara simultan
menyematkan pedagogi berbasis teknologi, koneksi STEM interdisipliner,
dan perspektif keberlanjutan) merepresentasikan kompetensi profesional
tingkat tinggi yang disebut \emph{teacher design capacity} \citep{brown2009item}. Kapasitas ini bukan sekadar kumpulan keterampilan yang
berdiri sendiri, melainkan kompetensi yang muncul (emergen) yang
menuntut guru mengintegrasikan berbagai jenis pengetahuan secara
simultan \\citep{mckenney2015teacherdesign}. Satu aspek kunci integrasi ini melalui
kerangka TPACK, yaitu irisan domain pengetahuan yang memungkinkan guru
memanfaatkan teknologi untuk mendukung pedagogi yang selaras dengan
karakteristik konten \\citep{mishra2006technologicalpedagogical}. Kajian selanjutnya memperluas
logika integratif tersebut ke pendidikan STEM, di mana perancangan
pembelajaran yang efektif menuntut penggabungan yang disengaja antara
inkuiri ilmiah, pemanfaatan teknologi, desain rekayasa, dan penalaran
matematis \\citep{kelley2016aconceptual,pitot2024establishinga,portilloblanco2025buildingan}. Lebih mutakhir, tuntutan global terhadap pendidikan
keberlanjutan menambahkan dimensi integrasi ketiga kompetensi ESD
mengharuskan guru mengintegrasikan isu keberlanjutan, pendekatan inkuiri
terhadap tantangan lingkungan, serta pemikiran evaluatif atas dilema
sosio-ilmiah dalam pembelajaran IPA \\citep{purwianingsih2022programfor,unesco2017item,vidal2025preservice}.

Meskipun setiap dimensi diakui penting, temuan empiris masih menunjukkan
literatur yang terpisah-pisah. Penelitian tentang pengembangan TPACK
pada calon guru IPA cukup banyak (misalnya, \citep{offermann2025technologicalpedagogical,salleh2025howpre,stinkenrosner2023technologyimplementation}), demikian pula studi
kompetensi integrasi STEM \\citep{tucker2024examiningstem,mansour2024scienceand}
dan, meski lebih terbatas, kemampuan pedagogis ESD \\citep{purwianingsih2022programfor,vidal2025preservice}. Namun, ketiga rumpun penelitian ini
umumnya berjalan sendiri-sendiri. Masih sedikit studi yang menelaah
bagaimana calon guru mengembangkan kemampuan mengintegrasikan TPACK,
STEM, dan ESD secara bersamaan, dan belum ada yang memodelkan hubungan
struktural ketiganya sebagai jalur mediasi melalui mana intervensi
pembelajaran memengaruhi kualitas RPP integratif.

Project-Based Learning (PjBL) secara teoretis merupakan pendekatan yang
menjanjikan untuk meningkatkan kompetensi calon guru dalam merancang RPP
integratif. PjBL dipahami sebagai inkuiri jangka panjang yang
terstruktur melalui pertanyaan pendorong autentik dan diakhiri dengan
produk yang dipublikasikan \\citep{krajcik2014item}. Sejumlah studi
menunjukkan bahwa PjBL efektif meningkatkan kompetensi tersebut secara
terpisah, misalnya peningkatan TPACK melalui scaffolding PjBL \\citep{dewi2022projectbased} dan peningkatan kualitas desain unit STEM interdisipliner
pada calon guru \\citep{pitot2024establishinga,portilloblanco2025buildingan}.
Namun, jalur pengaruh PjBL terhadap kualitas RPP integrative (apakah
langsung atau melalui mediasi peningkatan TPACK, STEM, dan ESD) masih
belum dibuktikan secara empiris.

Kesenjangan ini berdampak pada teori dan praktik. Secara teoretis, perlu
dipahami apakah PjBL meningkatkan kualitas RPP terutama melalui
penguatan dimensi integrasi (TPACK/STEM/ESD) atau melalui pengaruh
langsung. Temuan ini dapat memperjelas mekanisme bagaimana intervensi
pedagogis membentuk kompetensi desain, sekaligus menyempurnakan kerangka
\emph{Teacher Design Capacity} \\citep{brown2009item} dengan merinci jalur
pembentukan kompetensi tersebut. Secara praktis, pemetaan dimensi
integrasi yang paling responsif terhadap PjBL membantu guru menyesuaikan
fokus pembelajaran dan strategi \emph{scaffolding}, termasuk memberi
dukungan tambahan pada dimensi yang belum berkembang optimal melalui
PjBL saja.

Penelitian ini menutup kesenjangan dengan mengusulkan dan menguji model
struktural yang menempatkan PjBL sebagai prediktor eksogen bagi tiga
dimensi integrasi (TPACK, STEM, dan ESD), yang selanjutnya memengaruhi
kualitas RPP integratif. Investigasi ini dipandu oleh lima pertanyaan
penelitian:

\textbf{RQ1.} Bagaimana kompetensi integrasi TPACK, STEM, dan ESD calon
guru IPA dalam desain RPP berubah dari pre- ke post-intervensi PjBL?

\textbf{RQ2.} Apakah implementasi PjBL secara signifikan mempengaruhi
kualitas desain RPP di seluruh dimensi integrasi TPACK, STEM, dan ESD?

\textbf{RQ3.} Dimensi integrasi mana (TPACK, STEM, atau ESD) yang paling
kuat dipengaruhi oleh PjBL?

\textbf{RQ4.} Bagaimana kompetensi integrasi TPACK, STEM, dan ESD
berkontribusi terhadap kualitas RPP integratif secara keseluruhan
setelah implementasi PjBL?

\textbf{RQ5.} Apakah peningkatan TPACK, STEM, dan ESD memediasi pengaruh
PjBL terhadap kualitas RPP integratif?

Penelitian ini memberi kontribusi pada literatur dalam tiga hal.
Pertama, penelitian ini menjadi uji empiris awal atas model integratif
yang menempatkan TPACK, STEM, dan ESD secara simultan sebagai mediator
hubungan antara intervensi pedagogis dan kualitas RPP. Kedua, kualitas
RPP integratif diukur sebagai konstruk tingkat tinggi melalui penilaian
rubrik (bukan self-report), sehingga pengukuran lebih kontekstual dan
valid. Ketiga, melalui PLS-SEM untuk memetakan efek langsung, tidak
langsung, dan total, penelitian ini menjelaskan secara rinci mekanisme
bagaimana PjBL membentuk kompetensi calon guru IPA dalam mendesain
pembelajaran, dengan implikasi langsung bagi perancangan kurikulum di
lembaga pendidikan guru.
