\section{Conclusion}\label{conclusion}

Penelitian ini menyelidiki pengaruh \emph{Project-Based Learning}
terhadap kompetensi calon guru IPA dalam merencanakan pembelajaran
integratif. Pembelajaran integratif yang dimaksud, didefinisikan sebagai
integrasi simultan dimensi TPACK, STEM, dan ESD dalam desain rencana
pembelajaran. Analisis dilakukan menggunakan uji berpasangan, N-Gain,
dan PLS-SEM untuk menjawab lima pertanyaan penelitian (RQ1--RQ5) serta
menguji tiga hipotesis (H1--H3). Ringkasan temuan utama adalah sebagai
berikut.

RQ1: Keempat konstruk (TPACK, STEM, ESD, dan kualitas rencana
pembelajaran integratif) meningkat secara signifikan setelah intervensi
PjBL, dengan \emph{effect sizes} yang konsisten besar dan nilai N-Gain
pada kategori Sedang. TPACK dan STEM menunjukkan respons peningkatan
paling tinggi. Sebaliknya, ESD memiliki N-Gain paling rendah, tidak ada
partisipan yang mencapai kategori N-Gain Tinggi, dan 28,4\% partisipan
tetap berada pada kategori Rendah.

RQ2: Analisis PLS-SEM menunjukkan bahwa PjBL secara signifikan
memprediksi ketiga dimensi integrasi (TPACK, STEM, dan ESD), semuanya
dengan \emph{effect sizes} besar. Dengan demikian, H1 didukung
sepenuhnya.

RQ3: Dari ketiga dimensi, TPACK adalah yang paling kuat dipengaruhi oleh
PjBL, disusul STEM dan kemudian ESD. Urutan pengaruh (TPACK
> STEM > ESD) ditetapkan berdasarkan koefisien
jalur dan \emph{effect sizes}. Temuan ini menunjukkan bahwa karakter
PjBL yang berorientasi teknologi dan menekankan penciptaan artefak lebih
efektif dalam menguatkan \emph{technological pedagogical content
knowledge}.

RQ4: Ketiga dimensi integrasi berkontribusi signifikan terhadap kualitas
rencana pembelajaran integratif dengan \emph{effect sizes} besar, yaitu
STEM, TPACK, dan ESD. Oleh karena itu, H2 didukung, sekaligus memperkuat
konseptualisasi perencanaan pembelajaran integratif sebagai konstruk
tingkat tinggi.

RQ5: PjBL tidak berpengaruh langsung terhadap kualitas rencana
pembelajaran, tetapi berpengaruh melalui peningkatan tiga kompetensi
integrasi. Tiga jalur mediasi yang signifikan adalah melalui STEM,
TPACK, dan ESD. Secara keseluruhan, efek tidak langsung tergolong kuat,
sedangkan efek langsung tidak signifikan; sehingga H3 didukung
sepenuhnya.

Sebagai penutup, penelitian ini memberikan bukti yang robust bahwa PjBL
efektif dalam meningkatkan kompetensi calon guru IPA dalam merancang
pembelajaran integratif. Temuan mediasi penuh melalui jalur TPACK, STEM,
dan ESD tidak hanya memperdalam pemahaman teoretis tentang bagaimana
intervensi pedagogis berubah menjadi kompetensi desain, tetapi juga
memberikan arahan praktis bagi program pendidikan guru dalam membangun
keterampilan integrasi multidimensional pada calon guru IPA.

\textbf{Acknowledgements}

The author(s) express gratitude to the students who were involved in
this study.

\textbf{Funding Statement}

This research was funded by Universitas Negeri Semarang Budget
Implementation List (DIPA), Number: T/237/UN37/HK.02/2024.

\textbf{CRediT authorship contribution} \textbf{statement}

\textbf{Novi Ratna Dewi}: Conceptualization, Writing -- original draft,
Methodology, Investigation, Formal analysis, Writing -- review \&
editing, Supervision. \textbf{Rizki Nor Amelia}: Investigation, Data
curation, Writing -- review \& editing, Project administration.
\textbf{Septiko Aji}: Data curation, Visualization. \textbf{Ismail Okta
Kurniawan}: Data curation, Software.

\textbf{Declaration of generative AI and AI-assisted technologies in the
writing process}

During the preparation of this work, the author used ChatGPT (OpenAI) to
expand the search for relevant references from their existing
collection, DeepL to translate text from Indonesian to English, and
Grammarly to improve language and readability. After using this tool,
the author reviewed and edited the content as needed and takes full
responsibility for the content of the publication.

