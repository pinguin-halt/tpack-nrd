\section{Literature review}\label{literature-review}

\subsection{Project-based learning dalam pendidikan
guru}\label{project-based-learning-dalam-pendidikan-guru}

Project-Based Learning (PjBL) adalah pendekatan pembelajaran yang
berpusat pada pertanyaan autentik dan kompleks untuk mendorong
investigasi berkelanjutan, inkuiri kolaboratif, serta menghasilkan
artefak nyata \\citep{krajcik2014item}. Elemen kunci PjBL mencakup
pertanyaan pendorong berbasis masalah dunia nyata, inkuiri melalui
penelitian, kolaborasi, produksi artefak yang dibagikan secara publik,
dan refleksi terstruktur \\citep{bell2010projectbased,krajcik2014item}. Sejumlah
studi mendukung efektivitasnya dalam meningkatkan kompetensi calon guru,
termasuk TPACK dan kemampuan desain pembelajaran melalui scaffolding
\\citep{dewi2022projectbased}, kompetensi pedagogis melalui pembelajaran proyek
\\citep{novallyan2025optimizationof}, serta kompetensi terkait teknologi \\citep{akbulut2021developingpre}. Temuan lain menunjukkan PjBL/PBL cenderung memperkuat
kolaborasi, produksi artefak, dan praktik inkuiri, meski elemen seperti
pertanyaan pendorong yang sepenuhnya dihasilkan siswa masih menantang
\\citep{markula2022thekey}, dan penelitian terbaru juga melaporkan
peningkatan pada kualitas desain unit STEM interdisipliner serta design
thinking calon guru \\citep{pitot2024establishinga,portilloblanco2025buildingan,yuksel2025designbased}. Secara umum, studi-studi tersebut menempatkan PjBL
sebagai intervensi yang layak untuk meningkatkan berbagai dimensi
kompetensi guru, namun pengaruh simultannya terhadap banyak dimensi
integrasi dalam satu model struktural masih jarang diteliti.

\subsection{Technological pedagogical content knowledge
(tpack)}\label{technological-pedagogical-content-knowledge-tpack}

Kerangka TPACK yang diperkenalkan oleh \citet{mishra2006technologicalpedagogical}, dengan
dasar konsep Pedagogical Content Knowledge (PCK) dari \citet{shulman1986thosewho},
menjelaskan pengetahuan yang saling beririsan untuk mendukung pengajaran
efektif berbantuan teknologi. TPACK mencakup tujuh komponen: Technology
Knowledge (TK), Pedagogical Knowledge (PK), Content Knowledge (CK), tiga
irisan berpasangan (TPK, TCK, PCK), serta inti integratif TPACK yang
merepresentasikan kemampuan mengajarkan konten spesifik menggunakan
teknologi yang tepat melalui strategi pedagogis yang sesuai.

Pengukuran TPACK bergeser dari survei laporan diri menuju asesmen
berbasis kinerja, misalnya evaluasi rencana pembelajaran dengan rubrik
yang menangkap integrasi teknologi--pedagogi--konten \\citep{offermann2025technologicalpedagogical}; Sejumlah studi SEM terbaru menunjukkan keterkaitan TPACK dengan
konstruk lain: \\citet{mansour2024scienceand} menemukan hubungan signifikan
antara komponen TPACK dan efikasi pengajaran STEM, \\citet{salleh2025howpre}
menegaskan peran pengetahuan konten dan efikasi diri teknologi sebagai
prediktor, dan \\citet{stinkenrosner2023technologyimplementation} melaporkan modul
implementasi teknologi yang terstruktur dapat meningkatkan TPACK serta
orientasi perilaku penggunaan teknologi. Namun, TPACK masih jarang
dimodelkan secara simultan bersama STEM dan ESD dalam satu kerangka
struktural, sehingga menjadi celah yang ingin dijembatani dalam
penelitian ini.

\subsection{Pendidikan STEM dan desain pembelajaran
integratif}\label{pendidikan-stem-dan-desain-pembelajaran-integratif}

Pendidikan STEM dipahami sebagai integrasi yang disengaja antara sains,
teknologi, teknik, dan matematika, dan kini semakin kuat menjadi
filosofi pendidikan sekaligus kerangka kurikulum \\citep{kelley2016aconceptual}. Perbedaan antara STEM yang diajarkan terpisah per disiplin dan
STEM integrative berdampak langsung pada desain rencana pembelajaran.
\\citet{kelley2016aconceptual} menawarkan kerangka STEM terintegrasi yang
menekankan pembelajaran situated, desain teknik sebagai strategi
pedagogis, inkuiri ilmiah sebagai proses membangun pengetahuan, serta
pemikiran matematis sebagai dasar analitis; kualitas rencana
pembelajaran dapat ditinjau dari akurasi-kedalaman konten sains,
integrasi teknologi yang purposif, hadirnya pemikiran desain teknik, dan
penerapan penalaran matematis.

Bagi calon guru IPA, kompetensi integrasi STEM menuntut pengembangan
efikasi diri STEM (keyakinan dan kemampuan merancang pembelajaran
interdisipliner (\\citep{tucker2024examiningstem}) yang dapat diperkuat melalui
perencanaan kolaboratif lintas mata kuliah dan integrasi eksplisit
engineering design dalam kerangka PBL-STEM \\citep{pitot2024establishinga,portilloblanco2025buildingan}, serta latihan desain berkelanjutan
berbasis umpan balik \\citep{wu2021howto}. Tantangan ini cenderung lebih
berat pada integrasi yang terkait ESD karena calon guru perlu memasukkan
konteks keberlanjutan yang sering kurang familiar dibanding topik STEM
tradisional.

\subsection{Education for sustainable development (ESD) dalam pengajaran
IPA}\label{education-for-sustainable-development-esd-dalam-pengajaran-ipa}

Education for Sustainable Development (ESD) merupakan paradigma
pendidikan global yang membekali peserta didik dengan pengetahuan,
keterampilan, nilai, dan sikap untuk merespons tantangan keberlanjutan
yang saling terhubung \\citep{unesco2017item}. Dalam pendidikan guru IPA, ESD
menuntut kompetensi yang melampaui pedagogi IPA tradisional, yaitu
kemampuan mengaitkan konsep ilmiah dengan isu keberlanjutan (ESD-PCK),
menerapkan pendekatan inkuiri untuk mengkaji persoalan lingkungan dan
sosial (ESD-INQ), serta menumbuhkan pemikiran evaluatif terhadap dilema
sosio-ilmiah dan berbagai trade-off (ESD-EVA) \\citep{unece2012item}.

Sejumlah studi menunjukkan upaya integrasi ESD pada pendidikan calon
guru, misalnya program \\citet{purwianingsih2022programfor} yang mengintegrasikan
ESD ke dalam TPACK calon guru biologi. \\citet{shumba2013mainstreamingesd} juga
menegaskan bahwa guru memerlukan pengetahuan pedagogis-konten spesifik
ESD. Namun, integrasi ESD masih relatif kurang berkembang dibandingkan
TPACK dan integrasi STEM, termasuk dalam desain rencana pembelajaran;
selain itu, action competence ESD calon guru cenderung masih lemah dan
sering diukur lewat self-report potong lintang, sehingga asesmen
berbasis kinerja dan tugas desain menjadi penting \\citep{vidal2025preservice,singhpillay2023preservice}. Temuan lain menunjukkan pengetahuan SDG calon
guru pada pelatihan awal juga sering terbatas, sehingga dukungan
kurikuler ESD yang eksplisit tetap diperlukan \\citep{calero2024astudy};
implikasinya, kompetensi ESD kemungkinan memerlukan scaffolding yang
lebih intensif atau lebih panjang, dan responsnya terhadap intervensi
PjBL jangka pendek dapat berbeda.

\subsection{Perencanaan pembelajaran integratif sebagai kompetensi
desain
guru}\label{perencanaan-pembelajaran-integratif-sebagai-kompetensi-desain-guru}

Konsep \emph{teacher design capacity} yang diperkenalkan oleh \citet{brown2009item} dan dielaborasi oleh \citet{mckenney2015teacherdesign} menjadi landasan
teoretis untuk memandang perencanaan pembelajaran integratif sebagai
kompetensi profesional tingkat tinggi. \\citet{brown2009item} menekankan bahwa
penggunaan kurikulum yang efektif menuntut guru berperan sebagai
desainer yang secara aktif menafsirkan, menyesuaikan, dan mengembangkan
materi pembelajaran sesuai konteks; kapasitas ini bersifat dinamis dan
berkembang melalui keterlibatan dalam tugas desain serta umpan balik.
\\citet{mckenney2015teacherdesign} menambahkan bahwa kompetensi desain guru ditopang
oleh basis pengetahuan spesifik (pengetahuan teknologi, pedagogis, dan
materi subjek) yang terbentuk dari interaksi antara pengetahuan
personal, pengetahuan formal, dan pengalaman praktik.

Dalam studi ini, kompetensi perencanaan pembelajaran integratif
dioperasionalisasikan sebagai kualitas rencana pembelajaran yang secara
simultan mengintegrasikan dimensi TPACK, STEM, dan ESD. Rencana
pembelajaran dipahami sebagai konstruk tingkat tinggi
(\emph{higher-order construct}, HOC) yang merepresentasikan kualitas
emergen dari integrasi yang koheren antardimensi, bukan sekadar
penjumlahan skor komponen. Karena itu, asesmen berbasis rubrik (bukan
laporan diri) digunakan untuk menangkap kompetensi desain yang
benar-benar didemonstrasikan, bukan yang hanya dipersepsikan.

\subsection{Kerangka konseptual dan model yang
dihipotesiskan}\label{kerangka-konseptual-dan-model-yang-dihipotesiskan}

Berdasarkan landasan teoretis tersebut, studi ini mengusulkan sebuah
model struktural yang menempatkan PjBL sebagai konstruk eksogen yang
memengaruhi tiga konstruk endogen orde pertama, yaitu kompetensi
integrasi TPACK, STEM, dan ESD, yang selanjutnya berkontribusi pada
kualitas rencana pembelajaran integratif (lihat Gambar 1). Model ini
mengasumsikan adanya efek langsung (PjBL -> TPACK, PjBL
-> STEM, PjBL -> ESD) serta efek tidak
langsung (PjBL -> TPACK/STEM/ESD -> Kualitas
Rencana Pembelajaran Integratif). Dalam kerangka ini, ketiga dimensi
integrasi tersebut dihipotesiskan berperan sebagai mediator antara PjBL
dan kualitas rencana pembelajaran.

\begin{figure}
\centering
\includegraphics[width=5.24934in,height=1.72792in,alt={Model struktural yang dihipotesiskan}]{figures/media/model.png}
\caption{Model struktural yang dihipotesiskan}
\end{figure}

\emph{Gambar 1.} Model struktural yang dihipotesiskan.

PjBL mempengaruhi tiga dimensi integrasi (TPACK, STEM, ESD) yang
kemudian berkontribusi terhadap kualitas rencana pembelajaran
integratif. Garis putus-putus menunjukkan jalur langsung PjBL ke RPP,
yang dihipotesiskan tidak signifikan (mediasi penuh).

Justifikasi teoretis untuk setiap jalur adalah sebagai berikut:

\begin{enumerate}
\def\labelenumi{\arabic{enumi}.}
\item
  PjBL -> TPACK: Penekanan PjBL pada penciptaan artefak dan
  investigasi berbasis inkuiri secara alami memerlukan mobilisasi
  teknologi untuk pedagogi spesifik konten \\citep{dewi2022projectbased}.
\item
  PjBL -> STEM: Pertanyaan pendorong autentik PjBL biasanya
  mencakup berbagai disiplin STEM, memerlukan pemikiran desain
  interdisipliner \\citep{krajcik2014item}.
\item
  PjBL -> ESD: Fokus PjBL pada masalah dunia nyata
  menciptakan peluang untuk mengatasi isu keberlanjutan, meskipun
  kekuatan tautan ini mungkin bergantung pada scaffolding eksplisit
  \\citep{purwianingsih2022programfor}.
\item
  TPACK/STEM/ESD -> Kualitas Rencana Pembelajaran
  Integratif: Setiap dimensi menyumbang elemen desain
  substantif---integrasi teknologi, koneksi interdisipliner, dan
  perspektif keberlanjutan---yang secara kolektif menentukan kualitas
  rencana pembelajaran integratif \\citep{brown2009item,mckenney2015teacherdesign}.
\item
  Mediasi: PjBL dihipotesiskan mempengaruhi kualitas rencana
  pembelajaran tidak secara langsung tetapi melalui peningkatan
  kompetensi integrasi, konsisten dengan pandangan bahwa intervensi
  pedagogis beroperasi dengan mengembangkan basis pengetahuan
  profesional spesifik yang kemudian termanifestasi dalam kinerja
  desain.
\end{enumerate}

Hipotesis berikut dirumuskan berdasarkan kerangka teoretis

H1: PjBL secara positif dan signifikan mempengaruhi kompetensi integrasi
TPACK, STEM, dan ESD.

H2: Kompetensi integrasi TPACK, STEM, dan ESD secara signifikan
berkontribusi terhadap kualitas RPP integratif.

H3: TPACK, STEM, dan ESD memediasi hubungan antara PjBL dan kualitas RPP
integratif.
