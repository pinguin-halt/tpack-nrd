\section{Methods}\label{methods}

\subsection{Desain penelitian}\label{desain-penelitian}

Penelitian ini menerapkan desain pra-eksperimental satu kelompok
pretest--posttest \\citep{creswell2018item}. Partisipan menyusun
rencana pembelajaran sebelum dan sesudah intervensi PjBL; kualitasnya
dinilai dengan rubrik terstandar pada tiga dimensi integrasi, yaitu
TPACK, STEM, dan ESD. Desain ini dipilih karena penelitian berfokus pada
pendokumentasian perubahan kompetensi setelah intervensi serta pemodelan
hubungan struktural antara kualitas implementasi PjBL, dimensi
integrasi, dan kualitas rencana pembelajaran menggunakan PLS-SEM, bukan
pada pembuktian kausal yang ketat melalui perbandingan antar-kelompok.

\subsection{Partisipan}\label{partisipan}

Partisipan penelitian ini terdiri atas 95 calon guru IPA pada program
studi Pendidikan IPA sebuah universitas Pendidikan negeri di Semarang
Indonesia. Seluruh partisipan merupakan mahasiswa sarjana tahun ketiga
atau keempat yang telah menuntaskan mata kuliah dasar konten sains,
pedagogi umum, dan teknologi pendidikan. Sampel dipilih secara purposif,
yakni mahasiswa yang sekaligus mengambil mata kuliah Strategi dan Desain
Pembelajaran IPA sebagai konteks natural pelaksanaan intervensi PjBL.
Ukuran sampel (N=95) melampaui ambang minimum PLS-SEM; \\citet{hair2022item} merekomendasikan minimal 10 kali jumlah maksimum jalur struktural
yang menuju satu konstruk (pada model ini, empat jalur menuju konstruk
RPP, sehingga minimum 40).

\subsection{Intervensi: implementasi
PjBL}\label{intervensi-implementasi-pjbl}

Intervensi PjBL dilaksanakan selama 16 kali pertemuan melalui sesi yang
terstruktur. Partisipan mengikuti siklus proyek untuk menyusun rencana
pembelajaran integratif yang memadukan TPACK, STEM, dan ESD secara
simultan. Intervensi mengacu pada lima tahap yang diadaptasi dari
\\citet{krajcik2014item}: (1) Orientasi dan pertanyaan pendorong; (2)
Perencanaan dan investigasi; (3) Penciptaan artefak. (4) Peer review dan
revisi; (5) Presentasi dan refleksi.

Kualitas implementasi PjBL dievaluasi melalui instrumen observasi yang
diisi oleh 10 observer.

\subsection{Instrumen}\label{instrumen}

\subsection{Rubrik rencana pembelajaran integratif
(pretest-posttest)}\label{rubrik-rencana-pembelajaran-integratif-pretest-posttest}

Instrumen utama adalah rubrik untuk mengevaluasi kualitas rencana
pembelajaran integratif, diskor pada skala Likert empat poin (1 = tidak
memenuhi kriteria, 2 = sebagian memenuhi, 3 = memenuhi, 4 = melampaui
kriteria). Rubrik menilai tiga dimensi integrasi yang terdiri dari 14
indikator:

\begin{enumerate}
\def\labelenumi{\alph{enumi}.}
\item
  TPACK (7 indikator): TK, PK, CK, TPK, TCK, PCK, TPACK integratif.
\item
  STEM (4 indikator): Integrasi konten Sains (S), Aplikasi Teknologi
  (T), Proses desain Teknik/Engineering (E), dan Penalaran Matematis
  (M).
\item
  ESD (3 indikator): ESD-Pedagogical Content Knowledge (ESD-PCK),
  ESD-Inquiry (ESD-INQ), dan ESD-Evaluative thinking (ESD-EVA).
\end{enumerate}

Skor komposit setiap dimensi dihitung sebagai rata-rata (mean) dari
indikator penyusunnya. Skor kualitas keseluruhan rencana pembelajaran
integratif (RPPInt\_total) dihitung sebagai grand mean dari 14
indikator. Rubrik disusun melalui penilaian ahli oleh lima spesialis
pendidikan IPA dan menunjukkan validitas konten yang memadai.

\subsection{Instrumen observasi implementasi
PjBL}\label{instrumen-observasi-implementasi-pjbl}

Kualitas implementasi PjBL diukur menggunakan instrumen observasi lima
item, dengan setiap item diskor pada skala 1--4 yang sesuai dengan lima
tahap PjBL. Instrumen diisi oleh instruktur mata kuliah yang
mengobservasi proses implementasi.

\subsection{Prosedur pengumpulan data}\label{prosedur-pengumpulan-data}

Pengumpulan data mengikuti timeline tiga tahap: (a) pretest; (b)
intervensi PjBL; dan (c) posttest. Rencana pembelajaran dianonimkan
sebelum penskoran untuk mengurangi bias penilai.

\subsection{Analisis data}\label{analisis-data}

Analisis data berlangsung dalam dua fase.

\begin{enumerate}
\def\labelenumi{\alph{enumi}.}
\tightlist
\item
  Fase 1: Perbandingan pre-post (RQ1)
\end{enumerate}

Analisis yang dilakukan untuk menjawab RQ1: Statistik deskriptif (mean,
standar deviasi, minimum, maksimum); Uji normalitas (uji Shapiro-Wilk),
Uji inferensial berpasangan (paired-samples t-test untuk data
terdistribusi normal, dan uji Wilcoxon signed-rank untuk data
terdistribusi tidak normal dengan \(\alpha = 0,05)\); Effect sizes
(Cohen's \(d\) untuk uji parametrik dan korelasi rank-biserial \(r\)
untuk uji non-parametrik). Effect sizes diinterpretasikan mengikuti
\\citet{cohen1988item}: kecil (\(d = 0,2\)), sedang (\(d = 0,5\)), dan besar
(\(d = 0,8\)). Normalized gain (N-Gain) dihitung menggunakan formula
\\citet{hake1998interactiveengagement}: N-Gain \(= \frac{post - pre}{\max - pre}\), di mana max
\(= 4\) (skor rubrik maksimum). Nilai N-Gain dikategorikan sebagai
Tinggi (\(> 0,7\)), Sedang (\(0,3\)--\(0,7\)), atau Rendah (\(< 0,3\)).
Seluruh analisis Fase 1 dilakukan di Python 3.11 menggunakan pandas,
scipy, dan pingouin.

\begin{enumerate}
\def\labelenumi{\alph{enumi}.}
\setcounter{enumi}{1}
\tightlist
\item
  Fase 2: Structural equation modeling (RQ2--RQ5)
\end{enumerate}

Partial Least Squares Structural Equation Modeling (PLS-SEM) digunakan
untuk menjawab RQ2 sampai RQ5 karena: (a) sifat
eksploratori-konfirmatori penelitian; (b) inklusi konstruk indikator
tunggal; (c) ukuran sampel moderat (\(N = 95\)); dan (d) kapasitas
PLS-SEM untuk menangani data non-normal dan pengukuran formatif \\citep{hair2022item}.

\emph{Spesifikasi model.} Model struktural terdiri dari lima konstruk:
PjBL (eksogen), TPACK, STEM, ESD, dan RPP (endogen). Seluruh konstruk
dispesifikasikan sebagai reflektif (Mode A). Jalur struktural mencakup
tujuh hubungan: PjBL \(\rightarrow\) TPACK, PjBL \(\rightarrow\) STEM,
PjBL \(\rightarrow\) ESD, PjBL \(\rightarrow\) RPP (langsung), TPACK
\(\rightarrow\) RPP, STEM \(\rightarrow\) RPP, dan ESD \(\rightarrow\)
RPP. Konstruk RPP dioperasionalisasikan sebagai konstruk indikator
tunggal menggunakan skor kualitas rencana pembelajaran integratif
komposit (RPPInt\_total\_post), dengan loading indikator difiksasi ke
1,000. Matriks data untuk analisis SEM menggunakan skor posttest untuk
TPACK, STEM, ESD, dan RPP, serta skor observasi PjBL, menghasilkan 19
variabel manifes.

\emph{Evaluasi model pengukuran.} Model outer (pengukuran) dinilai
menggunakan kriteria PLS-SEM standar \\citep{hair2022item}:

\begin{enumerate}
\def\labelenumi{\alph{enumi}.}
\item
  Reliabilitas indikator: outer loadings \(\geq 0,708\) (indikator
  antara 0,40 dan 0,70 dipertahankan jika penghapusannya tidak
  meningkatkan AVE atau CR, mengikuti rekomendasi \\citet{hair2022item}
  untuk
  penelitian eksploratori).
\item
  Validitas konvergen: Average Variance Extracted (AVE) \(\geq 0,50\).
\item
  Reliabilitas konsistensi internal: Composite Reliability (CR)
  \(\geq 0,70\) dan Cronbach's \(\alpha \geq 0,70\).
\item
  Validitas diskriminan: rasio Heterotrait-Monotrait (HTMT) \(< 0,90\)
  \\citep{henseler2015anew}, dilengkapi dengan kriteria Fornell-Larcker.
\end{enumerate}

\emph{Evaluasi model struktural.} Model inner (struktural) dinilai
melalui: Koefisien jalur (\(\beta\)); Signifikansi statistic; Koefisien
determinasi (\(R^{2}\)); Effect size (\(f^{2}\)); Relevansi prediktif
(\(Q^{2}\))

\emph{Analisis komparatif (RQ3).} Pengaruh relatif PjBL terhadap setiap
dimensi integrasi dinilai dengan membandingkan koefisien jalur dan nilai
\(f^{2}\) terkait.

\emph{Analisis mediasi (RQ5).} Efek tidak langsung dihitung sebagai
produk koefisien jalur penyusun. Signifikansi statistik efek tidak
langsung ditentukan melalui metode confidence interval bootstrap
\\citep{preacher2008asymptoticand}: efek tidak langsung dianggap signifikan jika
confidence interval percentile 95\% tidak mencakup nol. Uji Sobel juga
dihitung sebagai cross-check. Variance Accounted For (VAF) dihitung
untuk mengklasifikasikan tipe mediasi: mediasi penuh (VAF \(> 80\%\)),
mediasi parsial (\(20\% <\) VAF \(< 80\%\)), atau tanpa mediasi (VAF
\(< 20\%\)) \\citep{hair2022item}.

\emph{Perangkat lunak.} Analisis PLS-SEM dilakukan menggunakan paket
Python plspm (versi 0.5.7) dengan random seed 42 untuk reprodusibilitas.
Seluruh prosedur bootstrap menggunakan 5.000 iterasi dengan konfigurasi
single-process untuk memastikan eksekusi deterministik. Gambar
dihasilkan menggunakan matplotlib.
