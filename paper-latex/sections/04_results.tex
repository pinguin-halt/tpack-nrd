\section{Results}\label{results}

\subsection{Statistik deskriptif}\label{statistik-deskriptif}

Statistik deskriptif untuk seluruh konstruk dilakukan sebelum uji
hipotesis. Tabel 1 menyajikan rata-rata dan simpangan baku tingkat
konstruk pada pretest dan posttest.

\textbf{Tabel 1.} Statistik deskriptif per konstruk (pretest vs
posttest)

{\def\LTcaptype{none} % do not increment counter
\begin{longtable}[]{@{}
  >{\raggedright\arraybackslash}p{(\linewidth - 12\tabcolsep) * \real{0.1529}}
  >{\raggedright\arraybackslash}p{(\linewidth - 12\tabcolsep) * \real{0.1294}}
  >{\raggedright\arraybackslash}p{(\linewidth - 12\tabcolsep) * \real{0.1294}}
  >{\raggedright\arraybackslash}p{(\linewidth - 12\tabcolsep) * \real{0.1294}}
  >{\raggedright\arraybackslash}p{(\linewidth - 12\tabcolsep) * \real{0.1294}}
  >{\raggedright\arraybackslash}p{(\linewidth - 12\tabcolsep) * \real{0.1294}}
  >{\raggedright\arraybackslash}p{(\linewidth - 12\tabcolsep) * \real{0.1765}}@{}}
\toprule\noalign{}
\begin{minipage}[b]{\linewidth}\raggedright
Konstruk
\end{minipage} & \begin{minipage}[b]{\linewidth}\raggedright
\[N\]
\end{minipage} & \begin{minipage}[b]{\linewidth}\raggedright
Pre \(M\)
\end{minipage} & \begin{minipage}[b]{\linewidth}\raggedright
Pre \(SD\)
\end{minipage} & \begin{minipage}[b]{\linewidth}\raggedright
Post \(M\)
\end{minipage} & \begin{minipage}[b]{\linewidth}\raggedright
Post \(SD\)
\end{minipage} & \begin{minipage}[b]{\linewidth}\raggedright
\[M_{diff}\]
\end{minipage} \\
\midrule\noalign{}
\endhead
\bottomrule\noalign{}
\endlastfoot
TPACK & 95 & 2,306 & 0,390 & 3,319 & 0,294 & 1,013 \\
STEM & 95 & 2,172 & 0,393 & 3,222 & 0,378 & 1,051 \\
ESD & 95 & 1,929 & 0,328 & 2,726 & 0,259 & 0,798 \\
RPP Integratif & 95 & 2,136 & 0,256 & 3,089 & 0,237 & 0,954 \\
\end{longtable}
}

Sebagaimana ditunjukkan pada Tabel 1, seluruh konstruk menunjukkan
peningkatan substansial setelah intervensi PjBL (terlihat dari nilai
\(M_{diff})\). Perbedaan rata-rata terbesar diamati pada STEM, diikuti
TPACK, RPP Integratif, dan ESD.

\begin{figure}
\centering
\includegraphics[width=4.8in,alt={Gambar 2. Perbandingan Skor Rata-rata Pre-test vs Post-test per Konstruk}]{figures/media/fig1_pre_post_comparison.png}
\caption{Gambar 2. Perbandingan Skor Rata-rata Pre-test vs Post-test per Konstruk}
\end{figure}

\subsection{RM1: Perubahan pre-post}\label{rm1-perubahan-pre-post}

\subsection{Uji normalitas dan uji berpasangan dan ukuran
efek}\label{uji-normalitas-dan-uji-berpasangan-dan-ukuran-efek}

Normalitas dinilai menggunakan uji Shapiro-Wilk. Tabel 2 merangkum
hasilnya.

\textbf{Tabel 2.} Hasil uji normalitas shapiro-wilk untuk skor selisih

{\def\LTcaptype{none} % do not increment counter
\begin{longtable}[]{@{}
  >{\raggedright\arraybackslash}p{(\linewidth - 6\tabcolsep) * \real{0.1806}}
  >{\raggedright\arraybackslash}p{(\linewidth - 6\tabcolsep) * \real{0.1111}}
  >{\raggedright\arraybackslash}p{(\linewidth - 6\tabcolsep) * \real{0.1111}}
  >{\raggedright\arraybackslash}p{(\linewidth - 6\tabcolsep) * \real{0.1250}}@{}}
\toprule\noalign{}
\begin{minipage}[b]{\linewidth}\raggedright
Konstruk
\end{minipage} & \begin{minipage}[b]{\linewidth}\raggedright
\[W\]
\end{minipage} & \begin{minipage}[b]{\linewidth}\raggedright
\[p\]
\end{minipage} & \begin{minipage}[b]{\linewidth}\raggedright
Normal
\end{minipage} \\
\midrule\noalign{}
\endhead
\bottomrule\noalign{}
\endlastfoot
TPACK & 0,968 & ,020 & Tidak \\
STEM & 0,977 & ,088 & Ya \\
ESD & 0,990 & ,704 & Ya \\
RPP Integratif & 0,986 & ,418 & Ya \\
\end{longtable}
}

Sebagaimana ditunjukkan pada Tabel 2, skor selisih untuk STEM, ESD, dan
RPP Integratif terdistribusi normal (\(p > ,05\)). Namun, skor selisih
TPACK menyimpang signifikan dari normalitas. Oleh karena itu,
paired-samples t-test diterapkan pada STEM, ESD, dan RPP Integratif,
sementara Wilcoxon signed-rank test digunakan untuk TPACK.

\textbf{Tabel 3.} Hasil uji berpasangan

{\def\LTcaptype{none} % do not increment counter
\begin{longtable}[]{@{}
  >{\raggedright\arraybackslash}p{(\linewidth - 10\tabcolsep) * \real{0.1591}}
  >{\raggedright\arraybackslash}p{(\linewidth - 10\tabcolsep) * \real{0.1364}}
  >{\centering\arraybackslash}p{(\linewidth - 10\tabcolsep) * \real{0.2273}}
  >{\centering\arraybackslash}p{(\linewidth - 10\tabcolsep) * \real{0.1477}}
  >{\raggedright\arraybackslash}p{(\linewidth - 10\tabcolsep) * \real{0.1364}}
  >{\raggedright\arraybackslash}p{(\linewidth - 10\tabcolsep) * \real{0.1705}}@{}}
\toprule\noalign{}
\begin{minipage}[b]{\linewidth}\raggedright
Konstruk
\end{minipage} & \begin{minipage}[b]{\linewidth}\raggedright
Uji
\end{minipage} & \begin{minipage}[b]{\linewidth}\centering
Statistik
\end{minipage} & \begin{minipage}[b]{\linewidth}\centering
\[p\]
\end{minipage} & \begin{minipage}[b]{\linewidth}\raggedright
Cohen's \(d\)
\end{minipage} & \begin{minipage}[b]{\linewidth}\raggedright
Interpretasi
\end{minipage} \\
\midrule\noalign{}
\endhead
\bottomrule\noalign{}
\endlastfoot
TPACK & Wilcoxon & \(W = 0,0\) & \(< ,001\) & 2,705 & Besar \\
STEM & Paired t & \(t(94) = 25,90\) & \(< ,001\) & 2,657 & Besar \\
ESD & Paired t & \(t(94) = 22,79\) & \(< ,001\) & 2,338 & Besar \\
RPP Integratif & Paired t & \(t(94) = 40,88\) & \(< ,001\) & 4,194 &
Besar \\
\end{longtable}
}

Sebagaimana ditunjukkan pada Tabel 3, keempat konstruk menunjukkan
peningkatan signifikan secara statistik dari pretest ke posttest
(\(p < ,001\)). Ukuran efek seragam besar, dengan nilai Cohen's \(d\)
melebihi 2,3 untuk seluruh konstruk.

\subsection{Normalized gain}\label{normalized-gain}

Proporsi peningkatan maksimum yang dapat dicapai diukur menggunakan
normalized gain (N-Gain). Tabel 4 menyajikan ringkasan N-Gain, sementara
Gambar 3. memvisualisasikan distribusi kategori gain.

\textbf{Tabel 4.} Ringkasan N-Gain

{\def\LTcaptype{none} % do not increment counter
\begin{longtable}[]{@{}
  >{\raggedright\arraybackslash}p{(\linewidth - 12\tabcolsep) * \real{0.1605}}
  >{\raggedright\arraybackslash}p{(\linewidth - 12\tabcolsep) * \real{0.1358}}
  >{\raggedright\arraybackslash}p{(\linewidth - 12\tabcolsep) * \real{0.1358}}
  >{\raggedright\arraybackslash}p{(\linewidth - 12\tabcolsep) * \real{0.1358}}
  >{\raggedright\arraybackslash}p{(\linewidth - 12\tabcolsep) * \real{0.1358}}
  >{\raggedright\arraybackslash}p{(\linewidth - 12\tabcolsep) * \real{0.1358}}
  >{\raggedright\arraybackslash}p{(\linewidth - 12\tabcolsep) * \real{0.1358}}@{}}
\toprule\noalign{}
\begin{minipage}[b]{\linewidth}\raggedright
Konstruk
\end{minipage} & \begin{minipage}[b]{\linewidth}\raggedright
N-Gain \(M\)
\end{minipage} & \begin{minipage}[b]{\linewidth}\raggedright
\[SD\]
\end{minipage} & \begin{minipage}[b]{\linewidth}\raggedright
Kategori
\end{minipage} & \begin{minipage}[b]{\linewidth}\raggedright
High (\%)
\end{minipage} & \begin{minipage}[b]{\linewidth}\raggedright
Medium (\%)
\end{minipage} & \begin{minipage}[b]{\linewidth}\raggedright
Low (\%)
\end{minipage} \\
\midrule\noalign{}
\endhead
\bottomrule\noalign{}
\endlastfoot
TPACK & 0,596 & 0,160 & Medium & 26,3 & 72,6 & 1,1 \\
STEM & 0,574 & 0,179 & Medium & 24,2 & 69,5 & 6,3 \\
ESD & 0,376 & 0,129 & Medium & 0,0 & 71,6 & 28,4 \\
RPP Integratif & 0,513 & 0,103 & Medium & 3,2 & 95,8 & 1,1 \\
\end{longtable}
}

Sebagaimana ditunjukkan pada Tabel 4, seluruh konstruk mencapai kategori
Medium gain (\(0,3\)--\(0,7\)). TPACK memperoleh rata-rata N-Gain
tertinggi, diikuti STEM, RPP Integratif, dan ESD.

\begin{figure}
\centering
\includegraphics[width=3.57087in,height=2.08571in,alt={Gambar 3. Distribusi kategori N-Gain setiap konstruk}]{figures/media/fig2_ngain_distribution.png}
\caption{Gambar 3. Distribusi kategori N-Gain setiap konstruk}
\end{figure}

Gambar 3. Distribusi kategori N-Gain setiap konstruk

Analisis pre-post (RM1) menunjukkan bahwa PjBL secara signifikan
meningkatkan keempat konstruk dengan ukuran efek besar, meskipun ESD
menunjukkan gain terkecil. Untuk memahami hubungan struktural antar
konstruk ini, kami beralih ke analisis PLS-SEM yang menjawab RM2--RM5.

\subsection{RM2--RM5: Analisis PLS-SEM}\label{rm2rm5-analisis-pls-sem}

\subsection{Evaluasi model pengukuran}\label{evaluasi-model-pengukuran}

Evaluasi model pengukuran untuk memastikan reliabilitas dan validitas
yang memadai sebelum uji hipotesis struktural. Tabel 5 menyajikan outer
loadings untuk seluruh indikator.

\textbf{Tabel 5.} Outer loadings

{\def\LTcaptype{none} % do not increment counter
\begin{longtable}[]{@{}
  >{\raggedright\arraybackslash}p{(\linewidth - 6\tabcolsep) * \real{0.1528}}
  >{\raggedright\arraybackslash}p{(\linewidth - 6\tabcolsep) * \real{0.2083}}
  >{\raggedright\arraybackslash}p{(\linewidth - 6\tabcolsep) * \real{0.1389}}
  >{\raggedright\arraybackslash}p{(\linewidth - 6\tabcolsep) * \real{0.2361}}@{}}
\toprule\noalign{}
\begin{minipage}[b]{\linewidth}\raggedright
Konstruk
\end{minipage} & \begin{minipage}[b]{\linewidth}\raggedright
Indikator
\end{minipage} & \begin{minipage}[b]{\linewidth}\raggedright
Loading
\end{minipage} & \begin{minipage}[b]{\linewidth}\raggedright
\[\geq 0,708\]
\end{minipage} \\
\midrule\noalign{}
\endhead
\bottomrule\noalign{}
\endlastfoot
ESD & ESD-EVA & 0,780 & Ya \\
ESD & ESD-INQ & 0,894 & Ya \\
ESD & ESD-PCK & 0,867 & Ya \\
PjBL & PjBL01 & 0,815 & Ya \\
PjBL & PjBL02 & 0,804 & Ya \\
PjBL & PjBL03 & 0,835 & Ya \\
PjBL & PjBL04 & 0,850 & Ya \\
PjBL & PjBL05 & 0,840 & Ya \\
TPACK & TK & 0,689 & Tidak \\
TPACK & PK & 0,224 & Tidak \\
TPACK & CK & 0,673 & Tidak \\
TPACK & TPK & 0,760 & Ya \\
TPACK & TCK & 0,795 & Ya \\
TPACK & PCK & 0,400 & Tidak \\
TPACK & TPACK\_int & 0,759 & Ya \\
STEM & Science & 0,484 & Tidak \\
STEM & Technology & 0,713 & Ya \\
STEM & Engineering & 0,577 & Tidak \\
STEM & Mathematics & 0,916 & Ya \\
RPP & RPPInt\_total & 1,000 & Ya \\
\end{longtable}
}

Indikator dengan loading antara 0,40 dan 0,70 dipertahankan mengikuti
rekomendasi \\citet{hair2022item} untuk penelitian eksploratoris.

\textbf{Tabel 6a.} Reliabilitas konstruk dan validitas konvergen

{\def\LTcaptype{none} % do not increment counter
\begin{longtable}[]{@{}
  >{\raggedright\arraybackslash}p{(\linewidth - 6\tabcolsep) * \real{0.1528}}
  >{\raggedright\arraybackslash}p{(\linewidth - 6\tabcolsep) * \real{0.1111}}
  >{\raggedright\arraybackslash}p{(\linewidth - 6\tabcolsep) * \real{0.1111}}
  >{\raggedright\arraybackslash}p{(\linewidth - 6\tabcolsep) * \real{0.2083}}@{}}
\toprule\noalign{}
\begin{minipage}[b]{\linewidth}\raggedright
Konstruk
\end{minipage} & \begin{minipage}[b]{\linewidth}\raggedright
AVE
\end{minipage} & \begin{minipage}[b]{\linewidth}\raggedright
CR
\end{minipage} & \begin{minipage}[b]{\linewidth}\raggedright
Cronbach's alpha
\end{minipage} \\
\midrule\noalign{}
\endhead
\bottomrule\noalign{}
\endlastfoot
ESD & 0,720 & 0,886 & 0,807 \\
PjBL & 0,687 & 0,917 & 0,886 \\
RPP & 1,000 & 1,000 & --- \\
STEM & 0,480 & 0,814 & 0,694 \\
TPACK & 0,418 & 0,834 & 0,766 \\
\end{longtable}
}

Threshold: AVE \textgreater= 0,50, CR \textgreater= 0,70, alpha
\textgreater= 0,70. RPP adalah konstruk single-indicator.

\textbf{Tabel 6b.} Matriks heterotrait-monotrait (HTMT)

{\def\LTcaptype{none} % do not increment counter
\begin{longtable}[]{@{}
  >{\raggedright\arraybackslash}p{(\linewidth - 8\tabcolsep) * \real{0.1528}}
  >{\raggedright\arraybackslash}p{(\linewidth - 8\tabcolsep) * \real{0.1111}}
  >{\raggedright\arraybackslash}p{(\linewidth - 8\tabcolsep) * \real{0.1250}}
  >{\raggedright\arraybackslash}p{(\linewidth - 8\tabcolsep) * \real{0.1111}}
  >{\raggedright\arraybackslash}p{(\linewidth - 8\tabcolsep) * \real{0.1111}}@{}}
\toprule\noalign{}
\begin{minipage}[b]{\linewidth}\raggedright
Konstruk
\end{minipage} & \begin{minipage}[b]{\linewidth}\raggedright
PjBL
\end{minipage} & \begin{minipage}[b]{\linewidth}\raggedright
TPACK
\end{minipage} & \begin{minipage}[b]{\linewidth}\raggedright
STEM
\end{minipage} & \begin{minipage}[b]{\linewidth}\raggedright
ESD
\end{minipage} \\
\midrule\noalign{}
\endhead
\bottomrule\noalign{}
\endlastfoot
PjBL & --- & & & \\
TPACK & 0,837 & --- & & \\
STEM & 0,871 & 0,770 & --- & \\
ESD & 0,716 & 0,298 & 0,456 & --- \\
\end{longtable}
}

Seluruh nilai \(< 0,90\), mendukung validitas diskriminan.

\subsection{Model struktural: Efek langsung
(RM2)}\label{model-struktural-efek-langsung-rm2}

Uji hubungan struktural untuk menjawab RM2 dilakukan setelah model
pengukuran terkonfirmasi. Tabel 7 menyajikan koefisien jalur, statistik
bootstrap (5.000 iterasi, seed = 42), dan ukuran efek untuk seluruh
jalur langsung.

\textbf{Tabel 7.} Model struktural --- koefisien jalur dan signifikansi

{\def\LTcaptype{none} % do not increment counter
\begin{longtable}[]{@{}
  >{\raggedright\arraybackslash}p{(\linewidth - 26\tabcolsep) * \real{0.1194}}
  >{\raggedright\arraybackslash}p{(\linewidth - 26\tabcolsep) * \real{0.0597}}
  >{\raggedright\arraybackslash}p{(\linewidth - 26\tabcolsep) * \real{0.0597}}
  >{\raggedright\arraybackslash}p{(\linewidth - 26\tabcolsep) * \real{0.0597}}
  >{\raggedright\arraybackslash}p{(\linewidth - 26\tabcolsep) * \real{0.0448}}
  >{\raggedright\arraybackslash}p{(\linewidth - 26\tabcolsep) * \real{0.0448}}
  >{\raggedright\arraybackslash}p{(\linewidth - 26\tabcolsep) * \real{0.0448}}
  >{\raggedright\arraybackslash}p{(\linewidth - 26\tabcolsep) * \real{0.0597}}
  >{\centering\arraybackslash}p{(\linewidth - 26\tabcolsep) * \real{0.0597}}
  >{\raggedright\arraybackslash}p{(\linewidth - 26\tabcolsep) * \real{0.0896}}
  >{\raggedright\arraybackslash}p{(\linewidth - 26\tabcolsep) * \real{0.0597}}
  >{\raggedright\arraybackslash}p{(\linewidth - 26\tabcolsep) * \real{0.0597}}
  >{\raggedright\arraybackslash}p{(\linewidth - 26\tabcolsep) * \real{0.0896}}
  >{\raggedright\arraybackslash}p{(\linewidth - 26\tabcolsep) * \real{0.0597}}@{}}
\toprule\noalign{}
\multicolumn{2}{@{}>{\raggedright\arraybackslash}p{(\linewidth - 26\tabcolsep) * \real{0.1791} + 2\tabcolsep}}{%
\begin{minipage}[b]{\linewidth}\raggedright
Jalur
\end{minipage}} &
\multicolumn{2}{>{\raggedright\arraybackslash}p{(\linewidth - 26\tabcolsep) * \real{0.1194} + 2\tabcolsep}}{%
\begin{minipage}[b]{\linewidth}\raggedright
\[\beta\]
\end{minipage}} &
\multicolumn{2}{>{\raggedright\arraybackslash}p{(\linewidth - 26\tabcolsep) * \real{0.0896} + 2\tabcolsep}}{%
\begin{minipage}[b]{\linewidth}\raggedright
\[SE\]
\end{minipage}} &
\multicolumn{2}{>{\raggedright\arraybackslash}p{(\linewidth - 26\tabcolsep) * \real{0.1045} + 2\tabcolsep}}{%
\begin{minipage}[b]{\linewidth}\raggedright
\[t\]
\end{minipage}} & \begin{minipage}[b]{\linewidth}\centering
\[p\]
\end{minipage} & \begin{minipage}[b]{\linewidth}\raggedright
CI 95\%
\end{minipage} &
\multicolumn{2}{>{\raggedright\arraybackslash}p{(\linewidth - 26\tabcolsep) * \real{0.1194} + 2\tabcolsep}}{%
\begin{minipage}[b]{\linewidth}\raggedright
Sig.
\end{minipage}} & \begin{minipage}[b]{\linewidth}\raggedright
\[f^{2}\]
\end{minipage} & \begin{minipage}[b]{\linewidth}\raggedright
\end{minipage} \\
\midrule\noalign{}
\endhead
\bottomrule\noalign{}
\endlastfoot
PjBL \(\rightarrow\) TPACK &
\multicolumn{2}{>{\raggedright\arraybackslash}p{(\linewidth - 26\tabcolsep) * \real{0.1194} + 2\tabcolsep}}{%
0,727} &
\multicolumn{2}{>{\raggedright\arraybackslash}p{(\linewidth - 26\tabcolsep) * \real{0.1045} + 2\tabcolsep}}{%
0,055} &
\multicolumn{2}{>{\raggedright\arraybackslash}p{(\linewidth - 26\tabcolsep) * \real{0.0896} + 2\tabcolsep}}{%
13,295} &
\multicolumn{2}{>{\raggedright\arraybackslash}p{(\linewidth - 26\tabcolsep) * \real{0.1194} + 2\tabcolsep}}{%
\(< ,001\)} & {[}0,610; 0,823{]} & Ya &
\multicolumn{3}{>{\raggedright\arraybackslash}p{(\linewidth - 26\tabcolsep) * \real{0.2090} + 4\tabcolsep}@{}}{%
1,123 (Big)} \\
PjBL \(\rightarrow\) STEM &
\multicolumn{2}{>{\raggedright\arraybackslash}p{(\linewidth - 26\tabcolsep) * \real{0.1194} + 2\tabcolsep}}{%
0,683} &
\multicolumn{2}{>{\raggedright\arraybackslash}p{(\linewidth - 26\tabcolsep) * \real{0.1045} + 2\tabcolsep}}{%
0,054} &
\multicolumn{2}{>{\raggedright\arraybackslash}p{(\linewidth - 26\tabcolsep) * \real{0.0896} + 2\tabcolsep}}{%
12,616} &
\multicolumn{2}{>{\raggedright\arraybackslash}p{(\linewidth - 26\tabcolsep) * \real{0.1194} + 2\tabcolsep}}{%
\(< ,001\)} & {[}0,573; 0,782{]} & Ya &
\multicolumn{3}{>{\raggedright\arraybackslash}p{(\linewidth - 26\tabcolsep) * \real{0.2090} + 4\tabcolsep}@{}}{%
0,872 (Big)} \\
PjBL \(\rightarrow\) ESD &
\multicolumn{2}{>{\raggedright\arraybackslash}p{(\linewidth - 26\tabcolsep) * \real{0.1194} + 2\tabcolsep}}{%
0,617} &
\multicolumn{2}{>{\raggedright\arraybackslash}p{(\linewidth - 26\tabcolsep) * \real{0.1045} + 2\tabcolsep}}{%
0,065} &
\multicolumn{2}{>{\raggedright\arraybackslash}p{(\linewidth - 26\tabcolsep) * \real{0.0896} + 2\tabcolsep}}{%
9,496} &
\multicolumn{2}{>{\raggedright\arraybackslash}p{(\linewidth - 26\tabcolsep) * \real{0.1194} + 2\tabcolsep}}{%
\(< ,001\)} & {[}0,485; 0,739{]} & Ya &
\multicolumn{3}{>{\raggedright\arraybackslash}p{(\linewidth - 26\tabcolsep) * \real{0.2090} + 4\tabcolsep}@{}}{%
0,614 (Big)} \\
PjBL \(\rightarrow\) RPP &
\multicolumn{2}{>{\raggedright\arraybackslash}p{(\linewidth - 26\tabcolsep) * \real{0.1194} + 2\tabcolsep}}{%
0,030} &
\multicolumn{2}{>{\raggedright\arraybackslash}p{(\linewidth - 26\tabcolsep) * \real{0.1045} + 2\tabcolsep}}{%
0,045} &
\multicolumn{2}{>{\raggedright\arraybackslash}p{(\linewidth - 26\tabcolsep) * \real{0.0896} + 2\tabcolsep}}{%
0,770} &
\multicolumn{2}{>{\raggedright\arraybackslash}p{(\linewidth - 26\tabcolsep) * \real{0.1194} + 2\tabcolsep}}{%
,441} & {[}-0,055; 0,123{]} & Tidak &
\multicolumn{3}{>{\raggedright\arraybackslash}p{(\linewidth - 26\tabcolsep) * \real{0.2090} + 4\tabcolsep}@{}}{%
0,007 (Negligible)} \\
TPACK \(\rightarrow\) RPP &
\multicolumn{2}{>{\raggedright\arraybackslash}p{(\linewidth - 26\tabcolsep) * \real{0.1194} + 2\tabcolsep}}{%
0,425} &
\multicolumn{2}{>{\raggedright\arraybackslash}p{(\linewidth - 26\tabcolsep) * \real{0.1045} + 2\tabcolsep}}{%
0,037} &
\multicolumn{2}{>{\raggedright\arraybackslash}p{(\linewidth - 26\tabcolsep) * \real{0.0896} + 2\tabcolsep}}{%
11,346} &
\multicolumn{2}{>{\raggedright\arraybackslash}p{(\linewidth - 26\tabcolsep) * \real{0.1194} + 2\tabcolsep}}{%
\(< ,001\)} & {[}0,345; 0,491{]} & Ya &
\multicolumn{3}{>{\raggedright\arraybackslash}p{(\linewidth - 26\tabcolsep) * \real{0.2090} + 4\tabcolsep}@{}}{%
2,783 (Big)} \\
STEM \(\rightarrow\) RPP &
\multicolumn{2}{>{\raggedright\arraybackslash}p{(\linewidth - 26\tabcolsep) * \real{0.1194} + 2\tabcolsep}}{%
0,484} &
\multicolumn{2}{>{\raggedright\arraybackslash}p{(\linewidth - 26\tabcolsep) * \real{0.1045} + 2\tabcolsep}}{%
0,037} &
\multicolumn{2}{>{\raggedright\arraybackslash}p{(\linewidth - 26\tabcolsep) * \real{0.0896} + 2\tabcolsep}}{%
13,116} &
\multicolumn{2}{>{\raggedright\arraybackslash}p{(\linewidth - 26\tabcolsep) * \real{0.1194} + 2\tabcolsep}}{%
\(< ,001\)} & {[}0,412; 0,556{]} & Ya &
\multicolumn{3}{>{\raggedright\arraybackslash}p{(\linewidth - 26\tabcolsep) * \real{0.2090} + 4\tabcolsep}@{}}{%
5,444 (Big)} \\
ESD \(\rightarrow\) RPP &
\multicolumn{2}{>{\raggedright\arraybackslash}p{(\linewidth - 26\tabcolsep) * \real{0.1194} + 2\tabcolsep}}{%
0,345} &
\multicolumn{2}{>{\raggedright\arraybackslash}p{(\linewidth - 26\tabcolsep) * \real{0.1045} + 2\tabcolsep}}{%
0,040} &
\multicolumn{2}{>{\raggedright\arraybackslash}p{(\linewidth - 26\tabcolsep) * \real{0.0896} + 2\tabcolsep}}{%
8,355} &
\multicolumn{2}{>{\raggedright\arraybackslash}p{(\linewidth - 26\tabcolsep) * \real{0.1194} + 2\tabcolsep}}{%
\(< ,001\)} & {[}0,264; 0,423{]} & Ya &
\multicolumn{3}{>{\raggedright\arraybackslash}p{(\linewidth - 26\tabcolsep) * \real{0.2090} + 4\tabcolsep}@{}}{%
2,399 (Big)} \\
\end{longtable}
}

Sebagaimana ditunjukkan pada Tabel 7, PjBL memberikan efek positif
signifikan terhadap ketiga dimensi integrasi: TPACK (\(\beta = 0,727\),
\(p < ,001\), \(f^{2} = 1,123\)), STEM (\(\beta = 0,683\), \(p < ,001\),
\(f^{2} = 0,872\)), dan ESD (\(\beta = 0,617\), \(p < ,001\),
\(f^{2} = 0,614\)). Seluruh ukuran efek besar. Penting dicatat, jalur
langsung dari PjBL ke RPP tidak signifikan (\(\beta = 0,030\),
\(p = ,441\)), menunjukkan bahwa PjBL tidak langsung mempengaruhi
kualitas RPP tetapi bekerja melalui konstruk mediator.

Model menjelaskan 97,7\% varians RPP (\(R^{2} = 0,977\)), 52,9\% TPACK
(\(R^{2} = 0,529\)), 46,6\% STEM (\(R^{2} = 0,466\)), dan 38,0\% ESD
(\(R^{2} = 0,380\)). Relevansi prediktif (\(Q^{2}\)) bernilai positif
dan substansial untuk seluruh konstruk endogen: RPP (0,974), TPACK
(0,503), STEM (0,430), dan ESD (0,355), menunjukkan kapasitas prediktif
yang kuat melampaui prediksi rerata sederhana.

\subsection{Analisis komparatif: Dimensi dominan
(RM3)}\label{analisis-komparatif-dimensi-dominan-rm3}

Berdasarkan temuan RM2, RM3 menanyakan dimensi integrasi mana yang
paling responsif terhadap PjBL. Tabel 8 membandingkan koefisien jalur
PjBL ke dimensi.

\textbf{Tabel 8.} RM3 --- Perbandingan Koefisien Jalur PjBL ke dimensi

{\def\LTcaptype{none} % do not increment counter
\begin{longtable}[]{@{}
  >{\raggedright\arraybackslash}p{(\linewidth - 12\tabcolsep) * \real{0.0897}}
  >{\raggedright\arraybackslash}p{(\linewidth - 12\tabcolsep) * \real{0.1282}}
  >{\raggedright\arraybackslash}p{(\linewidth - 12\tabcolsep) * \real{0.1538}}
  >{\raggedright\arraybackslash}p{(\linewidth - 12\tabcolsep) * \real{0.1154}}
  >{\centering\arraybackslash}p{(\linewidth - 12\tabcolsep) * \real{0.1667}}
  >{\raggedright\arraybackslash}p{(\linewidth - 12\tabcolsep) * \real{0.1538}}
  >{\raggedright\arraybackslash}p{(\linewidth - 12\tabcolsep) * \real{0.1667}}@{}}
\toprule\noalign{}
\begin{minipage}[b]{\linewidth}\raggedright
Rank
\end{minipage} & \begin{minipage}[b]{\linewidth}\raggedright
Dimensi
\end{minipage} & \begin{minipage}[b]{\linewidth}\raggedright
\[\beta\]
\end{minipage} & \begin{minipage}[b]{\linewidth}\raggedright
\[t\]
\end{minipage} & \begin{minipage}[b]{\linewidth}\centering
\[p\]
\end{minipage} & \begin{minipage}[b]{\linewidth}\raggedright
\[f^{2}\]
\end{minipage} & \begin{minipage}[b]{\linewidth}\raggedright
Signifikan
\end{minipage} \\
\midrule\noalign{}
\endhead
\bottomrule\noalign{}
\endlastfoot
1 & TPACK & 0,727 & 13,295 & \(< ,001\) & 1,123 & Ya \\
2 & STEM & 0,683 & 12,616 & \(< ,001\) & 0,872 & Ya \\
3 & ESD & 0,617 & 9,496 & \(< ,001\) & 0,614 & Ya \\
\end{longtable}
}

Sebagaimana ditunjukkan pada Tabel 8, TPACK muncul sebagai dimensi
paling responsif (\(\beta = 0,727\), \(f^{2} = 1,123\)), diikuti STEM
(\(\beta = 0,683\), \(f^{2} = 0,872\)) dan ESD (\(\beta = 0,617\),
\(f^{2} = 0,614\)). Seluruh jalur signifikan dengan ukuran efek besar,
menghasilkan urutan: TPACK > STEM > ESD.

\begin{figure}
\centering
\includegraphics[width=4.6in,alt={Gambar 4. Koefisien Jalur dari PjBL ke Konstruk Mediator}]{figures/media/fig3_sem_rm3_paths.png}
\caption{Gambar 4. Koefisien Jalur dari PjBL ke Konstruk Mediator}
\end{figure}

\begin{enumerate}
\def\labelenumi{\arabic{enumi}.}
\setcounter{enumi}{3}
\tightlist
\item
  \emph{Higher-Order Construct: Dimensi yang Berkontribusi terhadap
  Kualitas RPP (RM4)}
\end{enumerate}

RM4 fokus pada bagaimana dimensi-dimensi ini berkontribusi terhadap
kualitas RPP integratif. Gambar 5. mengilustrasikan model struktural
lengkap dengan jalur dari PjBL melalui konstruk mediator ke RPP.

\begin{figure}
\centering
\includegraphics[width=3.87402in,height=2.30315in,alt={Gambar 5. Prediktor Kualitas RPP Integratif}]{figures/media/fig4_sem_full_model_hoc_proxy.png}
\caption{Gambar 5. Prediktor Kualitas RPP Integratif}
\end{figure}

Gambar 5. Prediktor kualitas RPP integratif

Gambar 5. Menunjukkan ketiga dimensi berkontribusi signifikan: STEM
(\(\beta = 0,484\)), TPACK (\(\beta = 0,425\)), dan ESD
(\(\beta = 0,345\)). Jalur langsung dari PjBL ke RPP tidak signifikan
(\(\beta = 0,030\), garis putus-putus).

Sebagaimana ditunjukkan pada Gambar 5 dan Tabel 7, ketiga dimensi
integrasi berkontribusi signifikan dan substansial terhadap kualitas RPP
integratif. STEM adalah kontributor terkuat (\(\beta = 0,484\)), diikuti
TPACK (\(\beta = 0,425\)) dan ESD (\(\beta = 0,345\)). Nilai \(f^{2}\)
yang konsisten besar (seluruh \(> 2,3\)) menunjukkan bahwa setiap
dimensi memberikan kontribusi bermakna dan non-redundan terhadap
kualitas RPP keseluruhan.

\subsection{Analisis mediasi (RM5)}\label{analisis-mediasi-rm5}

RM5 menguji apakah TPACK, STEM, dan ESD memediasi hubungan antara PjBL
dan kualitas RPP. Tabel 9 menyajikan hasil analisis mediasi.

\textbf{Tabel 9.} Analisis mediasi --- efek tidak langsung

{\def\LTcaptype{none} % do not increment counter
\begin{longtable}[]{@{}
  >{\raggedright\arraybackslash}p{(\linewidth - 26\tabcolsep) * \real{0.1206}}
  >{\raggedright\arraybackslash}p{(\linewidth - 26\tabcolsep) * \real{0.0922}}
  >{\raggedright\arraybackslash}p{(\linewidth - 26\tabcolsep) * \real{0.0922}}
  >{\raggedright\arraybackslash}p{(\linewidth - 26\tabcolsep) * \real{0.0426}}
  >{\raggedright\arraybackslash}p{(\linewidth - 26\tabcolsep) * \real{0.0426}}
  >{\raggedright\arraybackslash}p{(\linewidth - 26\tabcolsep) * \real{0.0426}}
  >{\raggedright\arraybackslash}p{(\linewidth - 26\tabcolsep) * \real{0.0567}}
  >{\centering\arraybackslash}p{(\linewidth - 26\tabcolsep) * \real{0.0567}}
  >{\centering\arraybackslash}p{(\linewidth - 26\tabcolsep) * \real{0.0567}}
  >{\raggedright\arraybackslash}p{(\linewidth - 26\tabcolsep) * \real{0.0638}}
  >{\raggedright\arraybackslash}p{(\linewidth - 26\tabcolsep) * \real{0.0567}}
  >{\raggedright\arraybackslash}p{(\linewidth - 26\tabcolsep) * \real{0.0567}}
  >{\raggedright\arraybackslash}p{(\linewidth - 26\tabcolsep) * \real{0.0426}}
  >{\centering\arraybackslash}p{(\linewidth - 26\tabcolsep) * \real{0.0922}}@{}}
\toprule\noalign{}
\begin{minipage}[b]{\linewidth}\raggedright
Jalur Tidak Langsung
\end{minipage} &
\multicolumn{2}{>{\raggedright\arraybackslash}p{(\linewidth - 26\tabcolsep) * \real{0.1844} + 2\tabcolsep}}{%
\begin{minipage}[b]{\linewidth}\raggedright
\[\beta_{indirect}\]
\end{minipage}} &
\multicolumn{2}{>{\raggedright\arraybackslash}p{(\linewidth - 26\tabcolsep) * \real{0.0851} + 2\tabcolsep}}{%
\begin{minipage}[b]{\linewidth}\raggedright
\[SE\]
\end{minipage}} &
\multicolumn{2}{>{\raggedright\arraybackslash}p{(\linewidth - 26\tabcolsep) * \real{0.0993} + 2\tabcolsep}}{%
\begin{minipage}[b]{\linewidth}\raggedright
\[t\]
\end{minipage}} &
\multicolumn{2}{>{\centering\arraybackslash}p{(\linewidth - 26\tabcolsep) * \real{0.1135} + 2\tabcolsep}}{%
\begin{minipage}[b]{\linewidth}\centering
\[p\]
\end{minipage}} & \begin{minipage}[b]{\linewidth}\raggedright
CI 95\%
\end{minipage} & \begin{minipage}[b]{\linewidth}\raggedright
VAF
\end{minipage} & \begin{minipage}[b]{\linewidth}\raggedright
Sobel \(z\)
\end{minipage} &
\multicolumn{2}{>{\raggedright\arraybackslash}p{(\linewidth - 26\tabcolsep) * \real{0.1348} + 2\tabcolsep}@{}}{%
\begin{minipage}[b]{\linewidth}\raggedright
Sobel \(p\)
\end{minipage}} \\
\midrule\noalign{}
\endhead
\bottomrule\noalign{}
\endlastfoot
PjBL \(\rightarrow\) TPACK \(\rightarrow\) RPP & 0,309 &
\multicolumn{2}{>{\raggedright\arraybackslash}p{(\linewidth - 26\tabcolsep) * \real{0.1348} + 2\tabcolsep}}{%
0,035} &
\multicolumn{2}{>{\raggedright\arraybackslash}p{(\linewidth - 26\tabcolsep) * \real{0.0851} + 2\tabcolsep}}{%
8,725} &
\multicolumn{2}{>{\raggedright\arraybackslash}p{(\linewidth - 26\tabcolsep) * \real{0.1135} + 2\tabcolsep}}{%
\(< ,001\)} &
\multicolumn{2}{>{\centering\arraybackslash}p{(\linewidth - 26\tabcolsep) * \real{0.1206} + 2\tabcolsep}}{%
{[}0,239; 0,375{]}} & 35,0\% &
\multicolumn{2}{>{\raggedright\arraybackslash}p{(\linewidth - 26\tabcolsep) * \real{0.0993} + 2\tabcolsep}}{%
8,686} & \(< ,001\) \\
PjBL \(\rightarrow\) STEM \(\rightarrow\) RPP & 0,330 &
\multicolumn{2}{>{\raggedright\arraybackslash}p{(\linewidth - 26\tabcolsep) * \real{0.1348} + 2\tabcolsep}}{%
0,033} &
\multicolumn{2}{>{\raggedright\arraybackslash}p{(\linewidth - 26\tabcolsep) * \real{0.0851} + 2\tabcolsep}}{%
10,039} &
\multicolumn{2}{>{\raggedright\arraybackslash}p{(\linewidth - 26\tabcolsep) * \real{0.1135} + 2\tabcolsep}}{%
\(< ,001\)} &
\multicolumn{2}{>{\centering\arraybackslash}p{(\linewidth - 26\tabcolsep) * \real{0.1206} + 2\tabcolsep}}{%
{[}0,271; 0,401{]}} & 37,4\% &
\multicolumn{2}{>{\raggedright\arraybackslash}p{(\linewidth - 26\tabcolsep) * \real{0.0993} + 2\tabcolsep}}{%
9,052} & \(< ,001\) \\
PjBL \(\rightarrow\) ESD \(\rightarrow\) RPP & 0,213 &
\multicolumn{2}{>{\raggedright\arraybackslash}p{(\linewidth - 26\tabcolsep) * \real{0.1348} + 2\tabcolsep}}{%
0,031} &
\multicolumn{2}{>{\raggedright\arraybackslash}p{(\linewidth - 26\tabcolsep) * \real{0.0851} + 2\tabcolsep}}{%
6,685} &
\multicolumn{2}{>{\raggedright\arraybackslash}p{(\linewidth - 26\tabcolsep) * \real{0.1135} + 2\tabcolsep}}{%
\(< ,001\)} &
\multicolumn{2}{>{\centering\arraybackslash}p{(\linewidth - 26\tabcolsep) * \real{0.1206} + 2\tabcolsep}}{%
{[}0,151; 0,275{]}} & 24,1\% &
\multicolumn{2}{>{\raggedright\arraybackslash}p{(\linewidth - 26\tabcolsep) * \real{0.0993} + 2\tabcolsep}}{%
6,335} & \(< ,001\) \\
Total indirect & 0,852 &
\multicolumn{2}{>{\raggedright\arraybackslash}p{(\linewidth - 26\tabcolsep) * \real{0.1348} + 2\tabcolsep}}{%
0,040} &
\multicolumn{2}{>{\raggedright\arraybackslash}p{(\linewidth - 26\tabcolsep) * \real{0.0851} + 2\tabcolsep}}{%
21,016} &
\multicolumn{2}{>{\raggedright\arraybackslash}p{(\linewidth - 26\tabcolsep) * \real{0.1135} + 2\tabcolsep}}{%
\(< ,001\)} &
\multicolumn{2}{>{\centering\arraybackslash}p{(\linewidth - 26\tabcolsep) * \real{0.1206} + 2\tabcolsep}}{%
{[}0,769; 0,928{]}} & \begin{minipage}[t]{\linewidth}\raggedright
\begin{center}\rule{0.5\linewidth}{0.5pt}\end{center}
\end{minipage} &
\multicolumn{2}{>{\raggedright\arraybackslash}p{(\linewidth - 26\tabcolsep) * \real{0.0993} + 2\tabcolsep}}{%
\begin{minipage}[t]{\linewidth}\raggedright
\begin{center}\rule{0.5\linewidth}{0.5pt}\end{center}
\end{minipage}} & \begin{minipage}[t]{\linewidth}\centering
\begin{center}\rule{0.5\linewidth}{0.5pt}\end{center}
\end{minipage} \\
Langsung (PjBL \(\rightarrow\) RPP) & 0,030 &
\multicolumn{2}{>{\raggedright\arraybackslash}p{(\linewidth - 26\tabcolsep) * \real{0.1348} + 2\tabcolsep}}{%
0,045} &
\multicolumn{2}{>{\raggedright\arraybackslash}p{(\linewidth - 26\tabcolsep) * \real{0.0851} + 2\tabcolsep}}{%
0,770} &
\multicolumn{2}{>{\raggedright\arraybackslash}p{(\linewidth - 26\tabcolsep) * \real{0.1135} + 2\tabcolsep}}{%
,441} &
\multicolumn{2}{>{\centering\arraybackslash}p{(\linewidth - 26\tabcolsep) * \real{0.1206} + 2\tabcolsep}}{%
{[}-0,055; 0,123{]}} & \begin{minipage}[t]{\linewidth}\raggedright
\begin{center}\rule{0.5\linewidth}{0.5pt}\end{center}
\end{minipage} &
\multicolumn{2}{>{\raggedright\arraybackslash}p{(\linewidth - 26\tabcolsep) * \real{0.0993} + 2\tabcolsep}}{%
\begin{minipage}[t]{\linewidth}\raggedright
\begin{center}\rule{0.5\linewidth}{0.5pt}\end{center}
\end{minipage}} & \begin{minipage}[t]{\linewidth}\centering
\begin{center}\rule{0.5\linewidth}{0.5pt}\end{center}
\end{minipage} \\
Total effect & 0,882 &
\multicolumn{2}{>{\raggedright\arraybackslash}p{(\linewidth - 26\tabcolsep) * \real{0.1348} + 2\tabcolsep}}{%
0,021} &
\multicolumn{2}{>{\raggedright\arraybackslash}p{(\linewidth - 26\tabcolsep) * \real{0.0851} + 2\tabcolsep}}{%
42,952} &
\multicolumn{2}{>{\raggedright\arraybackslash}p{(\linewidth - 26\tabcolsep) * \real{0.1135} + 2\tabcolsep}}{%
\(< ,001\)} &
\multicolumn{2}{>{\centering\arraybackslash}p{(\linewidth - 26\tabcolsep) * \real{0.1206} + 2\tabcolsep}}{%
{[}0,838; 0,917{]}} & \begin{minipage}[t]{\linewidth}\raggedright
\begin{center}\rule{0.5\linewidth}{0.5pt}\end{center}
\end{minipage} &
\multicolumn{2}{>{\raggedright\arraybackslash}p{(\linewidth - 26\tabcolsep) * \real{0.0993} + 2\tabcolsep}}{%
\begin{minipage}[t]{\linewidth}\raggedright
\begin{center}\rule{0.5\linewidth}{0.5pt}\end{center}
\end{minipage}} & \begin{minipage}[t]{\linewidth}\centering
\begin{center}\rule{0.5\linewidth}{0.5pt}\end{center}
\end{minipage} \\
\end{longtable}
}

\begin{figure}
\centering
\includegraphics[width=4.8in,alt={Gambar 6. Efek Tidak Langsung Spesifik melalui Konstruk Mediator}]{figures/media/fig5_sem_mediation_paths.png}
\caption{Gambar 6. Efek Tidak Langsung Spesifik melalui Konstruk Mediator}
\end{figure}

\begin{quote}
Sebagaimana ditunjukkan pada Tabel 9, ketiga jalur tidak langsung
signifikan secara statistik berdasarkan confidence interval bootstrap:
\end{quote}

\begin{enumerate}
\def\labelenumi{\alph{enumi}.}
\item
  PjBL \(\rightarrow\) STEM \(\rightarrow\) RPP adalah jalur mediasi
  terkuat (indirect \(\beta = 0,330\), \(p < ,001\), CI 95\% {[}0,271;
  0,401{]}). VAF 37,4\% menunjukkan mediasi parsial, dan uji Sobel
  mengonfirmasi signifikansi (\(z = 9,052\), \(p < ,001\)).
\item
  PjBL \(\rightarrow\) TPACK \(\rightarrow\) RPP menghasilkan efek tidak
  langsung signifikan (indirect \(\beta = 0,309\), \(p < ,001\), CI 95\%
  {[}0,239; 0,375{]}). VAF 35,0\% konsisten dengan mediasi parsial.
\item
  PjBL \(\rightarrow\) ESD \(\rightarrow\) RPP juga signifikan (indirect
  \(\beta = 0,213\), \(p < ,001\), CI 95\% {[}0,151; 0,275{]}). VAF
  24,1\% menunjukkan mediasi parsial.
\end{enumerate}

\begin{quote}
Total indirect effect signifikan (\(\beta = 0,852\), \(p < ,001\)),
sedangkan direct effect PjBL terhadap RPP tidak signifikan
(\(\beta = 0,030\), \(p = ,441\)). Pola ini konsisten dengan full
mediation secara agregat: PjBL mempengaruhi kualitas RPP integratif
bukan secara langsung, tetapi melalui peningkatan ketiga dimensi
integrasi. Total effect PjBL terhadap RPP signifikan (\(\beta = 0,882\),
\(p < ,001\)), mengonfirmasi bahwa pengaruh keseluruhan PjBL terhadap
kualitas RPP substansial dan bekerja melalui dimensi mediator.
\end{quote}

