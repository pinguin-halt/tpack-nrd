\section{Discussion}\label{discussion}

\begin{verbatim}
1.  *Efektivitas PjBL dalam meningkatkan kompetensi integratif
    (RQ1)*
\end{verbatim}

Analisis pre--post menunjukkan peningkatan yang signifikan secara
statistik pada keempat konstruk, dengan ukuran efek yang konsisten besar
(Cohen's d > 2,3). Temuan ini menjadi bukti kuat bahwa
kompetensi perencanaan pembelajaran integratif partisipan meningkat
secara substansial setelah intervensi PjBL dibandingkan sebelum
intervensi. Hasil ini sejalan dengan \\citet{dewi2022projectbased} yang melaporkan
bahwa scaffolding PjBL meningkatkan TPACK dan kemampuan desain
pembelajaran calon guru, sekaligus memperluas bukti bahwa peningkatan
tidak hanya pada integrasi teknologi, tetapi juga mencakup dimensi STEM
dan ESD.

Analisis N-Gain memperlihatkan perbedaan tingkat respons antar konstruk.
TPACK (\emph{M} = 0,596) dan STEM (\emph{M} = 0,574) mencatat gain
tertinggi dan keduanya mendekati batas atas kategori sedang. RPP
Integratif (\emph{M} = 0,513) berada pada tingkat sedang bagian tengah.
Sebaliknya, ESD memiliki N-Gain terendah (\emph{M} = 0,376), hampir
tidak melewati ambang batas sedang; tidak ada partisipan yang mencapai
kategori tinggi dan 28,4\% masih berada pada kategori rendah.

Pola diferensial ini penting secara pedagogis. Kenaikan yang lebih kuat
pada TPACK dan STEM kemungkinan dipengaruhi oleh kesesuaian
karakteristik PjBL dengan kedua kompetensi tersebut. PjBL menekankan
produk/artefak yang menuntut integrasi teknologi (menguatkan TPACK) dan
umumnya menggunakan pertanyaan pendorong lintas disiplin yang
memfasilitasi cara berpikir interdisipliner (mendukung STEM).
Sebaliknya, ESD (terutama kemampuan mengintegrasikan perspektif
keberlanjutan, inkuiri lingkungan, dan penalaran evaluatif atas isu
sosio-ilmiah) memerlukan reorientasi pedagogis yang tidak selalu muncul
otomatis dari praktik PjBL yang bersifat umum. \\citet{purwianingsih2022programfor} juga menemukan pola serupa, bahwa integrasi ESD dalam TPACK calon
guru membutuhkan program yang terstruktur dan terfokus, bukan sekadar
paparan insidental. Karena itu, meskipun PjBL dapat menjadi konteks yang
mendukung pengembangan ESD, diperlukan scaffolding eksplisit (misalnya
pertanyaan pendorong berbasis keberlanjutan, refleksi terstruktur
terkait SDGs, atau kriteria desain ESD khusus) agar peningkatan ESD
dapat melampaui kategori sedang.

Bukti terbaru memperkuat interpretasi ini. Studi tentang desain
pembelajaran PBL-STEM pada calon guru secara konsisten menunjukkan
peningkatan lebih cepat pada kualitas desain interdisipliner dan
perencanaan berbasis teknologi dibandingkan kompetensi aksi
keberlanjutan, kecuali ketika ESD secara sengaja diintegrasikan ke dalam
arsitektur tugas desain \\citep{pitot2024establishinga,portilloblanco2025buildingan,vidal2025preservice}. Sejalan dengan itu, riset pendidikan guru
berbasis keberlanjutan menunjukkan bahwa penguatan kompetensi ESD lebih
efektif ketika refleksi dan tugas keberlanjutan berbasis konteks
komunitas dirancang secara eksplisit \\citep{singhpillay2023preservice,ozdemiryilmazer2025towardsa}.

Pola ini juga selaras dengan bukti implementasi PBL pada konteks IPA
yang menunjukkan bahwa guru cenderung lebih cepat mengimplementasikan
elemen PBL yang tampak (kolaborasi, produksi artefak, presentasi,
refleksi) dibandingkan komponen yang menuntut kendali belajar siswa yang
lebih mendalam; kondisi ini dapat menjelaskan mengapa penalaran
berorientasi ESD berkembang lebih lambat tanpa dukungan desain yang
eksplisit \\citep{markula2022thekey}.

\subsection{PjBL sebagai penggerak dimensi integrasi
(RQ2)}\label{pjbl-sebagai-penggerak-dimensi-integrasi-rq2}

Analisis PLS-SEM mengungkapkan bahwa PjBL memberikan efek positif yang
signifikan secara statistik dan kuat pada ketiga dimensi integrasi:
TPACK (\(\beta = 0,727\), \(p < 0,001\)), STEM (\(\beta = 0,683\),
\(p < 0,001\)), dan ESD (\(\beta = 0,617\), \(p < 0,001\)). Seluruh
effect sizes besar (\(f^{2} > 0,35\)), menunjukkan signifikansi praktis
substansial. Jalur langsung dari PjBL ke RPP tidak signifikan
(\(\beta = 0,030\), \(p = 0,441\)). Temuan ini sepenuhnya mendukung H1:
PjBL secara signifikan mempengaruhi ketiga dimensi integrasi.

Jalur PjBL \(\rightarrow\) TPACK yang kuat (\(\beta = 0,727\)) selaras
dengan kerangka teoretis. Elemen-elemen inti PjBL (pertanyaan pendorong
autentik, investigasi kolaboratif, dan penciptaan artefak) secara
inheren mendorong calon guru mempertimbangkan peran teknologi dalam
mendukung tujuan pedagogis dan penyampaian konten. Temuan ini memperluas
hasil SEM \\citet{mansour2024scienceand} di Qatar yang menunjukkan hubungan
signifikan antara pendekatan pedagogis dan integrasi TPACK. Nilai effect
size yang besar (\(f^{2} = 1,123\)) menunjukkan bahwa PjBL efektif dalam
mengembangkan kompetensi integrasi teknologi.

Interpretasi ini sejalan dengan temuan studi intervensi terbaru bahwa
penerapan teknologi yang terstruktur dalam pendidikan calon guru IPA
meningkatkan TPACK sekaligus orientasi perilaku terhadap penggunaan
teknologi di kelas, sehingga menunjukkan bahwa tugas desain pedagogis
dapat memperkuat pengetahuan dan kesiapan enactment secara simultan
\\citep{stinkenrosner2023technologyimplementation,gurer2021theinfluence}

Jalur PjBL \(\rightarrow\) STEM yang signifikan (\(\beta = 0,683\),
\(f^{2} = 0,872\)) menegaskan bahwa karakter interdisipliner PjBL
mendorong aktivasi penalaran STEM. Dalam merancang rencana pembelajaran
berbasis PjBL, partisipan perlu mengaitkan konten sains dengan tantangan
desain teknik, memanfaatkan teknologi untuk pengumpulan dan analisis
data, serta menggunakan penalaran matematis dalam pemecahan masalah.
Kesesuaian tuntutan struktural PjBL dengan integrasi STEM menjelaskan
kuatnya hubungan tersebut.

Jalur PjBL \(\rightarrow\) ESD yang signifikan (\(\beta = 0,617\),
\(f^{2} = 0,614\)) merupakan temuan penting yang tidak selaras dengan
pola N-Gain. Meskipun ESD menunjukkan N-Gain terendah pada analisis
pre--post, model struktural menunjukkan bahwa kualitas implementasi PjBL
secara signifikan memprediksi kompetensi ESD pada data posttest.
Perbedaan ini dapat dijelaskan karena analisis pre--post menilai
peningkatan absolut dari baseline yang lebih rendah, sedangkan SEM
menilai hubungan struktural pada posttest. Dengan demikian, PjBL
berpengaruh terhadap pengembangan ESD secara struktural, tetapi
peningkatan absolutnya dapat terbatas oleh titik awal yang rendah dan
jarak pedagogis yang lebih besar antara struktur umum PjBL dan
kompetensi spesifik ESD.

Secara keseluruhan, PjBL menjelaskan varians substansial dalam ketiga
dimensi integrasi: 52,9\% dalam TPACK, 46,6\% dalam STEM, dan 38,0\%
dalam ESD. Nilai \(R^{2}\) moderat hingga substansial ini menunjukkan
bahwa PjBL sebagai prediktor eksogen tunggal menangkap porsi bermakna
faktor yang memengaruhi kompetensi integrasi. Varians yang belum
terjelaskan kemungkinan berasal dari faktor lain (misalnya pengetahuan
konten awal, literasi digital, efikasi diri mengajar, keyakinan tentang
keberlanjutan, dan keterampilan metakognitif) yang belum dimasukkan
dalam model.

\subsection{Dimensi integrasi yang paling responsif
(RQ3)}\label{dimensi-integrasi-yang-paling-responsif-rq3}

Analisis komparatif menunjukkan bahwa TPACK merupakan dimensi yang
paling kuat dipengaruhi (\(\beta\)=0,727; \(f^2\)=1,123), disusul STEM (\(\beta\)=0,683;
\(f^2\)=0,872) dan ESD (\(\beta\)=0,617; \(f^2\)=0,614). Urutan pengaruh (TPACK
> STEM > ESD) ditetapkan berdasarkan koefisien
jalur dan \emph{effect size} pada model struktural. Ketiga jalur
signifikan dengan \emph{effect size} besar, sehingga PjBL dapat
dinyatakan efektif dalam mengembangkan ketiga dimensi tersebut, dengan
hubungan terkuat pada TPACK.

Secara praktis, urutan ini berimplikasi pada pengembangan kurikulum.
Dominannya pengaruh PjBL terhadap integrasi TPACK mengindikasikan bahwa
orientasi PjBL yang berbasis teknologi dan berfokus pada penciptaan
artefak secara natural mengaktivasi TPACK. Karena itu, program
pendidikan guru dapat menjadikan PjBL sebagai strategi utama untuk
mengembangkan kompetensi TPACK, sambil menambahkan intervensi pendukung
bagi dimensi yang memerlukan penguatan.

Selaras dengan literatur perencanaan pembelajaran, calon guru umumnya
berkembang lebih awal pada struktur perencanaan, integrasi alat, dan
koherensi disipliner, sedangkan pertimbangan desain yang lebih kompleks
(misalnya keadilan atau keberlanjutan) cenderung muncul setelah adanya
\emph{scaffolding} tambahan dan siklus umpan balik iteratif \\citep{karlstrom2021howdo,beckmann2023informaland,davis2024preserviceteachers}.

\begin{enumerate}
\def\labelenumi{\arabic{enumi}.}
\setcounter{enumi}{3}
\tightlist
\item
  \emph{Dimensi sebagai konstituen kualitas rencana pembelajaran
  integratif (RQ4)}
\end{enumerate}

Ketiga dimensi integrasi berkontribusi signifikan terhadap kualitas
rencana pembelajaran integratif, dengan effect size yang secara
konsisten besar: STEM (\(\beta\)=0,484, f\^{}2=5,444), TPACK (\(\beta\)=0,425,
f\^{}2=2,783), dan ESD (\(\beta\)=0,345, f\^{}2=2,399). Temuan ini mendukung H2
dan menegaskan perencanaan pembelajaran integratif sebagai kompetensi
tingkat tinggi yang dibentuk oleh tiga dimensi yang berbeda namun saling
melengkapi.

Nilai \(f^{2}\) yang tinggi (seluruhnya \textgreater2,5) menunjukkan
bahwa setiap dimensi memberikan kontribusi yang substansial dan
non-redundan terhadap kualitas rencana pembelajaran. Secara teoretis,
hal ini relevan bagi kerangka Teacher Design Capacity \citep{brown2009item,mckenney2015teacherdesign}, karena mengindikasikan bahwa kompetensi
perencanaan pembelajaran integratif tidak dapat direduksi menjadi satu
dimensi saja, melainkan membutuhkan pengembangan terkoordinasi integrasi
teknologi, penalaran interdisipliner, dan perspektif keberlanjutan.
Dengan demikian, program pendidikan guru yang hanya menekankan satu
dimensi---misalnya berfokus eksklusif pada TPACK---tidak memadai untuk
membangun kompetensi desain integratif secara utuh.

Argumen tersebut selaras dengan bukti riset perencanaan pembelajaran
kontemporer yang memandang kualitas perencanaan calon guru sebagai
luaran komposit dari berbagai sumber pengetahuan dan keputusan desain,
bukan keterampilan domain tunggal, terutama dalam konteks pendidikan
sains \\citep{krepf2022structuringthe,tellezacosta2023preservice,pleasants2024whatmakes}.

Namun, terdapat catatan metodologis penting. Nilai \(R^{2}\) untuk RPP
sebesar 0,977 tergolong sangat tinggi dan kemungkinan mencerminkan sifat
komposisional konstruk, karena RPPInt\_total\_post dihitung sebagai
rerata skor indikator TPACK, STEM, dan ESD, sehingga ketiga dimensi
menjadi prediktor yang hampir sempurna secara konstruksi. Akibatnya,
nilai \(f^{2}\) yang besar dan \(R^{2}\) yang tinggi tidak sepenuhnya
merepresentasikan hubungan empiris, tetapi juga menangkap keniscayaan
matematis. Hal ini tidak meniadakan implikasi konseptual bahwa
perencanaan pembelajaran integratif bersifat multidimensi, tetapi
koefisien jalur perlu ditafsirkan terutama sebagai konsekuensi
operasionalisasi konstruk, bukan semata-mata besaran efek empiris.

\subsection{Peran mediasi dimensi integrasi
(RQ5)}\label{peran-mediasi-dimensi-integrasi-rq5}

Analisis mediasi mengungkapkan pola mediasi penuh secara agregat: efek
tidak langsung total signifikan (\(\beta = 0,852\), \(p < 0,001\)),
sementara efek langsung PjBL terhadap RPP tidak signifikan
(\(\beta = 0,030\), \(p = 0,441\)). Temuan ini mendukung H3 dan
menegaskan implikasi teoretis bahwa PjBL tidak meningkatkan kualitas
rencana pembelajaran integratif secara langsung, tetapi bekerja melalui
peningkatan tiga kompetensi integrasi.

Ketiga jalur mediasi juga signifikan secara statistik:

\begin{enumerate}
\def\labelenumi{\alph{enumi}.}
\item
  STEM adalah mediator terkuat (\(\beta\) tidak langsung \(= 0,330\),
  VAF \(= 37,4\%\)), menunjukkan bahwa pengaruh PjBL terhadap kualitas
  rencana pembelajaran terutama berlangsung melalui pengembangan
  kompetensi integrasi STEM.
\item
  TPACK adalah mediator terkuat kedua (\(\beta\) tidak langsung
  \(= 0,309\), VAF \(= 35,0\%\)), mengindikasikan bahwa integrasi
  teknologi merupakan saluran penting yang memperkuat perencanaan
  pembelajaran melalui PjBL.
\item
  ESD juga merupakan mediator signifikan (\(\beta\) tidak langsung
  \(= 0,213\), VAF \(= 24,1\%\)), sehingga meskipun peningkatan
  absolutnya paling rendah pada analisis pre--post, ESD tetap menjadi
  jalur mediasi yang bermakna dalam model struktural.
\end{enumerate}

Tiga jalur mediasi yang signifikan ini memperluas temuan penelitian
sebelumnya yang masih menunjukkan pola mediasi campuran atau parsial.
Mediasi penuh mengindikasikan bahwa PjBL memengaruhi kualitas rencana
pembelajaran terutama melalui penguatan keterampilan yang spesifik,
bukan melalui mekanisme yang bersifat umum dan tidak terarah \\citep{hair2022item}. Jalur mediasi ESD juga penting: meskipun ESD membutuhkan
scaffolding tambahan untuk memaksimalkan peningkatan absolut
(sebagaimana terlihat pada analisis N-Gain), model struktural menegaskan
bahwa pengembangan kompetensi ESD merupakan bagian inti dari mekanisme
bagaimana PjBL meningkatkan kualitas rencana pembelajaran.

Dalam konteks pembelajaran guru, penjelasan berbasis mekanisme ini
sejalan dengan studi tentang perkembangan perencanaan calon guru, yang
menunjukkan bahwa dampak intervensi biasanya muncul terlebih dahulu
sebagai pergeseran kompetensi antara (misalnya pengetahuan perencanaan,
penalaran pedagogis, dan langkah reflektif dalam desain) sebelum tampak
pada indikator kualitas perencanaan secara holistik \\citep{karlstrom2021howdo,beckmann2023informaland}.

\subsection{Implikasi teoretis}\label{implikasi-teoretis}

Temuan sebelumnya berimplikasi pada pengembangan teori. Pertama,
penelitian ini memberikan uji empiris pertama terhadap model struktural
yang menempatkan TPACK, STEM, dan ESD sebagai mediator simultan antara
PjBL dan kualitas rencana pembelajaran integratif. Mediasi penuh (dengan
ketiga jalur yang signifikan) menunjukkan bahwa model kapasitas desain
guru perlu merinci jalur komposisional bagaimana intervensi memengaruhi
kompetensi desain, bukan memandang kapasitas desain sebagai luaran
tunggal dari pengalaman pedagogis \\citep{brown2009item}.

Kedua, penelitian ini menunjukkan bahwa ketiga dimensi integrasi
dipengaruhi secara signifikan oleh PjBL, sehingga menantang asumsi bahwa
sebagian dimensi (misalnya ESD) kurang dapat dikembangkan melalui PjBL.
Koefisien jalur yang kuat (seluruh \(\beta\)\textgreater0,6) dan effect size
yang besar (\(f^{2} > 0,5\)) menegaskan bahwa PjBL dapat menjadi
intervensi komprehensif untuk mengembangkan berbagai kompetensi
integrasi secara bersamaan.

Ketiga, penelitian ini memvalidasi perencanaan pembelajaran integratif
sebagai konstruk yang dibentuk oleh (bukan sekadar berkorelasi dengan)
kompetensi TPACK, STEM, dan ESD. Signifikansi konsisten dan effect sizes
besar pada ketiga jalur dimensional memperkuat dasar empiris bagi
konseptualisasi konstruk tingkat tinggi.

Temuan ini selaras dengan studi terbaru tentang keberlanjutan
lintas-kurikuler dalam pendidikan guru, yang menegaskan bahwa kualitas
perencanaan holistik muncul ketika logika teknologi, disiplin, dan
keberlanjutan dirancang secara simultan, bukan secara berurutan
\\citep{ozdemiryilmazer2025towardsa,vidal2025preservice}.

\subsection{Implikasi praktis}\label{implikasi-praktis}

Selain kontribusi teoretis, temuan penelitian ini memberikan rekomendasi
praktis bagi program pendidikan guru (Lembaga Pendidikan Tenaga
Kependidikan, LPTK). Pertama, PjBL perlu diadopsi sebagai strategi
pedagogis inti pada mata kuliah persiapan guru IPA karena terbukti
meningkatkan tiga kompetensi integrasi dengan effect size yang besar.
Hubungan struktural yang kuat (\(\beta\)\textgreater0,6 pada seluruh dimensi)
menegaskan bahwa PjBL efektif untuk mengembangkan kompetensi TPACK,
STEM, dan ESD.

Kedua, meskipun PjBL efektif pada semua dimensi, analisis N-Gain
menunjukkan peningkatan absolut ESD paling rendah. Karena itu, PjBL
perlu dilengkapi scaffolding ESD yang eksplisit untuk mengoptimalkan
pengembangan ESD, misalnya melalui pertanyaan pendorong berfokus
keberlanjutan, refleksi terstruktur terkait koneksi SDG, atau kriteria
desain ESD khusus dalam proses PjBL.

Ketiga, asesmen berbasis rubrik dalam penelitian ini menyediakan
instrumen praktis untuk menilai kualitas rencana pembelajaran integratif
calon guru. Rubrik 14 indikator (TPACK (7 item), STEM (4 item), dan ESD
(3 item)) dapat diadopsi atau disesuaikan untuk menilai kompetensi
desain multidimensi melalui tugas kinerja autentik, bukan ukuran laporan
diri.

Keempat, hasil mediasi menunjukkan bahwa ketiga dimensi integrasi
merupakan jalur yang diperlukan bagi PjBL dalam meningkatkan kualitas
rencana pembelajaran. Oleh karena itu, pendidik guru perlu memastikan
aktivitas PjBL mencakup ketiga dimensi tersebut, bukan hanya satu atau
dua dimensi.

Secara praktis dalam desain kurikulum, hal ini berarti brief proyek,
template perencanaan, dan rubrik umpan balik harus secara eksplisit
menuntut bukti keselarasan teknologi--pedagogi, koherensi STEM
interdisipliner, serta penalaran keberlanjutan pada setiap prototipe
pembelajaran \\citep{pitot2024establishinga,tellezacosta2023preservice,portilloblanco2025buildingan}.

\subsection{Kontribusi}\label{kontribusi}

Kontribusi utama penelitian ini adalah menyajikan uji empiris pertama
atas model struktural yang secara simultan menempatkan TPACK, STEM, dan
ESD sebagai mediator antara PjBL dan kualitas rencana pembelajaran
integratif. Temuan bahwa ketiga jalur tersebut berperan sebagai mediator
signifikan memperkuat kerangka \emph{Teacher Design Capacity} dengan
memperjelas jalur komposisional yang menjelaskan bagaimana intervensi
pedagogis diterjemahkan menjadi kompetensi desain. Berbeda dari studi
sebelumnya yang menyatakan bahwa beberapa dimensi (misalnya, ESD) kurang
responsif terhadap pengembangan berbasis PjBL, penelitian ini
menunjukkan bahwa PjBL berpengaruh signifikan terhadap ketiga dimensi
dengan \emph{effect size} besar; namun, analisis N-Gain mengindikasikan
bahwa ESD masih memerlukan \emph{scaffolding} tambahan untuk
mengoptimalkan peningkatan absolut.

\subsection{Keterbatasan}\label{keterbatasan}

Beberapa keterbatasan membatasi interpretasi dan generalisasi temuan.

Konstruk RPP Integratif diukur dengan indikator tunggal
(RPPInt\_total\_post), yaitu rata-rata aritmatika skor TPACK, STEM, dan
ESD. Pendekatan komposit ini berpotensi menghasilkan nilai \(R^2\) yang
tinggi secara mekanistis (0,977) pada konstruk RPP. Alternatif yang
lebih ketat adalah penilaian holistik independen terhadap kualitas RPP
oleh evaluator eksternal.

Partisipan berasal dari satu institusi pendidikan guru sehingga temuan
sulit digeneralisasi ke konteks institusional dan budaya lain. Ukuran
sampel (N = 95) memang melampaui syarat minimum PLS-SEM, tetapi masih
moderat untuk kompleksitas model. Replikasi pada berbagai institusi dan
sampel yang lebih besar diperlukan untuk memperkuat bukti.

Walaupun PjBL berpengaruh signifikan pada ketiga dimensi, analisis
N-Gain menunjukkan peningkatan ESD relatif lebih kecil secara absolut.
Intervensi yang lebih panjang dengan scaffolding ESD yang eksplisit
berpotensi meningkatkan kompetensi ESD secara lebih kuat.

Keterbatasan tersebut sejalan dengan literatur ESD pada pendidikan guru:
intervensi jangka pendek umumnya meningkatkan awareness dan niat
perencanaan, sedangkan action competence ESD yang lebih mendalam
biasanya memerlukan persiapan yang lebih panjang, iteratif, dan tertanam
dalam konteks \\citep{singhpillay2023preservice,vidal2025preservice}.

Meski demikian, penelitian ini tetap memberikan uji model integratif
yang koheren dan berbasis empiris. Konsistensi temuan pre--post (RQ1)
dan SEM (RQ2--RQ5) memperkuat keyakinan terhadap pola hasil, terutama
bahwa PjBL mempengaruhi ketiga dimensi integrasi secara signifikan
melalui mekanisme mediasi penuh.

\subsection{Arah masa depan}\label{arah-masa-depan}

Penelitian selanjutnya perlu mengatasi keterbatasan studi ini melalui
penyempurnaan metodologis. Pertama, replikasi multi-situs pada beragam
konteks institusional dan budaya diperlukan untuk meningkatkan
generalisabilitas. Kedua, intervensi yang lebih panjang disertai
scaffolding ESD terarah (misalnya pertanyaan pendorong berfokus
keberlanjutan dan refleksi SDG yang terstruktur) berpotensi menghasilkan
peningkatan absolut kompetensi ESD yang lebih besar.

Selain itu, studi metode campuran yang memadukan data kualitatif
(seperti wawancara dan observasi kelas) dapat memberi pemahaman yang
lebih mendalam mengenai mekanisme PjBL dalam membentuk pemikiran desain
integratif. Studi longitudinal yang menelusuri retensi kompetensi
integratif setelah periode intervensi juga akan memperkuat bukti tentang
kontribusi PjBL dalam mempersiapkan calon guru IPA.

