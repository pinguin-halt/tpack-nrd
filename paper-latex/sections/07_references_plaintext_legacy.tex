References

Akbulut, M. Ş., \& Öner, D. (2021). Developing pre-service teachers'
technology competencies: A project-based learning experience.
\emph{Çukurova Üniversitesi Eğitim Fakültesi Dergisi, 50}(1), 240--260.
https://dergipark.org.tr/en/pub/cuefd/issue/59484/753044

Bell, S. (2010). Project-based learning for the 21st century: Skills for
the future. \emph{The Clearing House: A Journal of Educational
Strategies, Issues and Ideas, 83}(2), 39--43.
https://doi.org/10.1080/00098650903505415

Brown, M. W. (2009). The teacher-tool relationship: Theorizing the
design and use of curriculum materials. In J. T. Remillard, B. A.
Herbel-Eisenmann, \& G. M. Lloyd (Eds.), \emph{Mathematics teachers at
work: Connecting curriculum materials and classroom instruction}
(pp.~17--36). Routledge. https://doi.org/10.4324/9780203884645-11

Cohen, J. (1988). \emph{Statistical power analysis for the behavioral
sciences} (2nd ed.). Lawrence Erlbaum Associates.

Creswell, J. W., \& Creswell, J. D. (2018). \emph{Research design:
Qualitative, quantitative, and mixed methods approaches} (5th ed.). SAGE
Publications.

Dewi, N. R., Rusilowati, A., Saptono, S., \& Haryani, S. (2022).
Project-based scaffolding TPACK model to improve learning design ability
and TPACK of pre-service science teachers. \emph{Jurnal Pendidikan IPA
Indonesia, 11}(3), 439--451. https://doi.org/10.15294/jpii.v11i3.38566

Gumbi, N. M., Sibaya, D., \& Chibisa, A. (2024). Exploring pre-service
teachers' perspectives on the integration of digital game-based learning
for sustainable STEM education. \emph{Sustainability, 16}(3), Article
1314. https://doi.org/10.3390/su16031314

Hair, J. F., Hult, G. T. M., Ringle, C. M., \& Sarstedt, M. (2022).
\emph{A primer on partial least squares structural equation modeling
(PLS-SEM)} (3rd ed.). SAGE Publications.

Hake, R. R. (1998). Interactive-engagement versus traditional methods: A
six-thousand-student survey of mechanics test data for introductory
physics courses. \emph{American Journal of Physics, 66}(1), 64--74.
https://doi.org/10.1119/1.18809

Henseler, J., Ringle, C. M., \& Sarstedt, M. (2015). A new criterion for
assessing discriminant validity in variance-based structural equation
modeling. \emph{Journal of the Academy of Marketing Science, 43}(1),
115--135. https://doi.org/10.1007/s11747-014-0403-8

Kelley, T. R., \& Knowles, J. G. (2016). A conceptual framework for
integrated STEM education. \emph{International Journal of STEM
Education, 3}, Article 11. https://doi.org/10.1186/s40594-016-0046-z

Krajcik, J. S., \& Shin, N. (2014). Project-based learning. In R. K.
Sawyer (Ed.), \emph{The Cambridge handbook of the learning sciences}
(2nd ed., pp.~275--297). Cambridge University Press.
https://doi.org/10.1017/CBO9781139519526.018

Mansour, N., Said, Z., Çevik, M., \& Abu-Tineh, A. (2024). Science and
mathematics teachers' integration of TPACK in STEM subjects in Qatar: A
structural equation modeling study. \emph{Education Sciences, 14}(10),
Article 1138. https://doi.org/10.3390/educsci14101138

McKenney, S., Kali, Y., Markauskaite, L., \& Voogt, J. (2015). Teacher
design knowledge for technology enhanced learning: An ecological
framework for investigating assets and needs. \emph{Instructional
Science, 43}(2), 181--202. https://doi.org/10.1007/s11251-014-9337-2

Mishra, P., \& Koehler, M. J. (2006). Technological pedagogical content
knowledge: A framework for teacher knowledge. \emph{Teachers College
Record, 108}(6), 1017--1054. https://doi.org/10.1177/016146810610800610

Novallyan, D., \& Nehru. (2025). Optimization of teaching profession
courses through project methods: Impact on biology education students.
\emph{International Journal of Education, Technology, and Science,
4}(2), 431. https://doi.org/10.57092/ijetz.v4i2.431

Offermann, L. R., Pham, H. T., Baskerville, K. A., \& Langkamer, K. L.
(2025). Technological pedagogical content knowledge (TPACK) among
educators: A meta-analytic review. \emph{Frontiers in Psychology, 16},
Article 1656795. https://doi.org/10.3389/fpsyg.2025.1656795

Preacher, K. J., \& Hayes, A. F. (2008). Asymptotic and resampling
strategies for assessing and comparing indirect effects in multiple
mediator models. \emph{Behavior Research Methods, 40}(3), 879--891.
https://doi.org/10.3758/BRM.40.3.879

Purwianingsih, W., Novidsa, I., \& Riandi, R. (2022). Program for
integrating education for sustainable development (ESD) into prospective
biology teachers' technological pedagogical content knowledge (TPACK).
\emph{Jurnal Pendidikan IPA Indonesia, 11}(2), 298--312.
https://doi.org/10.15294/jpii.v11i2.34772

Salleh, M. F. M., Awang, M. I., \& Aziz, N. A. A. (2025). How
pre-service science teachers develop TPACK competence: A systematic
literature review. \emph{Jurnal Pendidikan IPA Indonesia, 14}(1),
168--179. https://doi.org/10.15294/jpii.v14i1.16819

Shulman, L. S. (1986). Those who understand: Knowledge growth in
teaching. \emph{Educational Researcher, 15}(2), 4--14.
https://doi.org/10.3102/0013189X015002004

Shumba, O., \& Kampamba, R. (2013). Mainstreaming ESD into science
teacher education courses: A case for ESD pedagogical content knowledge
and learning as connection. \emph{Southern African Journal of
Environmental Education, 29}, 151--166.
https://www.ajol.info/index.php/sajee/article/view/122267

Tucker, S. I., Lommatsch, C. W., Burroughs, E. A., \& Yopp, D. (2024).
Examining STEM integration in elementary mathematics methods courses.
\emph{Journal of Research in STEM Education, 10}(SI), Article 313.
https://doi.org/10.31756/jrsmte.313si

Beckmann, T., \& Ehmke, T. (2023). Informal and formal lesson planning
in school internships: Practices among pre-service teachers.
\emph{Teaching and Teacher Education, 132}, 104249.
https://doi.org/10.1016/j.tate.2023.104249

Benedict-Chambers, A., \& Sherwood, C.-A. (2024). Planning for equitable
student sensemaking: An examination of preservice teachers' noticing of
elementary science lesson plans. \emph{Journal of Science Teacher
Education, 35}(8), 862-882.
https://doi.org/10.1080/1046560X.2024.2356944

Davis, E. A., \& Bautista, J. (2024). Preservice teachers' early lesson
planning for justice-oriented elementary science. \emph{Journal of
Science Teacher Education, 36}(4), 485-510.
https://doi.org/10.1080/1046560X.2024.2428489

Calero, M., Pina, T., Mayoral, O., Canto, J., Ull, M. A., \& Vilches, A.
(2024). A study about pre-service teachers' knowledge of the Sustainable
Development Goals. \emph{International Journal of Sustainability in
Higher Education, 26}(2), 352-371.
https://doi.org/10.1108/IJSHE-07-2023-0284

Gurer, M. D., \& Akkaya, R. (2021). The influence of pedagogical beliefs
on technology acceptance: A structural equation modeling study of
pre-service mathematics teachers. \emph{Journal of Mathematics Teacher
Education, 25}(4), 479-495. https://doi.org/10.1007/s10857-021-09504-5

Karlstrom, M., \& Hamza, K. M. (2021). How do we teach planning to
pre-service teachers - A tentative model. \emph{Journal of Science
Teacher Education, 32}(6), 664-685.
https://doi.org/10.1080/1046560X.2021.1875163

Krepf, M., \& Konig, J. (2022). Structuring the lesson: An empirical
investigation of pre-service teacher decision-making during the planning
of a demonstration lesson. \emph{Journal of Education for Teaching,
49}(5), 911-926. https://doi.org/10.1080/02607476.2022.2151877

Mansfield, J. (2022). Supporting the development of pre-service
teachers' pedagogical knowledge about planning for practical work.
\emph{Journal of Science Teacher Education, 34}(3), 225-247.
https://doi.org/10.1080/1046560X.2022.2042979

Markula, A., \& Aksela, M. (2022). The key characteristics of
project-based learning: How teachers implement projects in K-12 science
education. \emph{Disciplinary and Interdisciplinary Science Education
Research, 4}(1). https://doi.org/10.1186/s43031-021-00042-x

Ozdemir-Yilmazer, M. (2025). Towards a holistic understanding of
sustainable development in teacher education: Insights from pre-service
teachers' cross-curricular lesson planning. \emph{European Journal of
Education, 60}(4). https://doi.org/10.1111/ejed.70229

Pitot, L. N., McHugh, M. L., \& Kosiak, J. (2024). Establishing a PBL
STEM framework for pre-service teachers. \emph{Education Sciences,
14}(6), 571. https://doi.org/10.3390/educsci14060571

Pleasants, J., \& Parrish, J. (2024). What makes this lesson
engineering? What makes it science? Examining the thought processes of
pre-service elementary teachers. \emph{Journal of Science Teacher
Education, 36}(2), 159-179.
https://doi.org/10.1080/1046560X.2024.2380188

Portillo-Blanco, A., Zuza, K., Gutierrez-Jimenez, E., Guisasola, J., \&
Gutierrez-Berraondo, J. (2025). Building an autonomous car: Designing,
implementing, and evaluating an integrated STEM teaching-learning
sequence for pre-service secondary teachers. \emph{Education Sciences,
15}(4), 406. https://doi.org/10.3390/educsci15040406

Singh-Pillay, A. (2023). Pre-service teachers' experience of learning
about sustainability in technology education in South Africa.
\emph{Sustainability, 15}(3), 2149. https://doi.org/10.3390/su15032149

Stinken-Rosner, L., Hofer, E., Rodenhauser, A., \& Abels, S. (2023).
Technology implementation in pre-service science teacher education based
on the transformative view of TPACK: Effects on pre-service teachers'
TPACK, behavioral orientations and actions in practice. \emph{Education
Sciences, 13}(7), 732. https://doi.org/10.3390/educsci13070732

Tellez-Acosta, M. E., Acher, A., \& McDonald, S. P. (2023). Pre-service
elementary teachers learning to plan modeling-based investigations.
\emph{Journal of Science Teacher Education, 35}(3), 276-301.
https://doi.org/10.1080/1046560X.2023.2256563

Vidal, S., \& Kuckuck, M. (2025). Pre-service teacher action competence
in education for sustainable development: A scoping review.
\emph{Sustainability, 17}(9), 3856. https://doi.org/10.3390/su17093856

Yuksel, A. O. (2025). Design-based STEM activities in teacher education
and its effect on pre-service science teachers' design thinking skills.
\emph{Journal of Science Education and Technology}.
https://doi.org/10.1007/s10956-025-10215-2

Wu, B., Peng, X., \& Hu, Y. (2021). How to foster pre-service teachers'
STEM learning design expertise through virtual internship: A
design-based research. \emph{Educational Technology Research and
Development, 69}(6), 3307-3329.
https://doi.org/10.1007/s11423-021-10063-y

UNECE. (2012). \emph{Learning for the future: Competences in education
for sustainable development} (ECE/CEP/AC.13/2012/6). United Nations
Economic Commission for Europe.
https://unece.org/fileadmin/DAM/env/esd/ESD\_Publications/Competences\_Publication.pdf

UNESCO. (2017). \emph{Education for Sustainable Development Goals:
Learning objectives}. UNESCO Publishing.
https://doi.org/10.54675/CGBA9153
