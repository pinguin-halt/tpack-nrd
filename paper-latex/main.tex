\documentclass[preprint,authoryear]{elsarticle}

% Core
\usepackage[T1]{fontenc}
\usepackage[utf8]{inputenc}
\usepackage{microtype}

% Geometry approximated from elsevier_smart_health_template.docx
\usepackage[a4paper,left=0.53in,right=0.53in,top=1.00in,bottom=0.58in]{geometry}

% Math and symbols
\usepackage{amsmath,amssymb,amsthm}

% Tables and figures
\usepackage{graphicx}
\usepackage{booktabs}
\usepackage{tabularx}
\usepackage{longtable}
\usepackage{multirow}
\usepackage{array}
\usepackage{calc}
\usepackage{siunitx}
\usepackage{caption}
\usepackage{subcaption}

% Lists and spacing
\usepackage{enumitem}
\setlist{nosep}

% Links
\usepackage[hidelinks]{hyperref}

% Body size close to Word template compact look
\AtBeginDocument{\fontsize{9pt}{13pt}\selectfont}

% Paragraph spacing similar to compact journal style
\setlength{\parindent}{1em}
\setlength{\parskip}{0.25em}

% Table and caption tuning
\captionsetup{font=small,labelfont=bf}
\setlength{\tabcolsep}{4pt}
\renewcommand{\arraystretch}{1.15}

% Pandoc table helpers
\newcounter{none}
\providecommand{\real}[1]{#1}
\providecommand{\tightlist}{\setlength{\itemsep}{0pt}\setlength{\parskip}{0pt}}

% Number formatting helper (for statistics tables)
\sisetup{
  round-mode=places,
  round-precision=3,
  detect-weight=true,
  detect-family=true,
  group-separator={,}
}


\journal{Smart Health}

\begin{document}

\begin{frontmatter}

\title{How Project-Based Learning Enhances Pre-Service Science Teachers' Integrative Lesson Planning Competence: A Structural Equation Modeling Approach}

\author[aff1]{Novi Ratna Dewi\corref{cor1}}
\ead{noviratnadewi@mail.unnes.ac.id}

\author[aff1]{Rizki Nor Amelia}
\author[aff1]{Septiko Aji}
\author[aff2]{Ismail Okta Kurniawan}

\cortext[cor1]{Corresponding author}

\address[aff1]{Faculty of Mathematics and Natural Sciences, Universitas Negeri Semarang, Central Java, Indonesia}
\address[aff2]{Directorate of Information System and Public Relation, Universitas Negeri Semarang, Central Java, Indonesia}

\begin{abstract}
Mempersiapkan calon guru IPA untuk merancang RPP yang secara simultan mengintegrasikan Technological Pedagogical Content Knowledge (TPACK), Science-Technology-Engineering-Mathematics (STEM), dan Education for Sustainable Development (ESD) merupakan tantangan kritis namun kurang diteliti dalam pendidikan guru. Penelitian ini menguji pengaruh Project-Based Learning (PjBL) terhadap pengembangan ketiga dimensi integrasi tersebut dan kualitas RPP integratif, serta memodelkan hubungan struktural di antaranya. Desain pra-eksperimen one-group pretest--posttest diterapkan pada 95 calon guru IPA di sebuah universitas negeri di Semarang, Indonesia. Data diperoleh dari penilaian berbasis rubrik terhadap RPP (14 indikator) dan kuesioner kualitas implementasi PjBL, lalu dianalisis menggunakan uji berpasangan, normalized gain (N-Gain), serta Partial Least Squares Structural Equation Modeling (PLS-SEM) dengan 5.000 iterasi bootstrap. Seluruh konstruk meningkat signifikan setelah intervensi PjBL, dengan ukuran efek besar (Cohen's d>2,3) dan N-Gain kategori Medium; namun ESD memiliki N-Gain terlemah dan tidak ada partisipan mencapai N-Gain High. Hasil PLS-SEM menunjukkan PjBL secara signifikan memprediksi ketiga dimensi integrasi dengan ukuran efek besa: TPACK (\(\beta\)=0,727; p<0,001; \(f^2\)=1,123), STEM (\(\beta\)=0,683; p<0,001; \(f^2\)=0,872), dan ESD (\(\beta\)=0,617; p<0,001; \(f^2\)=0,614). Ketiga dimensi berkontribusi signifikan terhadap kualitas RPP integratif dengan ukuran efek besar. Pola full mediation teramati: PjBL memengaruhi kualitas RPP melalui peningkatan kompetensi integrasi, dengan efek tidak langsung signifikan melalui STEM (\(\beta\)=0,330), TPACK (\(\beta\)=0,309), dan ESD (\(\beta\)=0,213); total indirect effect substansial (\(\beta\)=0,852; p<0,001) sementara direct effect tidak signifikan. Temuan ini menjadi uji empiris pertama model struktural yang menghubungkan PjBL dengan desain RPP integratif melalui TPACK, STEM, dan ESD, serta memberi panduan praktis bagi institusi pendidikan guru yang berupaya mengembangkan kompetensi desain komprehensif pada calon guru IPA.
\end{abstract}

\begin{keyword}
Project-Based Learning \sep  TPACK \sep  Pendidikan STEM \sep  Education for Sustainable Development \sep  Desain RPP Integratif \sep  PLS-SEM
\end{keyword}

\end{frontmatter}

\section{Introduction}\label{introduction}

Lanskap pendidikan IPA abad ke-21 menuntut guru memiliki kompetensi
multifaset yang melampaui penguasaan konten disiplin ilmu semata. Guru
IPA kini diharapkan merancang pembelajaran yang secara bermakna
mengintegrasikan teknologi digital, mendorong penalaran interdisipliner,
dan mengatasi tantangan keberlanjutan yang mendesak \citep{kelley2016aconceptual,unesco2017item}. Ekspektasi ini sangat tinggi bagi calon guru IPA. Dalam
persiapan profesionalnya, mereka perlu dibekali tidak hanya pemahaman
teoretis tentang berbagai kerangka pedagogik, tetapi juga keterampilan
praktis untuk menggabungkan beragam kerangka tersebut menjadi RPP yang
runtut dan saling terhubung. Namun, program pendidikan guru yang ada
sering kali masih memperlakukan aspek-aspek ini (Technological
Pedagogical Content Knowledge (TPACK), integrasi
Science-Technology-Engineering-Mathematics (STEM), dan Education for
Sustainable Development (ESD)) sebagai bagian kurikulum yang terpisah.
Akibatnya, calon guru seringkali harus menghadapi sendiri kompleksitas
merancang RPP yang integratif.

Kemampuan merancang RPP integrative (RPP yang secara simultan
menyematkan pedagogi berbasis teknologi, koneksi STEM interdisipliner,
dan perspektif keberlanjutan) merepresentasikan kompetensi profesional
tingkat tinggi yang disebut \emph{teacher design capacity} \citep{brown2009item}. Kapasitas ini bukan sekadar kumpulan keterampilan yang
berdiri sendiri, melainkan kompetensi yang muncul (emergen) yang
menuntut guru mengintegrasikan berbagai jenis pengetahuan secara
simultan \\citep{mckenney2015teacherdesign}. Satu aspek kunci integrasi ini melalui
kerangka TPACK, yaitu irisan domain pengetahuan yang memungkinkan guru
memanfaatkan teknologi untuk mendukung pedagogi yang selaras dengan
karakteristik konten \\citep{mishra2006technologicalpedagogical}. Kajian selanjutnya memperluas
logika integratif tersebut ke pendidikan STEM, di mana perancangan
pembelajaran yang efektif menuntut penggabungan yang disengaja antara
inkuiri ilmiah, pemanfaatan teknologi, desain rekayasa, dan penalaran
matematis \\citep{kelley2016aconceptual,pitot2024establishinga,portilloblanco2025buildingan}. Lebih mutakhir, tuntutan global terhadap pendidikan
keberlanjutan menambahkan dimensi integrasi ketiga kompetensi ESD
mengharuskan guru mengintegrasikan isu keberlanjutan, pendekatan inkuiri
terhadap tantangan lingkungan, serta pemikiran evaluatif atas dilema
sosio-ilmiah dalam pembelajaran IPA \\citep{purwianingsih2022programfor,unesco2017item,vidal2025preservice}.

Meskipun setiap dimensi diakui penting, temuan empiris masih menunjukkan
literatur yang terpisah-pisah. Penelitian tentang pengembangan TPACK
pada calon guru IPA cukup banyak (misalnya, \citep{offermann2025technologicalpedagogical,salleh2025howpre,stinkenrosner2023technologyimplementation}), demikian pula studi
kompetensi integrasi STEM \\citep{tucker2024examiningstem,mansour2024scienceand}
dan, meski lebih terbatas, kemampuan pedagogis ESD \\citep{purwianingsih2022programfor,vidal2025preservice}. Namun, ketiga rumpun penelitian ini
umumnya berjalan sendiri-sendiri. Masih sedikit studi yang menelaah
bagaimana calon guru mengembangkan kemampuan mengintegrasikan TPACK,
STEM, dan ESD secara bersamaan, dan belum ada yang memodelkan hubungan
struktural ketiganya sebagai jalur mediasi melalui mana intervensi
pembelajaran memengaruhi kualitas RPP integratif.

Project-Based Learning (PjBL) secara teoretis merupakan pendekatan yang
menjanjikan untuk meningkatkan kompetensi calon guru dalam merancang RPP
integratif. PjBL dipahami sebagai inkuiri jangka panjang yang
terstruktur melalui pertanyaan pendorong autentik dan diakhiri dengan
produk yang dipublikasikan \\citep{krajcik2014item}. Sejumlah studi
menunjukkan bahwa PjBL efektif meningkatkan kompetensi tersebut secara
terpisah, misalnya peningkatan TPACK melalui scaffolding PjBL \\citep{dewi2022projectbased} dan peningkatan kualitas desain unit STEM interdisipliner
pada calon guru \\citep{pitot2024establishinga,portilloblanco2025buildingan}.
Namun, jalur pengaruh PjBL terhadap kualitas RPP integrative (apakah
langsung atau melalui mediasi peningkatan TPACK, STEM, dan ESD) masih
belum dibuktikan secara empiris.

Kesenjangan ini berdampak pada teori dan praktik. Secara teoretis, perlu
dipahami apakah PjBL meningkatkan kualitas RPP terutama melalui
penguatan dimensi integrasi (TPACK/STEM/ESD) atau melalui pengaruh
langsung. Temuan ini dapat memperjelas mekanisme bagaimana intervensi
pedagogis membentuk kompetensi desain, sekaligus menyempurnakan kerangka
\emph{Teacher Design Capacity} \\citep{brown2009item} dengan merinci jalur
pembentukan kompetensi tersebut. Secara praktis, pemetaan dimensi
integrasi yang paling responsif terhadap PjBL membantu guru menyesuaikan
fokus pembelajaran dan strategi \emph{scaffolding}, termasuk memberi
dukungan tambahan pada dimensi yang belum berkembang optimal melalui
PjBL saja.

Penelitian ini menutup kesenjangan dengan mengusulkan dan menguji model
struktural yang menempatkan PjBL sebagai prediktor eksogen bagi tiga
dimensi integrasi (TPACK, STEM, dan ESD), yang selanjutnya memengaruhi
kualitas RPP integratif. Investigasi ini dipandu oleh lima pertanyaan
penelitian:

\textbf{RQ1.} Bagaimana kompetensi integrasi TPACK, STEM, dan ESD calon
guru IPA dalam desain RPP berubah dari pre- ke post-intervensi PjBL?

\textbf{RQ2.} Apakah implementasi PjBL secara signifikan mempengaruhi
kualitas desain RPP di seluruh dimensi integrasi TPACK, STEM, dan ESD?

\textbf{RQ3.} Dimensi integrasi mana (TPACK, STEM, atau ESD) yang paling
kuat dipengaruhi oleh PjBL?

\textbf{RQ4.} Bagaimana kompetensi integrasi TPACK, STEM, dan ESD
berkontribusi terhadap kualitas RPP integratif secara keseluruhan
setelah implementasi PjBL?

\textbf{RQ5.} Apakah peningkatan TPACK, STEM, dan ESD memediasi pengaruh
PjBL terhadap kualitas RPP integratif?

Penelitian ini memberi kontribusi pada literatur dalam tiga hal.
Pertama, penelitian ini menjadi uji empiris awal atas model integratif
yang menempatkan TPACK, STEM, dan ESD secara simultan sebagai mediator
hubungan antara intervensi pedagogis dan kualitas RPP. Kedua, kualitas
RPP integratif diukur sebagai konstruk tingkat tinggi melalui penilaian
rubrik (bukan self-report), sehingga pengukuran lebih kontekstual dan
valid. Ketiga, melalui PLS-SEM untuk memetakan efek langsung, tidak
langsung, dan total, penelitian ini menjelaskan secara rinci mekanisme
bagaimana PjBL membentuk kompetensi calon guru IPA dalam mendesain
pembelajaran, dengan implikasi langsung bagi perancangan kurikulum di
lembaga pendidikan guru.

\section{Literature review}\label{literature-review}

\subsection{Project-based learning dalam pendidikan
guru}\label{project-based-learning-dalam-pendidikan-guru}

Project-Based Learning (PjBL) adalah pendekatan pembelajaran yang
berpusat pada pertanyaan autentik dan kompleks untuk mendorong
investigasi berkelanjutan, inkuiri kolaboratif, serta menghasilkan
artefak nyata \\citep{krajcik2014item}. Elemen kunci PjBL mencakup
pertanyaan pendorong berbasis masalah dunia nyata, inkuiri melalui
penelitian, kolaborasi, produksi artefak yang dibagikan secara publik,
dan refleksi terstruktur \\citep{bell2010projectbased,krajcik2014item}. Sejumlah
studi mendukung efektivitasnya dalam meningkatkan kompetensi calon guru,
termasuk TPACK dan kemampuan desain pembelajaran melalui scaffolding
\\citep{dewi2022projectbased}, kompetensi pedagogis melalui pembelajaran proyek
\\citep{novallyan2025optimizationof}, serta kompetensi terkait teknologi \\citep{akbulut2021developingpre}. Temuan lain menunjukkan PjBL/PBL cenderung memperkuat
kolaborasi, produksi artefak, dan praktik inkuiri, meski elemen seperti
pertanyaan pendorong yang sepenuhnya dihasilkan siswa masih menantang
\\citep{markula2022thekey}, dan penelitian terbaru juga melaporkan
peningkatan pada kualitas desain unit STEM interdisipliner serta design
thinking calon guru \\citep{pitot2024establishinga,portilloblanco2025buildingan,yuksel2025designbased}. Secara umum, studi-studi tersebut menempatkan PjBL
sebagai intervensi yang layak untuk meningkatkan berbagai dimensi
kompetensi guru, namun pengaruh simultannya terhadap banyak dimensi
integrasi dalam satu model struktural masih jarang diteliti.

\subsection{Technological pedagogical content knowledge
(tpack)}\label{technological-pedagogical-content-knowledge-tpack}

Kerangka TPACK yang diperkenalkan oleh \citet{mishra2006technologicalpedagogical}, dengan
dasar konsep Pedagogical Content Knowledge (PCK) dari \citet{shulman1986thosewho},
menjelaskan pengetahuan yang saling beririsan untuk mendukung pengajaran
efektif berbantuan teknologi. TPACK mencakup tujuh komponen: Technology
Knowledge (TK), Pedagogical Knowledge (PK), Content Knowledge (CK), tiga
irisan berpasangan (TPK, TCK, PCK), serta inti integratif TPACK yang
merepresentasikan kemampuan mengajarkan konten spesifik menggunakan
teknologi yang tepat melalui strategi pedagogis yang sesuai.

Pengukuran TPACK bergeser dari survei laporan diri menuju asesmen
berbasis kinerja, misalnya evaluasi rencana pembelajaran dengan rubrik
yang menangkap integrasi teknologi--pedagogi--konten \\citep{offermann2025technologicalpedagogical}; Sejumlah studi SEM terbaru menunjukkan keterkaitan TPACK dengan
konstruk lain: \\citet{mansour2024scienceand} menemukan hubungan signifikan
antara komponen TPACK dan efikasi pengajaran STEM, \\citet{salleh2025howpre}
menegaskan peran pengetahuan konten dan efikasi diri teknologi sebagai
prediktor, dan \\citet{stinkenrosner2023technologyimplementation} melaporkan modul
implementasi teknologi yang terstruktur dapat meningkatkan TPACK serta
orientasi perilaku penggunaan teknologi. Namun, TPACK masih jarang
dimodelkan secara simultan bersama STEM dan ESD dalam satu kerangka
struktural, sehingga menjadi celah yang ingin dijembatani dalam
penelitian ini.

\subsection{Pendidikan STEM dan desain pembelajaran
integratif}\label{pendidikan-stem-dan-desain-pembelajaran-integratif}

Pendidikan STEM dipahami sebagai integrasi yang disengaja antara sains,
teknologi, teknik, dan matematika, dan kini semakin kuat menjadi
filosofi pendidikan sekaligus kerangka kurikulum \\citep{kelley2016aconceptual}. Perbedaan antara STEM yang diajarkan terpisah per disiplin dan
STEM integrative berdampak langsung pada desain rencana pembelajaran.
\\citet{kelley2016aconceptual} menawarkan kerangka STEM terintegrasi yang
menekankan pembelajaran situated, desain teknik sebagai strategi
pedagogis, inkuiri ilmiah sebagai proses membangun pengetahuan, serta
pemikiran matematis sebagai dasar analitis; kualitas rencana
pembelajaran dapat ditinjau dari akurasi-kedalaman konten sains,
integrasi teknologi yang purposif, hadirnya pemikiran desain teknik, dan
penerapan penalaran matematis.

Bagi calon guru IPA, kompetensi integrasi STEM menuntut pengembangan
efikasi diri STEM (keyakinan dan kemampuan merancang pembelajaran
interdisipliner (\\citep{tucker2024examiningstem}) yang dapat diperkuat melalui
perencanaan kolaboratif lintas mata kuliah dan integrasi eksplisit
engineering design dalam kerangka PBL-STEM \\citep{pitot2024establishinga,portilloblanco2025buildingan}, serta latihan desain berkelanjutan
berbasis umpan balik \\citep{wu2021howto}. Tantangan ini cenderung lebih
berat pada integrasi yang terkait ESD karena calon guru perlu memasukkan
konteks keberlanjutan yang sering kurang familiar dibanding topik STEM
tradisional.

\subsection{Education for sustainable development (ESD) dalam pengajaran
IPA}\label{education-for-sustainable-development-esd-dalam-pengajaran-ipa}

Education for Sustainable Development (ESD) merupakan paradigma
pendidikan global yang membekali peserta didik dengan pengetahuan,
keterampilan, nilai, dan sikap untuk merespons tantangan keberlanjutan
yang saling terhubung \\citep{unesco2017item}. Dalam pendidikan guru IPA, ESD
menuntut kompetensi yang melampaui pedagogi IPA tradisional, yaitu
kemampuan mengaitkan konsep ilmiah dengan isu keberlanjutan (ESD-PCK),
menerapkan pendekatan inkuiri untuk mengkaji persoalan lingkungan dan
sosial (ESD-INQ), serta menumbuhkan pemikiran evaluatif terhadap dilema
sosio-ilmiah dan berbagai trade-off (ESD-EVA) \\citep{unece2012item}.

Sejumlah studi menunjukkan upaya integrasi ESD pada pendidikan calon
guru, misalnya program \\citet{purwianingsih2022programfor} yang mengintegrasikan
ESD ke dalam TPACK calon guru biologi. \\citet{shumba2013mainstreamingesd} juga
menegaskan bahwa guru memerlukan pengetahuan pedagogis-konten spesifik
ESD. Namun, integrasi ESD masih relatif kurang berkembang dibandingkan
TPACK dan integrasi STEM, termasuk dalam desain rencana pembelajaran;
selain itu, action competence ESD calon guru cenderung masih lemah dan
sering diukur lewat self-report potong lintang, sehingga asesmen
berbasis kinerja dan tugas desain menjadi penting \\citep{vidal2025preservice,singhpillay2023preservice}. Temuan lain menunjukkan pengetahuan SDG calon
guru pada pelatihan awal juga sering terbatas, sehingga dukungan
kurikuler ESD yang eksplisit tetap diperlukan \\citep{calero2024astudy};
implikasinya, kompetensi ESD kemungkinan memerlukan scaffolding yang
lebih intensif atau lebih panjang, dan responsnya terhadap intervensi
PjBL jangka pendek dapat berbeda.

\subsection{Perencanaan pembelajaran integratif sebagai kompetensi
desain
guru}\label{perencanaan-pembelajaran-integratif-sebagai-kompetensi-desain-guru}

Konsep \emph{teacher design capacity} yang diperkenalkan oleh \citet{brown2009item} dan dielaborasi oleh \citet{mckenney2015teacherdesign} menjadi landasan
teoretis untuk memandang perencanaan pembelajaran integratif sebagai
kompetensi profesional tingkat tinggi. \\citet{brown2009item} menekankan bahwa
penggunaan kurikulum yang efektif menuntut guru berperan sebagai
desainer yang secara aktif menafsirkan, menyesuaikan, dan mengembangkan
materi pembelajaran sesuai konteks; kapasitas ini bersifat dinamis dan
berkembang melalui keterlibatan dalam tugas desain serta umpan balik.
\\citet{mckenney2015teacherdesign} menambahkan bahwa kompetensi desain guru ditopang
oleh basis pengetahuan spesifik (pengetahuan teknologi, pedagogis, dan
materi subjek) yang terbentuk dari interaksi antara pengetahuan
personal, pengetahuan formal, dan pengalaman praktik.

Dalam studi ini, kompetensi perencanaan pembelajaran integratif
dioperasionalisasikan sebagai kualitas rencana pembelajaran yang secara
simultan mengintegrasikan dimensi TPACK, STEM, dan ESD. Rencana
pembelajaran dipahami sebagai konstruk tingkat tinggi
(\emph{higher-order construct}, HOC) yang merepresentasikan kualitas
emergen dari integrasi yang koheren antardimensi, bukan sekadar
penjumlahan skor komponen. Karena itu, asesmen berbasis rubrik (bukan
laporan diri) digunakan untuk menangkap kompetensi desain yang
benar-benar didemonstrasikan, bukan yang hanya dipersepsikan.

\subsection{Kerangka konseptual dan model yang
dihipotesiskan}\label{kerangka-konseptual-dan-model-yang-dihipotesiskan}

Berdasarkan landasan teoretis tersebut, studi ini mengusulkan sebuah
model struktural yang menempatkan PjBL sebagai konstruk eksogen yang
memengaruhi tiga konstruk endogen orde pertama, yaitu kompetensi
integrasi TPACK, STEM, dan ESD, yang selanjutnya berkontribusi pada
kualitas rencana pembelajaran integratif (lihat Gambar 1). Model ini
mengasumsikan adanya efek langsung (PjBL -> TPACK, PjBL
-> STEM, PjBL -> ESD) serta efek tidak
langsung (PjBL -> TPACK/STEM/ESD -> Kualitas
Rencana Pembelajaran Integratif). Dalam kerangka ini, ketiga dimensi
integrasi tersebut dihipotesiskan berperan sebagai mediator antara PjBL
dan kualitas rencana pembelajaran.

\begin{figure}
\centering
\includegraphics[width=5.24934in,height=1.72792in,alt={Model struktural yang dihipotesiskan}]{figures/media/model.png}
\caption{Model struktural yang dihipotesiskan}
\end{figure}

\emph{Gambar 1.} Model struktural yang dihipotesiskan.

PjBL mempengaruhi tiga dimensi integrasi (TPACK, STEM, ESD) yang
kemudian berkontribusi terhadap kualitas rencana pembelajaran
integratif. Garis putus-putus menunjukkan jalur langsung PjBL ke RPP,
yang dihipotesiskan tidak signifikan (mediasi penuh).

Justifikasi teoretis untuk setiap jalur adalah sebagai berikut:

\begin{enumerate}
\def\labelenumi{\arabic{enumi}.}
\item
  PjBL -> TPACK: Penekanan PjBL pada penciptaan artefak dan
  investigasi berbasis inkuiri secara alami memerlukan mobilisasi
  teknologi untuk pedagogi spesifik konten \\citep{dewi2022projectbased}.
\item
  PjBL -> STEM: Pertanyaan pendorong autentik PjBL biasanya
  mencakup berbagai disiplin STEM, memerlukan pemikiran desain
  interdisipliner \\citep{krajcik2014item}.
\item
  PjBL -> ESD: Fokus PjBL pada masalah dunia nyata
  menciptakan peluang untuk mengatasi isu keberlanjutan, meskipun
  kekuatan tautan ini mungkin bergantung pada scaffolding eksplisit
  \\citep{purwianingsih2022programfor}.
\item
  TPACK/STEM/ESD -> Kualitas Rencana Pembelajaran
  Integratif: Setiap dimensi menyumbang elemen desain
  substantif---integrasi teknologi, koneksi interdisipliner, dan
  perspektif keberlanjutan---yang secara kolektif menentukan kualitas
  rencana pembelajaran integratif \\citep{brown2009item,mckenney2015teacherdesign}.
\item
  Mediasi: PjBL dihipotesiskan mempengaruhi kualitas rencana
  pembelajaran tidak secara langsung tetapi melalui peningkatan
  kompetensi integrasi, konsisten dengan pandangan bahwa intervensi
  pedagogis beroperasi dengan mengembangkan basis pengetahuan
  profesional spesifik yang kemudian termanifestasi dalam kinerja
  desain.
\end{enumerate}

Hipotesis berikut dirumuskan berdasarkan kerangka teoretis

H1: PjBL secara positif dan signifikan mempengaruhi kompetensi integrasi
TPACK, STEM, dan ESD.

H2: Kompetensi integrasi TPACK, STEM, dan ESD secara signifikan
berkontribusi terhadap kualitas RPP integratif.

H3: TPACK, STEM, dan ESD memediasi hubungan antara PjBL dan kualitas RPP
integratif.

\section{Methods}\label{methods}

\subsection{Desain penelitian}\label{desain-penelitian}

Penelitian ini menerapkan desain pra-eksperimental satu kelompok
pretest--posttest \\citep{creswell2018item}. Partisipan menyusun
rencana pembelajaran sebelum dan sesudah intervensi PjBL; kualitasnya
dinilai dengan rubrik terstandar pada tiga dimensi integrasi, yaitu
TPACK, STEM, dan ESD. Desain ini dipilih karena penelitian berfokus pada
pendokumentasian perubahan kompetensi setelah intervensi serta pemodelan
hubungan struktural antara kualitas implementasi PjBL, dimensi
integrasi, dan kualitas rencana pembelajaran menggunakan PLS-SEM, bukan
pada pembuktian kausal yang ketat melalui perbandingan antar-kelompok.

\subsection{Partisipan}\label{partisipan}

Partisipan penelitian ini terdiri atas 95 calon guru IPA pada program
studi Pendidikan IPA sebuah universitas Pendidikan negeri di Semarang
Indonesia. Seluruh partisipan merupakan mahasiswa sarjana tahun ketiga
atau keempat yang telah menuntaskan mata kuliah dasar konten sains,
pedagogi umum, dan teknologi pendidikan. Sampel dipilih secara purposif,
yakni mahasiswa yang sekaligus mengambil mata kuliah Strategi dan Desain
Pembelajaran IPA sebagai konteks natural pelaksanaan intervensi PjBL.
Ukuran sampel (N=95) melampaui ambang minimum PLS-SEM; \\citet{hair2022item} merekomendasikan minimal 10 kali jumlah maksimum jalur struktural
yang menuju satu konstruk (pada model ini, empat jalur menuju konstruk
RPP, sehingga minimum 40).

\subsection{Intervensi: implementasi
PjBL}\label{intervensi-implementasi-pjbl}

Intervensi PjBL dilaksanakan selama 16 kali pertemuan melalui sesi yang
terstruktur. Partisipan mengikuti siklus proyek untuk menyusun rencana
pembelajaran integratif yang memadukan TPACK, STEM, dan ESD secara
simultan. Intervensi mengacu pada lima tahap yang diadaptasi dari
\\citet{krajcik2014item}: (1) Orientasi dan pertanyaan pendorong; (2)
Perencanaan dan investigasi; (3) Penciptaan artefak. (4) Peer review dan
revisi; (5) Presentasi dan refleksi.

Kualitas implementasi PjBL dievaluasi melalui instrumen observasi yang
diisi oleh 10 observer.

\subsection{Instrumen}\label{instrumen}

\subsection{Rubrik rencana pembelajaran integratif
(pretest-posttest)}\label{rubrik-rencana-pembelajaran-integratif-pretest-posttest}

Instrumen utama adalah rubrik untuk mengevaluasi kualitas rencana
pembelajaran integratif, diskor pada skala Likert empat poin (1 = tidak
memenuhi kriteria, 2 = sebagian memenuhi, 3 = memenuhi, 4 = melampaui
kriteria). Rubrik menilai tiga dimensi integrasi yang terdiri dari 14
indikator:

\begin{enumerate}
\def\labelenumi{\alph{enumi}.}
\item
  TPACK (7 indikator): TK, PK, CK, TPK, TCK, PCK, TPACK integratif.
\item
  STEM (4 indikator): Integrasi konten Sains (S), Aplikasi Teknologi
  (T), Proses desain Teknik/Engineering (E), dan Penalaran Matematis
  (M).
\item
  ESD (3 indikator): ESD-Pedagogical Content Knowledge (ESD-PCK),
  ESD-Inquiry (ESD-INQ), dan ESD-Evaluative thinking (ESD-EVA).
\end{enumerate}

Skor komposit setiap dimensi dihitung sebagai rata-rata (mean) dari
indikator penyusunnya. Skor kualitas keseluruhan rencana pembelajaran
integratif (RPPInt\_total) dihitung sebagai grand mean dari 14
indikator. Rubrik disusun melalui penilaian ahli oleh lima spesialis
pendidikan IPA dan menunjukkan validitas konten yang memadai.

\subsection{Instrumen observasi implementasi
PjBL}\label{instrumen-observasi-implementasi-pjbl}

Kualitas implementasi PjBL diukur menggunakan instrumen observasi lima
item, dengan setiap item diskor pada skala 1--4 yang sesuai dengan lima
tahap PjBL. Instrumen diisi oleh instruktur mata kuliah yang
mengobservasi proses implementasi.

\subsection{Prosedur pengumpulan data}\label{prosedur-pengumpulan-data}

Pengumpulan data mengikuti timeline tiga tahap: (a) pretest; (b)
intervensi PjBL; dan (c) posttest. Rencana pembelajaran dianonimkan
sebelum penskoran untuk mengurangi bias penilai.

\subsection{Analisis data}\label{analisis-data}

Analisis data berlangsung dalam dua fase.

\begin{enumerate}
\def\labelenumi{\alph{enumi}.}
\tightlist
\item
  Fase 1: Perbandingan pre-post (RQ1)
\end{enumerate}

Analisis yang dilakukan untuk menjawab RQ1: Statistik deskriptif (mean,
standar deviasi, minimum, maksimum); Uji normalitas (uji Shapiro-Wilk),
Uji inferensial berpasangan (paired-samples t-test untuk data
terdistribusi normal, dan uji Wilcoxon signed-rank untuk data
terdistribusi tidak normal dengan \(\alpha = 0,05)\); Effect sizes
(Cohen's \(d\) untuk uji parametrik dan korelasi rank-biserial \(r\)
untuk uji non-parametrik). Effect sizes diinterpretasikan mengikuti
\\citet{cohen1988item}: kecil (\(d = 0,2\)), sedang (\(d = 0,5\)), dan besar
(\(d = 0,8\)). Normalized gain (N-Gain) dihitung menggunakan formula
\\citet{hake1998interactiveengagement}: N-Gain \(= \frac{post - pre}{\max - pre}\), di mana max
\(= 4\) (skor rubrik maksimum). Nilai N-Gain dikategorikan sebagai
Tinggi (\(> 0,7\)), Sedang (\(0,3\)--\(0,7\)), atau Rendah (\(< 0,3\)).
Seluruh analisis Fase 1 dilakukan di Python 3.11 menggunakan pandas,
scipy, dan pingouin.

\begin{enumerate}
\def\labelenumi{\alph{enumi}.}
\setcounter{enumi}{1}
\tightlist
\item
  Fase 2: Structural equation modeling (RQ2--RQ5)
\end{enumerate}

Partial Least Squares Structural Equation Modeling (PLS-SEM) digunakan
untuk menjawab RQ2 sampai RQ5 karena: (a) sifat
eksploratori-konfirmatori penelitian; (b) inklusi konstruk indikator
tunggal; (c) ukuran sampel moderat (\(N = 95\)); dan (d) kapasitas
PLS-SEM untuk menangani data non-normal dan pengukuran formatif \\citep{hair2022item}.

\emph{Spesifikasi model.} Model struktural terdiri dari lima konstruk:
PjBL (eksogen), TPACK, STEM, ESD, dan RPP (endogen). Seluruh konstruk
dispesifikasikan sebagai reflektif (Mode A). Jalur struktural mencakup
tujuh hubungan: PjBL \(\rightarrow\) TPACK, PjBL \(\rightarrow\) STEM,
PjBL \(\rightarrow\) ESD, PjBL \(\rightarrow\) RPP (langsung), TPACK
\(\rightarrow\) RPP, STEM \(\rightarrow\) RPP, dan ESD \(\rightarrow\)
RPP. Konstruk RPP dioperasionalisasikan sebagai konstruk indikator
tunggal menggunakan skor kualitas rencana pembelajaran integratif
komposit (RPPInt\_total\_post), dengan loading indikator difiksasi ke
1,000. Matriks data untuk analisis SEM menggunakan skor posttest untuk
TPACK, STEM, ESD, dan RPP, serta skor observasi PjBL, menghasilkan 19
variabel manifes.

\emph{Evaluasi model pengukuran.} Model outer (pengukuran) dinilai
menggunakan kriteria PLS-SEM standar \\citep{hair2022item}:

\begin{enumerate}
\def\labelenumi{\alph{enumi}.}
\item
  Reliabilitas indikator: outer loadings \(\geq 0,708\) (indikator
  antara 0,40 dan 0,70 dipertahankan jika penghapusannya tidak
  meningkatkan AVE atau CR, mengikuti rekomendasi \\citet{hair2022item}
  untuk
  penelitian eksploratori).
\item
  Validitas konvergen: Average Variance Extracted (AVE) \(\geq 0,50\).
\item
  Reliabilitas konsistensi internal: Composite Reliability (CR)
  \(\geq 0,70\) dan Cronbach's \(\alpha \geq 0,70\).
\item
  Validitas diskriminan: rasio Heterotrait-Monotrait (HTMT) \(< 0,90\)
  \\citep{henseler2015anew}, dilengkapi dengan kriteria Fornell-Larcker.
\end{enumerate}

\emph{Evaluasi model struktural.} Model inner (struktural) dinilai
melalui: Koefisien jalur (\(\beta\)); Signifikansi statistic; Koefisien
determinasi (\(R^{2}\)); Effect size (\(f^{2}\)); Relevansi prediktif
(\(Q^{2}\))

\emph{Analisis komparatif (RQ3).} Pengaruh relatif PjBL terhadap setiap
dimensi integrasi dinilai dengan membandingkan koefisien jalur dan nilai
\(f^{2}\) terkait.

\emph{Analisis mediasi (RQ5).} Efek tidak langsung dihitung sebagai
produk koefisien jalur penyusun. Signifikansi statistik efek tidak
langsung ditentukan melalui metode confidence interval bootstrap
\\citep{preacher2008asymptoticand}: efek tidak langsung dianggap signifikan jika
confidence interval percentile 95\% tidak mencakup nol. Uji Sobel juga
dihitung sebagai cross-check. Variance Accounted For (VAF) dihitung
untuk mengklasifikasikan tipe mediasi: mediasi penuh (VAF \(> 80\%\)),
mediasi parsial (\(20\% <\) VAF \(< 80\%\)), atau tanpa mediasi (VAF
\(< 20\%\)) \\citep{hair2022item}.

\emph{Perangkat lunak.} Analisis PLS-SEM dilakukan menggunakan paket
Python plspm (versi 0.5.7) dengan random seed 42 untuk reprodusibilitas.
Seluruh prosedur bootstrap menggunakan 5.000 iterasi dengan konfigurasi
single-process untuk memastikan eksekusi deterministik. Gambar
dihasilkan menggunakan matplotlib.

\section{Results}\label{results}

\subsection{Statistik deskriptif}\label{statistik-deskriptif}

Statistik deskriptif untuk seluruh konstruk dilakukan sebelum uji
hipotesis. Tabel 1 menyajikan rata-rata dan simpangan baku tingkat
konstruk pada pretest dan posttest.

\textbf{Tabel 1.} Statistik deskriptif per konstruk (pretest vs
posttest)

{\def\LTcaptype{none} % do not increment counter
\begin{longtable}[]{@{}
  >{\raggedright\arraybackslash}p{(\linewidth - 12\tabcolsep) * \real{0.1529}}
  >{\raggedright\arraybackslash}p{(\linewidth - 12\tabcolsep) * \real{0.1294}}
  >{\raggedright\arraybackslash}p{(\linewidth - 12\tabcolsep) * \real{0.1294}}
  >{\raggedright\arraybackslash}p{(\linewidth - 12\tabcolsep) * \real{0.1294}}
  >{\raggedright\arraybackslash}p{(\linewidth - 12\tabcolsep) * \real{0.1294}}
  >{\raggedright\arraybackslash}p{(\linewidth - 12\tabcolsep) * \real{0.1294}}
  >{\raggedright\arraybackslash}p{(\linewidth - 12\tabcolsep) * \real{0.1765}}@{}}
\toprule\noalign{}
\begin{minipage}[b]{\linewidth}\raggedright
Konstruk
\end{minipage} & \begin{minipage}[b]{\linewidth}\raggedright
\[N\]
\end{minipage} & \begin{minipage}[b]{\linewidth}\raggedright
Pre \(M\)
\end{minipage} & \begin{minipage}[b]{\linewidth}\raggedright
Pre \(SD\)
\end{minipage} & \begin{minipage}[b]{\linewidth}\raggedright
Post \(M\)
\end{minipage} & \begin{minipage}[b]{\linewidth}\raggedright
Post \(SD\)
\end{minipage} & \begin{minipage}[b]{\linewidth}\raggedright
\[M_{diff}\]
\end{minipage} \\
\midrule\noalign{}
\endhead
\bottomrule\noalign{}
\endlastfoot
TPACK & 95 & 2,306 & 0,390 & 3,319 & 0,294 & 1,013 \\
STEM & 95 & 2,172 & 0,393 & 3,222 & 0,378 & 1,051 \\
ESD & 95 & 1,929 & 0,328 & 2,726 & 0,259 & 0,798 \\
RPP Integratif & 95 & 2,136 & 0,256 & 3,089 & 0,237 & 0,954 \\
\end{longtable}
}

Sebagaimana ditunjukkan pada Tabel 1, seluruh konstruk menunjukkan
peningkatan substansial setelah intervensi PjBL (terlihat dari nilai
\(M_{diff})\). Perbedaan rata-rata terbesar diamati pada STEM, diikuti
TPACK, RPP Integratif, dan ESD.

\begin{figure}
\centering
\includegraphics[width=4.8in,alt={Gambar 2. Perbandingan Skor Rata-rata Pre-test vs Post-test per Konstruk}]{figures/media/fig1_pre_post_comparison.png}
\caption{Gambar 2. Perbandingan Skor Rata-rata Pre-test vs Post-test per Konstruk}
\end{figure}

\subsection{RM1: Perubahan pre-post}\label{rm1-perubahan-pre-post}

\subsection{Uji normalitas dan uji berpasangan dan ukuran
efek}\label{uji-normalitas-dan-uji-berpasangan-dan-ukuran-efek}

Normalitas dinilai menggunakan uji Shapiro-Wilk. Tabel 2 merangkum
hasilnya.

\textbf{Tabel 2.} Hasil uji normalitas shapiro-wilk untuk skor selisih

{\def\LTcaptype{none} % do not increment counter
\begin{longtable}[]{@{}
  >{\raggedright\arraybackslash}p{(\linewidth - 6\tabcolsep) * \real{0.1806}}
  >{\raggedright\arraybackslash}p{(\linewidth - 6\tabcolsep) * \real{0.1111}}
  >{\raggedright\arraybackslash}p{(\linewidth - 6\tabcolsep) * \real{0.1111}}
  >{\raggedright\arraybackslash}p{(\linewidth - 6\tabcolsep) * \real{0.1250}}@{}}
\toprule\noalign{}
\begin{minipage}[b]{\linewidth}\raggedright
Konstruk
\end{minipage} & \begin{minipage}[b]{\linewidth}\raggedright
\[W\]
\end{minipage} & \begin{minipage}[b]{\linewidth}\raggedright
\[p\]
\end{minipage} & \begin{minipage}[b]{\linewidth}\raggedright
Normal
\end{minipage} \\
\midrule\noalign{}
\endhead
\bottomrule\noalign{}
\endlastfoot
TPACK & 0,968 & ,020 & Tidak \\
STEM & 0,977 & ,088 & Ya \\
ESD & 0,990 & ,704 & Ya \\
RPP Integratif & 0,986 & ,418 & Ya \\
\end{longtable}
}

Sebagaimana ditunjukkan pada Tabel 2, skor selisih untuk STEM, ESD, dan
RPP Integratif terdistribusi normal (\(p > ,05\)). Namun, skor selisih
TPACK menyimpang signifikan dari normalitas. Oleh karena itu,
paired-samples t-test diterapkan pada STEM, ESD, dan RPP Integratif,
sementara Wilcoxon signed-rank test digunakan untuk TPACK.

\textbf{Tabel 3.} Hasil uji berpasangan

{\def\LTcaptype{none} % do not increment counter
\begin{longtable}[]{@{}
  >{\raggedright\arraybackslash}p{(\linewidth - 10\tabcolsep) * \real{0.1591}}
  >{\raggedright\arraybackslash}p{(\linewidth - 10\tabcolsep) * \real{0.1364}}
  >{\centering\arraybackslash}p{(\linewidth - 10\tabcolsep) * \real{0.2273}}
  >{\centering\arraybackslash}p{(\linewidth - 10\tabcolsep) * \real{0.1477}}
  >{\raggedright\arraybackslash}p{(\linewidth - 10\tabcolsep) * \real{0.1364}}
  >{\raggedright\arraybackslash}p{(\linewidth - 10\tabcolsep) * \real{0.1705}}@{}}
\toprule\noalign{}
\begin{minipage}[b]{\linewidth}\raggedright
Konstruk
\end{minipage} & \begin{minipage}[b]{\linewidth}\raggedright
Uji
\end{minipage} & \begin{minipage}[b]{\linewidth}\centering
Statistik
\end{minipage} & \begin{minipage}[b]{\linewidth}\centering
\[p\]
\end{minipage} & \begin{minipage}[b]{\linewidth}\raggedright
Cohen's \(d\)
\end{minipage} & \begin{minipage}[b]{\linewidth}\raggedright
Interpretasi
\end{minipage} \\
\midrule\noalign{}
\endhead
\bottomrule\noalign{}
\endlastfoot
TPACK & Wilcoxon & \(W = 0,0\) & \(< ,001\) & 2,705 & Besar \\
STEM & Paired t & \(t(94) = 25,90\) & \(< ,001\) & 2,657 & Besar \\
ESD & Paired t & \(t(94) = 22,79\) & \(< ,001\) & 2,338 & Besar \\
RPP Integratif & Paired t & \(t(94) = 40,88\) & \(< ,001\) & 4,194 &
Besar \\
\end{longtable}
}

Sebagaimana ditunjukkan pada Tabel 3, keempat konstruk menunjukkan
peningkatan signifikan secara statistik dari pretest ke posttest
(\(p < ,001\)). Ukuran efek seragam besar, dengan nilai Cohen's \(d\)
melebihi 2,3 untuk seluruh konstruk.

\subsection{Normalized gain}\label{normalized-gain}

Proporsi peningkatan maksimum yang dapat dicapai diukur menggunakan
normalized gain (N-Gain). Tabel 4 menyajikan ringkasan N-Gain, sementara
Gambar 3. memvisualisasikan distribusi kategori gain.

\textbf{Tabel 4.} Ringkasan N-Gain

{\def\LTcaptype{none} % do not increment counter
\begin{longtable}[]{@{}
  >{\raggedright\arraybackslash}p{(\linewidth - 12\tabcolsep) * \real{0.1605}}
  >{\raggedright\arraybackslash}p{(\linewidth - 12\tabcolsep) * \real{0.1358}}
  >{\raggedright\arraybackslash}p{(\linewidth - 12\tabcolsep) * \real{0.1358}}
  >{\raggedright\arraybackslash}p{(\linewidth - 12\tabcolsep) * \real{0.1358}}
  >{\raggedright\arraybackslash}p{(\linewidth - 12\tabcolsep) * \real{0.1358}}
  >{\raggedright\arraybackslash}p{(\linewidth - 12\tabcolsep) * \real{0.1358}}
  >{\raggedright\arraybackslash}p{(\linewidth - 12\tabcolsep) * \real{0.1358}}@{}}
\toprule\noalign{}
\begin{minipage}[b]{\linewidth}\raggedright
Konstruk
\end{minipage} & \begin{minipage}[b]{\linewidth}\raggedright
N-Gain \(M\)
\end{minipage} & \begin{minipage}[b]{\linewidth}\raggedright
\[SD\]
\end{minipage} & \begin{minipage}[b]{\linewidth}\raggedright
Kategori
\end{minipage} & \begin{minipage}[b]{\linewidth}\raggedright
High (\%)
\end{minipage} & \begin{minipage}[b]{\linewidth}\raggedright
Medium (\%)
\end{minipage} & \begin{minipage}[b]{\linewidth}\raggedright
Low (\%)
\end{minipage} \\
\midrule\noalign{}
\endhead
\bottomrule\noalign{}
\endlastfoot
TPACK & 0,596 & 0,160 & Medium & 26,3 & 72,6 & 1,1 \\
STEM & 0,574 & 0,179 & Medium & 24,2 & 69,5 & 6,3 \\
ESD & 0,376 & 0,129 & Medium & 0,0 & 71,6 & 28,4 \\
RPP Integratif & 0,513 & 0,103 & Medium & 3,2 & 95,8 & 1,1 \\
\end{longtable}
}

Sebagaimana ditunjukkan pada Tabel 4, seluruh konstruk mencapai kategori
Medium gain (\(0,3\)--\(0,7\)). TPACK memperoleh rata-rata N-Gain
tertinggi, diikuti STEM, RPP Integratif, dan ESD.

\begin{figure}
\centering
\includegraphics[width=3.57087in,height=2.08571in,alt={Gambar 3. Distribusi kategori N-Gain setiap konstruk}]{figures/media/fig2_ngain_distribution.png}
\caption{Gambar 3. Distribusi kategori N-Gain setiap konstruk}
\end{figure}

Gambar 3. Distribusi kategori N-Gain setiap konstruk

Analisis pre-post (RM1) menunjukkan bahwa PjBL secara signifikan
meningkatkan keempat konstruk dengan ukuran efek besar, meskipun ESD
menunjukkan gain terkecil. Untuk memahami hubungan struktural antar
konstruk ini, kami beralih ke analisis PLS-SEM yang menjawab RM2--RM5.

\subsection{RM2--RM5: Analisis PLS-SEM}\label{rm2rm5-analisis-pls-sem}

\subsection{Evaluasi model pengukuran}\label{evaluasi-model-pengukuran}

Evaluasi model pengukuran untuk memastikan reliabilitas dan validitas
yang memadai sebelum uji hipotesis struktural. Tabel 5 menyajikan outer
loadings untuk seluruh indikator.

\textbf{Tabel 5.} Outer loadings

{\def\LTcaptype{none} % do not increment counter
\begin{longtable}[]{@{}
  >{\raggedright\arraybackslash}p{(\linewidth - 6\tabcolsep) * \real{0.1528}}
  >{\raggedright\arraybackslash}p{(\linewidth - 6\tabcolsep) * \real{0.2083}}
  >{\raggedright\arraybackslash}p{(\linewidth - 6\tabcolsep) * \real{0.1389}}
  >{\raggedright\arraybackslash}p{(\linewidth - 6\tabcolsep) * \real{0.2361}}@{}}
\toprule\noalign{}
\begin{minipage}[b]{\linewidth}\raggedright
Konstruk
\end{minipage} & \begin{minipage}[b]{\linewidth}\raggedright
Indikator
\end{minipage} & \begin{minipage}[b]{\linewidth}\raggedright
Loading
\end{minipage} & \begin{minipage}[b]{\linewidth}\raggedright
\[\geq 0,708\]
\end{minipage} \\
\midrule\noalign{}
\endhead
\bottomrule\noalign{}
\endlastfoot
ESD & ESD-EVA & 0,780 & Ya \\
ESD & ESD-INQ & 0,894 & Ya \\
ESD & ESD-PCK & 0,867 & Ya \\
PjBL & PjBL01 & 0,815 & Ya \\
PjBL & PjBL02 & 0,804 & Ya \\
PjBL & PjBL03 & 0,835 & Ya \\
PjBL & PjBL04 & 0,850 & Ya \\
PjBL & PjBL05 & 0,840 & Ya \\
TPACK & TK & 0,689 & Tidak \\
TPACK & PK & 0,224 & Tidak \\
TPACK & CK & 0,673 & Tidak \\
TPACK & TPK & 0,760 & Ya \\
TPACK & TCK & 0,795 & Ya \\
TPACK & PCK & 0,400 & Tidak \\
TPACK & TPACK\_int & 0,759 & Ya \\
STEM & Science & 0,484 & Tidak \\
STEM & Technology & 0,713 & Ya \\
STEM & Engineering & 0,577 & Tidak \\
STEM & Mathematics & 0,916 & Ya \\
RPP & RPPInt\_total & 1,000 & Ya \\
\end{longtable}
}

Indikator dengan loading antara 0,40 dan 0,70 dipertahankan mengikuti
rekomendasi \\citet{hair2022item} untuk penelitian eksploratoris.

\textbf{Tabel 6a.} Reliabilitas konstruk dan validitas konvergen

{\def\LTcaptype{none} % do not increment counter
\begin{longtable}[]{@{}
  >{\raggedright\arraybackslash}p{(\linewidth - 6\tabcolsep) * \real{0.1528}}
  >{\raggedright\arraybackslash}p{(\linewidth - 6\tabcolsep) * \real{0.1111}}
  >{\raggedright\arraybackslash}p{(\linewidth - 6\tabcolsep) * \real{0.1111}}
  >{\raggedright\arraybackslash}p{(\linewidth - 6\tabcolsep) * \real{0.2083}}@{}}
\toprule\noalign{}
\begin{minipage}[b]{\linewidth}\raggedright
Konstruk
\end{minipage} & \begin{minipage}[b]{\linewidth}\raggedright
AVE
\end{minipage} & \begin{minipage}[b]{\linewidth}\raggedright
CR
\end{minipage} & \begin{minipage}[b]{\linewidth}\raggedright
Cronbach's alpha
\end{minipage} \\
\midrule\noalign{}
\endhead
\bottomrule\noalign{}
\endlastfoot
ESD & 0,720 & 0,886 & 0,807 \\
PjBL & 0,687 & 0,917 & 0,886 \\
RPP & 1,000 & 1,000 & --- \\
STEM & 0,480 & 0,814 & 0,694 \\
TPACK & 0,418 & 0,834 & 0,766 \\
\end{longtable}
}

Threshold: AVE \textgreater= 0,50, CR \textgreater= 0,70, alpha
\textgreater= 0,70. RPP adalah konstruk single-indicator.

\textbf{Tabel 6b.} Matriks heterotrait-monotrait (HTMT)

{\def\LTcaptype{none} % do not increment counter
\begin{longtable}[]{@{}
  >{\raggedright\arraybackslash}p{(\linewidth - 8\tabcolsep) * \real{0.1528}}
  >{\raggedright\arraybackslash}p{(\linewidth - 8\tabcolsep) * \real{0.1111}}
  >{\raggedright\arraybackslash}p{(\linewidth - 8\tabcolsep) * \real{0.1250}}
  >{\raggedright\arraybackslash}p{(\linewidth - 8\tabcolsep) * \real{0.1111}}
  >{\raggedright\arraybackslash}p{(\linewidth - 8\tabcolsep) * \real{0.1111}}@{}}
\toprule\noalign{}
\begin{minipage}[b]{\linewidth}\raggedright
Konstruk
\end{minipage} & \begin{minipage}[b]{\linewidth}\raggedright
PjBL
\end{minipage} & \begin{minipage}[b]{\linewidth}\raggedright
TPACK
\end{minipage} & \begin{minipage}[b]{\linewidth}\raggedright
STEM
\end{minipage} & \begin{minipage}[b]{\linewidth}\raggedright
ESD
\end{minipage} \\
\midrule\noalign{}
\endhead
\bottomrule\noalign{}
\endlastfoot
PjBL & --- & & & \\
TPACK & 0,837 & --- & & \\
STEM & 0,871 & 0,770 & --- & \\
ESD & 0,716 & 0,298 & 0,456 & --- \\
\end{longtable}
}

Seluruh nilai \(< 0,90\), mendukung validitas diskriminan.

\subsection{Model struktural: Efek langsung
(RM2)}\label{model-struktural-efek-langsung-rm2}

Uji hubungan struktural untuk menjawab RM2 dilakukan setelah model
pengukuran terkonfirmasi. Tabel 7 menyajikan koefisien jalur, statistik
bootstrap (5.000 iterasi, seed = 42), dan ukuran efek untuk seluruh
jalur langsung.

\textbf{Tabel 7.} Model struktural --- koefisien jalur dan signifikansi

{\def\LTcaptype{none} % do not increment counter
\begin{longtable}[]{@{}
  >{\raggedright\arraybackslash}p{(\linewidth - 26\tabcolsep) * \real{0.1194}}
  >{\raggedright\arraybackslash}p{(\linewidth - 26\tabcolsep) * \real{0.0597}}
  >{\raggedright\arraybackslash}p{(\linewidth - 26\tabcolsep) * \real{0.0597}}
  >{\raggedright\arraybackslash}p{(\linewidth - 26\tabcolsep) * \real{0.0597}}
  >{\raggedright\arraybackslash}p{(\linewidth - 26\tabcolsep) * \real{0.0448}}
  >{\raggedright\arraybackslash}p{(\linewidth - 26\tabcolsep) * \real{0.0448}}
  >{\raggedright\arraybackslash}p{(\linewidth - 26\tabcolsep) * \real{0.0448}}
  >{\raggedright\arraybackslash}p{(\linewidth - 26\tabcolsep) * \real{0.0597}}
  >{\centering\arraybackslash}p{(\linewidth - 26\tabcolsep) * \real{0.0597}}
  >{\raggedright\arraybackslash}p{(\linewidth - 26\tabcolsep) * \real{0.0896}}
  >{\raggedright\arraybackslash}p{(\linewidth - 26\tabcolsep) * \real{0.0597}}
  >{\raggedright\arraybackslash}p{(\linewidth - 26\tabcolsep) * \real{0.0597}}
  >{\raggedright\arraybackslash}p{(\linewidth - 26\tabcolsep) * \real{0.0896}}
  >{\raggedright\arraybackslash}p{(\linewidth - 26\tabcolsep) * \real{0.0597}}@{}}
\toprule\noalign{}
\multicolumn{2}{@{}>{\raggedright\arraybackslash}p{(\linewidth - 26\tabcolsep) * \real{0.1791} + 2\tabcolsep}}{%
\begin{minipage}[b]{\linewidth}\raggedright
Jalur
\end{minipage}} &
\multicolumn{2}{>{\raggedright\arraybackslash}p{(\linewidth - 26\tabcolsep) * \real{0.1194} + 2\tabcolsep}}{%
\begin{minipage}[b]{\linewidth}\raggedright
\[\beta\]
\end{minipage}} &
\multicolumn{2}{>{\raggedright\arraybackslash}p{(\linewidth - 26\tabcolsep) * \real{0.0896} + 2\tabcolsep}}{%
\begin{minipage}[b]{\linewidth}\raggedright
\[SE\]
\end{minipage}} &
\multicolumn{2}{>{\raggedright\arraybackslash}p{(\linewidth - 26\tabcolsep) * \real{0.1045} + 2\tabcolsep}}{%
\begin{minipage}[b]{\linewidth}\raggedright
\[t\]
\end{minipage}} & \begin{minipage}[b]{\linewidth}\centering
\[p\]
\end{minipage} & \begin{minipage}[b]{\linewidth}\raggedright
CI 95\%
\end{minipage} &
\multicolumn{2}{>{\raggedright\arraybackslash}p{(\linewidth - 26\tabcolsep) * \real{0.1194} + 2\tabcolsep}}{%
\begin{minipage}[b]{\linewidth}\raggedright
Sig.
\end{minipage}} & \begin{minipage}[b]{\linewidth}\raggedright
\[f^{2}\]
\end{minipage} & \begin{minipage}[b]{\linewidth}\raggedright
\end{minipage} \\
\midrule\noalign{}
\endhead
\bottomrule\noalign{}
\endlastfoot
PjBL \(\rightarrow\) TPACK &
\multicolumn{2}{>{\raggedright\arraybackslash}p{(\linewidth - 26\tabcolsep) * \real{0.1194} + 2\tabcolsep}}{%
0,727} &
\multicolumn{2}{>{\raggedright\arraybackslash}p{(\linewidth - 26\tabcolsep) * \real{0.1045} + 2\tabcolsep}}{%
0,055} &
\multicolumn{2}{>{\raggedright\arraybackslash}p{(\linewidth - 26\tabcolsep) * \real{0.0896} + 2\tabcolsep}}{%
13,295} &
\multicolumn{2}{>{\raggedright\arraybackslash}p{(\linewidth - 26\tabcolsep) * \real{0.1194} + 2\tabcolsep}}{%
\(< ,001\)} & {[}0,610; 0,823{]} & Ya &
\multicolumn{3}{>{\raggedright\arraybackslash}p{(\linewidth - 26\tabcolsep) * \real{0.2090} + 4\tabcolsep}@{}}{%
1,123 (Big)} \\
PjBL \(\rightarrow\) STEM &
\multicolumn{2}{>{\raggedright\arraybackslash}p{(\linewidth - 26\tabcolsep) * \real{0.1194} + 2\tabcolsep}}{%
0,683} &
\multicolumn{2}{>{\raggedright\arraybackslash}p{(\linewidth - 26\tabcolsep) * \real{0.1045} + 2\tabcolsep}}{%
0,054} &
\multicolumn{2}{>{\raggedright\arraybackslash}p{(\linewidth - 26\tabcolsep) * \real{0.0896} + 2\tabcolsep}}{%
12,616} &
\multicolumn{2}{>{\raggedright\arraybackslash}p{(\linewidth - 26\tabcolsep) * \real{0.1194} + 2\tabcolsep}}{%
\(< ,001\)} & {[}0,573; 0,782{]} & Ya &
\multicolumn{3}{>{\raggedright\arraybackslash}p{(\linewidth - 26\tabcolsep) * \real{0.2090} + 4\tabcolsep}@{}}{%
0,872 (Big)} \\
PjBL \(\rightarrow\) ESD &
\multicolumn{2}{>{\raggedright\arraybackslash}p{(\linewidth - 26\tabcolsep) * \real{0.1194} + 2\tabcolsep}}{%
0,617} &
\multicolumn{2}{>{\raggedright\arraybackslash}p{(\linewidth - 26\tabcolsep) * \real{0.1045} + 2\tabcolsep}}{%
0,065} &
\multicolumn{2}{>{\raggedright\arraybackslash}p{(\linewidth - 26\tabcolsep) * \real{0.0896} + 2\tabcolsep}}{%
9,496} &
\multicolumn{2}{>{\raggedright\arraybackslash}p{(\linewidth - 26\tabcolsep) * \real{0.1194} + 2\tabcolsep}}{%
\(< ,001\)} & {[}0,485; 0,739{]} & Ya &
\multicolumn{3}{>{\raggedright\arraybackslash}p{(\linewidth - 26\tabcolsep) * \real{0.2090} + 4\tabcolsep}@{}}{%
0,614 (Big)} \\
PjBL \(\rightarrow\) RPP &
\multicolumn{2}{>{\raggedright\arraybackslash}p{(\linewidth - 26\tabcolsep) * \real{0.1194} + 2\tabcolsep}}{%
0,030} &
\multicolumn{2}{>{\raggedright\arraybackslash}p{(\linewidth - 26\tabcolsep) * \real{0.1045} + 2\tabcolsep}}{%
0,045} &
\multicolumn{2}{>{\raggedright\arraybackslash}p{(\linewidth - 26\tabcolsep) * \real{0.0896} + 2\tabcolsep}}{%
0,770} &
\multicolumn{2}{>{\raggedright\arraybackslash}p{(\linewidth - 26\tabcolsep) * \real{0.1194} + 2\tabcolsep}}{%
,441} & {[}-0,055; 0,123{]} & Tidak &
\multicolumn{3}{>{\raggedright\arraybackslash}p{(\linewidth - 26\tabcolsep) * \real{0.2090} + 4\tabcolsep}@{}}{%
0,007 (Negligible)} \\
TPACK \(\rightarrow\) RPP &
\multicolumn{2}{>{\raggedright\arraybackslash}p{(\linewidth - 26\tabcolsep) * \real{0.1194} + 2\tabcolsep}}{%
0,425} &
\multicolumn{2}{>{\raggedright\arraybackslash}p{(\linewidth - 26\tabcolsep) * \real{0.1045} + 2\tabcolsep}}{%
0,037} &
\multicolumn{2}{>{\raggedright\arraybackslash}p{(\linewidth - 26\tabcolsep) * \real{0.0896} + 2\tabcolsep}}{%
11,346} &
\multicolumn{2}{>{\raggedright\arraybackslash}p{(\linewidth - 26\tabcolsep) * \real{0.1194} + 2\tabcolsep}}{%
\(< ,001\)} & {[}0,345; 0,491{]} & Ya &
\multicolumn{3}{>{\raggedright\arraybackslash}p{(\linewidth - 26\tabcolsep) * \real{0.2090} + 4\tabcolsep}@{}}{%
2,783 (Big)} \\
STEM \(\rightarrow\) RPP &
\multicolumn{2}{>{\raggedright\arraybackslash}p{(\linewidth - 26\tabcolsep) * \real{0.1194} + 2\tabcolsep}}{%
0,484} &
\multicolumn{2}{>{\raggedright\arraybackslash}p{(\linewidth - 26\tabcolsep) * \real{0.1045} + 2\tabcolsep}}{%
0,037} &
\multicolumn{2}{>{\raggedright\arraybackslash}p{(\linewidth - 26\tabcolsep) * \real{0.0896} + 2\tabcolsep}}{%
13,116} &
\multicolumn{2}{>{\raggedright\arraybackslash}p{(\linewidth - 26\tabcolsep) * \real{0.1194} + 2\tabcolsep}}{%
\(< ,001\)} & {[}0,412; 0,556{]} & Ya &
\multicolumn{3}{>{\raggedright\arraybackslash}p{(\linewidth - 26\tabcolsep) * \real{0.2090} + 4\tabcolsep}@{}}{%
5,444 (Big)} \\
ESD \(\rightarrow\) RPP &
\multicolumn{2}{>{\raggedright\arraybackslash}p{(\linewidth - 26\tabcolsep) * \real{0.1194} + 2\tabcolsep}}{%
0,345} &
\multicolumn{2}{>{\raggedright\arraybackslash}p{(\linewidth - 26\tabcolsep) * \real{0.1045} + 2\tabcolsep}}{%
0,040} &
\multicolumn{2}{>{\raggedright\arraybackslash}p{(\linewidth - 26\tabcolsep) * \real{0.0896} + 2\tabcolsep}}{%
8,355} &
\multicolumn{2}{>{\raggedright\arraybackslash}p{(\linewidth - 26\tabcolsep) * \real{0.1194} + 2\tabcolsep}}{%
\(< ,001\)} & {[}0,264; 0,423{]} & Ya &
\multicolumn{3}{>{\raggedright\arraybackslash}p{(\linewidth - 26\tabcolsep) * \real{0.2090} + 4\tabcolsep}@{}}{%
2,399 (Big)} \\
\end{longtable}
}

Sebagaimana ditunjukkan pada Tabel 7, PjBL memberikan efek positif
signifikan terhadap ketiga dimensi integrasi: TPACK (\(\beta = 0,727\),
\(p < ,001\), \(f^{2} = 1,123\)), STEM (\(\beta = 0,683\), \(p < ,001\),
\(f^{2} = 0,872\)), dan ESD (\(\beta = 0,617\), \(p < ,001\),
\(f^{2} = 0,614\)). Seluruh ukuran efek besar. Penting dicatat, jalur
langsung dari PjBL ke RPP tidak signifikan (\(\beta = 0,030\),
\(p = ,441\)), menunjukkan bahwa PjBL tidak langsung mempengaruhi
kualitas RPP tetapi bekerja melalui konstruk mediator.

Model menjelaskan 97,7\% varians RPP (\(R^{2} = 0,977\)), 52,9\% TPACK
(\(R^{2} = 0,529\)), 46,6\% STEM (\(R^{2} = 0,466\)), dan 38,0\% ESD
(\(R^{2} = 0,380\)). Relevansi prediktif (\(Q^{2}\)) bernilai positif
dan substansial untuk seluruh konstruk endogen: RPP (0,974), TPACK
(0,503), STEM (0,430), dan ESD (0,355), menunjukkan kapasitas prediktif
yang kuat melampaui prediksi rerata sederhana.

\subsection{Analisis komparatif: Dimensi dominan
(RM3)}\label{analisis-komparatif-dimensi-dominan-rm3}

Berdasarkan temuan RM2, RM3 menanyakan dimensi integrasi mana yang
paling responsif terhadap PjBL. Tabel 8 membandingkan koefisien jalur
PjBL ke dimensi.

\textbf{Tabel 8.} RM3 --- Perbandingan Koefisien Jalur PjBL ke dimensi

{\def\LTcaptype{none} % do not increment counter
\begin{longtable}[]{@{}
  >{\raggedright\arraybackslash}p{(\linewidth - 12\tabcolsep) * \real{0.0897}}
  >{\raggedright\arraybackslash}p{(\linewidth - 12\tabcolsep) * \real{0.1282}}
  >{\raggedright\arraybackslash}p{(\linewidth - 12\tabcolsep) * \real{0.1538}}
  >{\raggedright\arraybackslash}p{(\linewidth - 12\tabcolsep) * \real{0.1154}}
  >{\centering\arraybackslash}p{(\linewidth - 12\tabcolsep) * \real{0.1667}}
  >{\raggedright\arraybackslash}p{(\linewidth - 12\tabcolsep) * \real{0.1538}}
  >{\raggedright\arraybackslash}p{(\linewidth - 12\tabcolsep) * \real{0.1667}}@{}}
\toprule\noalign{}
\begin{minipage}[b]{\linewidth}\raggedright
Rank
\end{minipage} & \begin{minipage}[b]{\linewidth}\raggedright
Dimensi
\end{minipage} & \begin{minipage}[b]{\linewidth}\raggedright
\[\beta\]
\end{minipage} & \begin{minipage}[b]{\linewidth}\raggedright
\[t\]
\end{minipage} & \begin{minipage}[b]{\linewidth}\centering
\[p\]
\end{minipage} & \begin{minipage}[b]{\linewidth}\raggedright
\[f^{2}\]
\end{minipage} & \begin{minipage}[b]{\linewidth}\raggedright
Signifikan
\end{minipage} \\
\midrule\noalign{}
\endhead
\bottomrule\noalign{}
\endlastfoot
1 & TPACK & 0,727 & 13,295 & \(< ,001\) & 1,123 & Ya \\
2 & STEM & 0,683 & 12,616 & \(< ,001\) & 0,872 & Ya \\
3 & ESD & 0,617 & 9,496 & \(< ,001\) & 0,614 & Ya \\
\end{longtable}
}

Sebagaimana ditunjukkan pada Tabel 8, TPACK muncul sebagai dimensi
paling responsif (\(\beta = 0,727\), \(f^{2} = 1,123\)), diikuti STEM
(\(\beta = 0,683\), \(f^{2} = 0,872\)) dan ESD (\(\beta = 0,617\),
\(f^{2} = 0,614\)). Seluruh jalur signifikan dengan ukuran efek besar,
menghasilkan urutan: TPACK > STEM > ESD.

\begin{figure}
\centering
\includegraphics[width=4.6in,alt={Gambar 4. Koefisien Jalur dari PjBL ke Konstruk Mediator}]{figures/media/fig3_sem_rm3_paths.png}
\caption{Gambar 4. Koefisien Jalur dari PjBL ke Konstruk Mediator}
\end{figure}

\begin{enumerate}
\def\labelenumi{\arabic{enumi}.}
\setcounter{enumi}{3}
\tightlist
\item
  \emph{Higher-Order Construct: Dimensi yang Berkontribusi terhadap
  Kualitas RPP (RM4)}
\end{enumerate}

RM4 fokus pada bagaimana dimensi-dimensi ini berkontribusi terhadap
kualitas RPP integratif. Gambar 5. mengilustrasikan model struktural
lengkap dengan jalur dari PjBL melalui konstruk mediator ke RPP.

\begin{figure}
\centering
\includegraphics[width=3.87402in,height=2.30315in,alt={Gambar 5. Prediktor Kualitas RPP Integratif}]{figures/media/fig4_sem_full_model_hoc_proxy.png}
\caption{Gambar 5. Prediktor Kualitas RPP Integratif}
\end{figure}

Gambar 5. Prediktor kualitas RPP integratif

Gambar 5. Menunjukkan ketiga dimensi berkontribusi signifikan: STEM
(\(\beta = 0,484\)), TPACK (\(\beta = 0,425\)), dan ESD
(\(\beta = 0,345\)). Jalur langsung dari PjBL ke RPP tidak signifikan
(\(\beta = 0,030\), garis putus-putus).

Sebagaimana ditunjukkan pada Gambar 5 dan Tabel 7, ketiga dimensi
integrasi berkontribusi signifikan dan substansial terhadap kualitas RPP
integratif. STEM adalah kontributor terkuat (\(\beta = 0,484\)), diikuti
TPACK (\(\beta = 0,425\)) dan ESD (\(\beta = 0,345\)). Nilai \(f^{2}\)
yang konsisten besar (seluruh \(> 2,3\)) menunjukkan bahwa setiap
dimensi memberikan kontribusi bermakna dan non-redundan terhadap
kualitas RPP keseluruhan.

\subsection{Analisis mediasi (RM5)}\label{analisis-mediasi-rm5}

RM5 menguji apakah TPACK, STEM, dan ESD memediasi hubungan antara PjBL
dan kualitas RPP. Tabel 9 menyajikan hasil analisis mediasi.

\textbf{Tabel 9.} Analisis mediasi --- efek tidak langsung

{\def\LTcaptype{none} % do not increment counter
\begin{longtable}[]{@{}
  >{\raggedright\arraybackslash}p{(\linewidth - 26\tabcolsep) * \real{0.1206}}
  >{\raggedright\arraybackslash}p{(\linewidth - 26\tabcolsep) * \real{0.0922}}
  >{\raggedright\arraybackslash}p{(\linewidth - 26\tabcolsep) * \real{0.0922}}
  >{\raggedright\arraybackslash}p{(\linewidth - 26\tabcolsep) * \real{0.0426}}
  >{\raggedright\arraybackslash}p{(\linewidth - 26\tabcolsep) * \real{0.0426}}
  >{\raggedright\arraybackslash}p{(\linewidth - 26\tabcolsep) * \real{0.0426}}
  >{\raggedright\arraybackslash}p{(\linewidth - 26\tabcolsep) * \real{0.0567}}
  >{\centering\arraybackslash}p{(\linewidth - 26\tabcolsep) * \real{0.0567}}
  >{\centering\arraybackslash}p{(\linewidth - 26\tabcolsep) * \real{0.0567}}
  >{\raggedright\arraybackslash}p{(\linewidth - 26\tabcolsep) * \real{0.0638}}
  >{\raggedright\arraybackslash}p{(\linewidth - 26\tabcolsep) * \real{0.0567}}
  >{\raggedright\arraybackslash}p{(\linewidth - 26\tabcolsep) * \real{0.0567}}
  >{\raggedright\arraybackslash}p{(\linewidth - 26\tabcolsep) * \real{0.0426}}
  >{\centering\arraybackslash}p{(\linewidth - 26\tabcolsep) * \real{0.0922}}@{}}
\toprule\noalign{}
\begin{minipage}[b]{\linewidth}\raggedright
Jalur Tidak Langsung
\end{minipage} &
\multicolumn{2}{>{\raggedright\arraybackslash}p{(\linewidth - 26\tabcolsep) * \real{0.1844} + 2\tabcolsep}}{%
\begin{minipage}[b]{\linewidth}\raggedright
\[\beta_{indirect}\]
\end{minipage}} &
\multicolumn{2}{>{\raggedright\arraybackslash}p{(\linewidth - 26\tabcolsep) * \real{0.0851} + 2\tabcolsep}}{%
\begin{minipage}[b]{\linewidth}\raggedright
\[SE\]
\end{minipage}} &
\multicolumn{2}{>{\raggedright\arraybackslash}p{(\linewidth - 26\tabcolsep) * \real{0.0993} + 2\tabcolsep}}{%
\begin{minipage}[b]{\linewidth}\raggedright
\[t\]
\end{minipage}} &
\multicolumn{2}{>{\centering\arraybackslash}p{(\linewidth - 26\tabcolsep) * \real{0.1135} + 2\tabcolsep}}{%
\begin{minipage}[b]{\linewidth}\centering
\[p\]
\end{minipage}} & \begin{minipage}[b]{\linewidth}\raggedright
CI 95\%
\end{minipage} & \begin{minipage}[b]{\linewidth}\raggedright
VAF
\end{minipage} & \begin{minipage}[b]{\linewidth}\raggedright
Sobel \(z\)
\end{minipage} &
\multicolumn{2}{>{\raggedright\arraybackslash}p{(\linewidth - 26\tabcolsep) * \real{0.1348} + 2\tabcolsep}@{}}{%
\begin{minipage}[b]{\linewidth}\raggedright
Sobel \(p\)
\end{minipage}} \\
\midrule\noalign{}
\endhead
\bottomrule\noalign{}
\endlastfoot
PjBL \(\rightarrow\) TPACK \(\rightarrow\) RPP & 0,309 &
\multicolumn{2}{>{\raggedright\arraybackslash}p{(\linewidth - 26\tabcolsep) * \real{0.1348} + 2\tabcolsep}}{%
0,035} &
\multicolumn{2}{>{\raggedright\arraybackslash}p{(\linewidth - 26\tabcolsep) * \real{0.0851} + 2\tabcolsep}}{%
8,725} &
\multicolumn{2}{>{\raggedright\arraybackslash}p{(\linewidth - 26\tabcolsep) * \real{0.1135} + 2\tabcolsep}}{%
\(< ,001\)} &
\multicolumn{2}{>{\centering\arraybackslash}p{(\linewidth - 26\tabcolsep) * \real{0.1206} + 2\tabcolsep}}{%
{[}0,239; 0,375{]}} & 35,0\% &
\multicolumn{2}{>{\raggedright\arraybackslash}p{(\linewidth - 26\tabcolsep) * \real{0.0993} + 2\tabcolsep}}{%
8,686} & \(< ,001\) \\
PjBL \(\rightarrow\) STEM \(\rightarrow\) RPP & 0,330 &
\multicolumn{2}{>{\raggedright\arraybackslash}p{(\linewidth - 26\tabcolsep) * \real{0.1348} + 2\tabcolsep}}{%
0,033} &
\multicolumn{2}{>{\raggedright\arraybackslash}p{(\linewidth - 26\tabcolsep) * \real{0.0851} + 2\tabcolsep}}{%
10,039} &
\multicolumn{2}{>{\raggedright\arraybackslash}p{(\linewidth - 26\tabcolsep) * \real{0.1135} + 2\tabcolsep}}{%
\(< ,001\)} &
\multicolumn{2}{>{\centering\arraybackslash}p{(\linewidth - 26\tabcolsep) * \real{0.1206} + 2\tabcolsep}}{%
{[}0,271; 0,401{]}} & 37,4\% &
\multicolumn{2}{>{\raggedright\arraybackslash}p{(\linewidth - 26\tabcolsep) * \real{0.0993} + 2\tabcolsep}}{%
9,052} & \(< ,001\) \\
PjBL \(\rightarrow\) ESD \(\rightarrow\) RPP & 0,213 &
\multicolumn{2}{>{\raggedright\arraybackslash}p{(\linewidth - 26\tabcolsep) * \real{0.1348} + 2\tabcolsep}}{%
0,031} &
\multicolumn{2}{>{\raggedright\arraybackslash}p{(\linewidth - 26\tabcolsep) * \real{0.0851} + 2\tabcolsep}}{%
6,685} &
\multicolumn{2}{>{\raggedright\arraybackslash}p{(\linewidth - 26\tabcolsep) * \real{0.1135} + 2\tabcolsep}}{%
\(< ,001\)} &
\multicolumn{2}{>{\centering\arraybackslash}p{(\linewidth - 26\tabcolsep) * \real{0.1206} + 2\tabcolsep}}{%
{[}0,151; 0,275{]}} & 24,1\% &
\multicolumn{2}{>{\raggedright\arraybackslash}p{(\linewidth - 26\tabcolsep) * \real{0.0993} + 2\tabcolsep}}{%
6,335} & \(< ,001\) \\
Total indirect & 0,852 &
\multicolumn{2}{>{\raggedright\arraybackslash}p{(\linewidth - 26\tabcolsep) * \real{0.1348} + 2\tabcolsep}}{%
0,040} &
\multicolumn{2}{>{\raggedright\arraybackslash}p{(\linewidth - 26\tabcolsep) * \real{0.0851} + 2\tabcolsep}}{%
21,016} &
\multicolumn{2}{>{\raggedright\arraybackslash}p{(\linewidth - 26\tabcolsep) * \real{0.1135} + 2\tabcolsep}}{%
\(< ,001\)} &
\multicolumn{2}{>{\centering\arraybackslash}p{(\linewidth - 26\tabcolsep) * \real{0.1206} + 2\tabcolsep}}{%
{[}0,769; 0,928{]}} & \begin{minipage}[t]{\linewidth}\raggedright
\begin{center}\rule{0.5\linewidth}{0.5pt}\end{center}
\end{minipage} &
\multicolumn{2}{>{\raggedright\arraybackslash}p{(\linewidth - 26\tabcolsep) * \real{0.0993} + 2\tabcolsep}}{%
\begin{minipage}[t]{\linewidth}\raggedright
\begin{center}\rule{0.5\linewidth}{0.5pt}\end{center}
\end{minipage}} & \begin{minipage}[t]{\linewidth}\centering
\begin{center}\rule{0.5\linewidth}{0.5pt}\end{center}
\end{minipage} \\
Langsung (PjBL \(\rightarrow\) RPP) & 0,030 &
\multicolumn{2}{>{\raggedright\arraybackslash}p{(\linewidth - 26\tabcolsep) * \real{0.1348} + 2\tabcolsep}}{%
0,045} &
\multicolumn{2}{>{\raggedright\arraybackslash}p{(\linewidth - 26\tabcolsep) * \real{0.0851} + 2\tabcolsep}}{%
0,770} &
\multicolumn{2}{>{\raggedright\arraybackslash}p{(\linewidth - 26\tabcolsep) * \real{0.1135} + 2\tabcolsep}}{%
,441} &
\multicolumn{2}{>{\centering\arraybackslash}p{(\linewidth - 26\tabcolsep) * \real{0.1206} + 2\tabcolsep}}{%
{[}-0,055; 0,123{]}} & \begin{minipage}[t]{\linewidth}\raggedright
\begin{center}\rule{0.5\linewidth}{0.5pt}\end{center}
\end{minipage} &
\multicolumn{2}{>{\raggedright\arraybackslash}p{(\linewidth - 26\tabcolsep) * \real{0.0993} + 2\tabcolsep}}{%
\begin{minipage}[t]{\linewidth}\raggedright
\begin{center}\rule{0.5\linewidth}{0.5pt}\end{center}
\end{minipage}} & \begin{minipage}[t]{\linewidth}\centering
\begin{center}\rule{0.5\linewidth}{0.5pt}\end{center}
\end{minipage} \\
Total effect & 0,882 &
\multicolumn{2}{>{\raggedright\arraybackslash}p{(\linewidth - 26\tabcolsep) * \real{0.1348} + 2\tabcolsep}}{%
0,021} &
\multicolumn{2}{>{\raggedright\arraybackslash}p{(\linewidth - 26\tabcolsep) * \real{0.0851} + 2\tabcolsep}}{%
42,952} &
\multicolumn{2}{>{\raggedright\arraybackslash}p{(\linewidth - 26\tabcolsep) * \real{0.1135} + 2\tabcolsep}}{%
\(< ,001\)} &
\multicolumn{2}{>{\centering\arraybackslash}p{(\linewidth - 26\tabcolsep) * \real{0.1206} + 2\tabcolsep}}{%
{[}0,838; 0,917{]}} & \begin{minipage}[t]{\linewidth}\raggedright
\begin{center}\rule{0.5\linewidth}{0.5pt}\end{center}
\end{minipage} &
\multicolumn{2}{>{\raggedright\arraybackslash}p{(\linewidth - 26\tabcolsep) * \real{0.0993} + 2\tabcolsep}}{%
\begin{minipage}[t]{\linewidth}\raggedright
\begin{center}\rule{0.5\linewidth}{0.5pt}\end{center}
\end{minipage}} & \begin{minipage}[t]{\linewidth}\centering
\begin{center}\rule{0.5\linewidth}{0.5pt}\end{center}
\end{minipage} \\
\end{longtable}
}

\begin{figure}
\centering
\includegraphics[width=4.8in,alt={Gambar 6. Efek Tidak Langsung Spesifik melalui Konstruk Mediator}]{figures/media/fig5_sem_mediation_paths.png}
\caption{Gambar 6. Efek Tidak Langsung Spesifik melalui Konstruk Mediator}
\end{figure}

\begin{quote}
Sebagaimana ditunjukkan pada Tabel 9, ketiga jalur tidak langsung
signifikan secara statistik berdasarkan confidence interval bootstrap:
\end{quote}

\begin{enumerate}
\def\labelenumi{\alph{enumi}.}
\item
  PjBL \(\rightarrow\) STEM \(\rightarrow\) RPP adalah jalur mediasi
  terkuat (indirect \(\beta = 0,330\), \(p < ,001\), CI 95\% {[}0,271;
  0,401{]}). VAF 37,4\% menunjukkan mediasi parsial, dan uji Sobel
  mengonfirmasi signifikansi (\(z = 9,052\), \(p < ,001\)).
\item
  PjBL \(\rightarrow\) TPACK \(\rightarrow\) RPP menghasilkan efek tidak
  langsung signifikan (indirect \(\beta = 0,309\), \(p < ,001\), CI 95\%
  {[}0,239; 0,375{]}). VAF 35,0\% konsisten dengan mediasi parsial.
\item
  PjBL \(\rightarrow\) ESD \(\rightarrow\) RPP juga signifikan (indirect
  \(\beta = 0,213\), \(p < ,001\), CI 95\% {[}0,151; 0,275{]}). VAF
  24,1\% menunjukkan mediasi parsial.
\end{enumerate}

\begin{quote}
Total indirect effect signifikan (\(\beta = 0,852\), \(p < ,001\)),
sedangkan direct effect PjBL terhadap RPP tidak signifikan
(\(\beta = 0,030\), \(p = ,441\)). Pola ini konsisten dengan full
mediation secara agregat: PjBL mempengaruhi kualitas RPP integratif
bukan secara langsung, tetapi melalui peningkatan ketiga dimensi
integrasi. Total effect PjBL terhadap RPP signifikan (\(\beta = 0,882\),
\(p < ,001\)), mengonfirmasi bahwa pengaruh keseluruhan PjBL terhadap
kualitas RPP substansial dan bekerja melalui dimensi mediator.
\end{quote}


\section{Discussion}\label{discussion}

\begin{verbatim}
1.  *Efektivitas PjBL dalam meningkatkan kompetensi integratif
    (RQ1)*
\end{verbatim}

Analisis pre--post menunjukkan peningkatan yang signifikan secara
statistik pada keempat konstruk, dengan ukuran efek yang konsisten besar
(Cohen's d > 2,3). Temuan ini menjadi bukti kuat bahwa
kompetensi perencanaan pembelajaran integratif partisipan meningkat
secara substansial setelah intervensi PjBL dibandingkan sebelum
intervensi. Hasil ini sejalan dengan \\citet{dewi2022projectbased} yang melaporkan
bahwa scaffolding PjBL meningkatkan TPACK dan kemampuan desain
pembelajaran calon guru, sekaligus memperluas bukti bahwa peningkatan
tidak hanya pada integrasi teknologi, tetapi juga mencakup dimensi STEM
dan ESD.

Analisis N-Gain memperlihatkan perbedaan tingkat respons antar konstruk.
TPACK (\emph{M} = 0,596) dan STEM (\emph{M} = 0,574) mencatat gain
tertinggi dan keduanya mendekati batas atas kategori sedang. RPP
Integratif (\emph{M} = 0,513) berada pada tingkat sedang bagian tengah.
Sebaliknya, ESD memiliki N-Gain terendah (\emph{M} = 0,376), hampir
tidak melewati ambang batas sedang; tidak ada partisipan yang mencapai
kategori tinggi dan 28,4\% masih berada pada kategori rendah.

Pola diferensial ini penting secara pedagogis. Kenaikan yang lebih kuat
pada TPACK dan STEM kemungkinan dipengaruhi oleh kesesuaian
karakteristik PjBL dengan kedua kompetensi tersebut. PjBL menekankan
produk/artefak yang menuntut integrasi teknologi (menguatkan TPACK) dan
umumnya menggunakan pertanyaan pendorong lintas disiplin yang
memfasilitasi cara berpikir interdisipliner (mendukung STEM).
Sebaliknya, ESD (terutama kemampuan mengintegrasikan perspektif
keberlanjutan, inkuiri lingkungan, dan penalaran evaluatif atas isu
sosio-ilmiah) memerlukan reorientasi pedagogis yang tidak selalu muncul
otomatis dari praktik PjBL yang bersifat umum. \\citet{purwianingsih2022programfor} juga menemukan pola serupa, bahwa integrasi ESD dalam TPACK calon
guru membutuhkan program yang terstruktur dan terfokus, bukan sekadar
paparan insidental. Karena itu, meskipun PjBL dapat menjadi konteks yang
mendukung pengembangan ESD, diperlukan scaffolding eksplisit (misalnya
pertanyaan pendorong berbasis keberlanjutan, refleksi terstruktur
terkait SDGs, atau kriteria desain ESD khusus) agar peningkatan ESD
dapat melampaui kategori sedang.

Bukti terbaru memperkuat interpretasi ini. Studi tentang desain
pembelajaran PBL-STEM pada calon guru secara konsisten menunjukkan
peningkatan lebih cepat pada kualitas desain interdisipliner dan
perencanaan berbasis teknologi dibandingkan kompetensi aksi
keberlanjutan, kecuali ketika ESD secara sengaja diintegrasikan ke dalam
arsitektur tugas desain \\citep{pitot2024establishinga,portilloblanco2025buildingan,vidal2025preservice}. Sejalan dengan itu, riset pendidikan guru
berbasis keberlanjutan menunjukkan bahwa penguatan kompetensi ESD lebih
efektif ketika refleksi dan tugas keberlanjutan berbasis konteks
komunitas dirancang secara eksplisit \\citep{singhpillay2023preservice,ozdemiryilmazer2025towardsa}.

Pola ini juga selaras dengan bukti implementasi PBL pada konteks IPA
yang menunjukkan bahwa guru cenderung lebih cepat mengimplementasikan
elemen PBL yang tampak (kolaborasi, produksi artefak, presentasi,
refleksi) dibandingkan komponen yang menuntut kendali belajar siswa yang
lebih mendalam; kondisi ini dapat menjelaskan mengapa penalaran
berorientasi ESD berkembang lebih lambat tanpa dukungan desain yang
eksplisit \\citep{markula2022thekey}.

\subsection{PjBL sebagai penggerak dimensi integrasi
(RQ2)}\label{pjbl-sebagai-penggerak-dimensi-integrasi-rq2}

Analisis PLS-SEM mengungkapkan bahwa PjBL memberikan efek positif yang
signifikan secara statistik dan kuat pada ketiga dimensi integrasi:
TPACK (\(\beta = 0,727\), \(p < 0,001\)), STEM (\(\beta = 0,683\),
\(p < 0,001\)), dan ESD (\(\beta = 0,617\), \(p < 0,001\)). Seluruh
effect sizes besar (\(f^{2} > 0,35\)), menunjukkan signifikansi praktis
substansial. Jalur langsung dari PjBL ke RPP tidak signifikan
(\(\beta = 0,030\), \(p = 0,441\)). Temuan ini sepenuhnya mendukung H1:
PjBL secara signifikan mempengaruhi ketiga dimensi integrasi.

Jalur PjBL \(\rightarrow\) TPACK yang kuat (\(\beta = 0,727\)) selaras
dengan kerangka teoretis. Elemen-elemen inti PjBL (pertanyaan pendorong
autentik, investigasi kolaboratif, dan penciptaan artefak) secara
inheren mendorong calon guru mempertimbangkan peran teknologi dalam
mendukung tujuan pedagogis dan penyampaian konten. Temuan ini memperluas
hasil SEM \\citet{mansour2024scienceand} di Qatar yang menunjukkan hubungan
signifikan antara pendekatan pedagogis dan integrasi TPACK. Nilai effect
size yang besar (\(f^{2} = 1,123\)) menunjukkan bahwa PjBL efektif dalam
mengembangkan kompetensi integrasi teknologi.

Interpretasi ini sejalan dengan temuan studi intervensi terbaru bahwa
penerapan teknologi yang terstruktur dalam pendidikan calon guru IPA
meningkatkan TPACK sekaligus orientasi perilaku terhadap penggunaan
teknologi di kelas, sehingga menunjukkan bahwa tugas desain pedagogis
dapat memperkuat pengetahuan dan kesiapan enactment secara simultan
\\citep{stinkenrosner2023technologyimplementation,gurer2021theinfluence}

Jalur PjBL \(\rightarrow\) STEM yang signifikan (\(\beta = 0,683\),
\(f^{2} = 0,872\)) menegaskan bahwa karakter interdisipliner PjBL
mendorong aktivasi penalaran STEM. Dalam merancang rencana pembelajaran
berbasis PjBL, partisipan perlu mengaitkan konten sains dengan tantangan
desain teknik, memanfaatkan teknologi untuk pengumpulan dan analisis
data, serta menggunakan penalaran matematis dalam pemecahan masalah.
Kesesuaian tuntutan struktural PjBL dengan integrasi STEM menjelaskan
kuatnya hubungan tersebut.

Jalur PjBL \(\rightarrow\) ESD yang signifikan (\(\beta = 0,617\),
\(f^{2} = 0,614\)) merupakan temuan penting yang tidak selaras dengan
pola N-Gain. Meskipun ESD menunjukkan N-Gain terendah pada analisis
pre--post, model struktural menunjukkan bahwa kualitas implementasi PjBL
secara signifikan memprediksi kompetensi ESD pada data posttest.
Perbedaan ini dapat dijelaskan karena analisis pre--post menilai
peningkatan absolut dari baseline yang lebih rendah, sedangkan SEM
menilai hubungan struktural pada posttest. Dengan demikian, PjBL
berpengaruh terhadap pengembangan ESD secara struktural, tetapi
peningkatan absolutnya dapat terbatas oleh titik awal yang rendah dan
jarak pedagogis yang lebih besar antara struktur umum PjBL dan
kompetensi spesifik ESD.

Secara keseluruhan, PjBL menjelaskan varians substansial dalam ketiga
dimensi integrasi: 52,9\% dalam TPACK, 46,6\% dalam STEM, dan 38,0\%
dalam ESD. Nilai \(R^{2}\) moderat hingga substansial ini menunjukkan
bahwa PjBL sebagai prediktor eksogen tunggal menangkap porsi bermakna
faktor yang memengaruhi kompetensi integrasi. Varians yang belum
terjelaskan kemungkinan berasal dari faktor lain (misalnya pengetahuan
konten awal, literasi digital, efikasi diri mengajar, keyakinan tentang
keberlanjutan, dan keterampilan metakognitif) yang belum dimasukkan
dalam model.

\subsection{Dimensi integrasi yang paling responsif
(RQ3)}\label{dimensi-integrasi-yang-paling-responsif-rq3}

Analisis komparatif menunjukkan bahwa TPACK merupakan dimensi yang
paling kuat dipengaruhi (\(\beta\)=0,727; \(f^2\)=1,123), disusul STEM (\(\beta\)=0,683;
\(f^2\)=0,872) dan ESD (\(\beta\)=0,617; \(f^2\)=0,614). Urutan pengaruh (TPACK
> STEM > ESD) ditetapkan berdasarkan koefisien
jalur dan \emph{effect size} pada model struktural. Ketiga jalur
signifikan dengan \emph{effect size} besar, sehingga PjBL dapat
dinyatakan efektif dalam mengembangkan ketiga dimensi tersebut, dengan
hubungan terkuat pada TPACK.

Secara praktis, urutan ini berimplikasi pada pengembangan kurikulum.
Dominannya pengaruh PjBL terhadap integrasi TPACK mengindikasikan bahwa
orientasi PjBL yang berbasis teknologi dan berfokus pada penciptaan
artefak secara natural mengaktivasi TPACK. Karena itu, program
pendidikan guru dapat menjadikan PjBL sebagai strategi utama untuk
mengembangkan kompetensi TPACK, sambil menambahkan intervensi pendukung
bagi dimensi yang memerlukan penguatan.

Selaras dengan literatur perencanaan pembelajaran, calon guru umumnya
berkembang lebih awal pada struktur perencanaan, integrasi alat, dan
koherensi disipliner, sedangkan pertimbangan desain yang lebih kompleks
(misalnya keadilan atau keberlanjutan) cenderung muncul setelah adanya
\emph{scaffolding} tambahan dan siklus umpan balik iteratif \\citep{karlstrom2021howdo,beckmann2023informaland,davis2024preserviceteachers}.

\begin{enumerate}
\def\labelenumi{\arabic{enumi}.}
\setcounter{enumi}{3}
\tightlist
\item
  \emph{Dimensi sebagai konstituen kualitas rencana pembelajaran
  integratif (RQ4)}
\end{enumerate}

Ketiga dimensi integrasi berkontribusi signifikan terhadap kualitas
rencana pembelajaran integratif, dengan effect size yang secara
konsisten besar: STEM (\(\beta\)=0,484, f\^{}2=5,444), TPACK (\(\beta\)=0,425,
f\^{}2=2,783), dan ESD (\(\beta\)=0,345, f\^{}2=2,399). Temuan ini mendukung H2
dan menegaskan perencanaan pembelajaran integratif sebagai kompetensi
tingkat tinggi yang dibentuk oleh tiga dimensi yang berbeda namun saling
melengkapi.

Nilai \(f^{2}\) yang tinggi (seluruhnya \textgreater2,5) menunjukkan
bahwa setiap dimensi memberikan kontribusi yang substansial dan
non-redundan terhadap kualitas rencana pembelajaran. Secara teoretis,
hal ini relevan bagi kerangka Teacher Design Capacity \citep{brown2009item,mckenney2015teacherdesign}, karena mengindikasikan bahwa kompetensi
perencanaan pembelajaran integratif tidak dapat direduksi menjadi satu
dimensi saja, melainkan membutuhkan pengembangan terkoordinasi integrasi
teknologi, penalaran interdisipliner, dan perspektif keberlanjutan.
Dengan demikian, program pendidikan guru yang hanya menekankan satu
dimensi---misalnya berfokus eksklusif pada TPACK---tidak memadai untuk
membangun kompetensi desain integratif secara utuh.

Argumen tersebut selaras dengan bukti riset perencanaan pembelajaran
kontemporer yang memandang kualitas perencanaan calon guru sebagai
luaran komposit dari berbagai sumber pengetahuan dan keputusan desain,
bukan keterampilan domain tunggal, terutama dalam konteks pendidikan
sains \\citep{krepf2022structuringthe,tellezacosta2023preservice,pleasants2024whatmakes}.

Namun, terdapat catatan metodologis penting. Nilai \(R^{2}\) untuk RPP
sebesar 0,977 tergolong sangat tinggi dan kemungkinan mencerminkan sifat
komposisional konstruk, karena RPPInt\_total\_post dihitung sebagai
rerata skor indikator TPACK, STEM, dan ESD, sehingga ketiga dimensi
menjadi prediktor yang hampir sempurna secara konstruksi. Akibatnya,
nilai \(f^{2}\) yang besar dan \(R^{2}\) yang tinggi tidak sepenuhnya
merepresentasikan hubungan empiris, tetapi juga menangkap keniscayaan
matematis. Hal ini tidak meniadakan implikasi konseptual bahwa
perencanaan pembelajaran integratif bersifat multidimensi, tetapi
koefisien jalur perlu ditafsirkan terutama sebagai konsekuensi
operasionalisasi konstruk, bukan semata-mata besaran efek empiris.

\subsection{Peran mediasi dimensi integrasi
(RQ5)}\label{peran-mediasi-dimensi-integrasi-rq5}

Analisis mediasi mengungkapkan pola mediasi penuh secara agregat: efek
tidak langsung total signifikan (\(\beta = 0,852\), \(p < 0,001\)),
sementara efek langsung PjBL terhadap RPP tidak signifikan
(\(\beta = 0,030\), \(p = 0,441\)). Temuan ini mendukung H3 dan
menegaskan implikasi teoretis bahwa PjBL tidak meningkatkan kualitas
rencana pembelajaran integratif secara langsung, tetapi bekerja melalui
peningkatan tiga kompetensi integrasi.

Ketiga jalur mediasi juga signifikan secara statistik:

\begin{enumerate}
\def\labelenumi{\alph{enumi}.}
\item
  STEM adalah mediator terkuat (\(\beta\) tidak langsung \(= 0,330\),
  VAF \(= 37,4\%\)), menunjukkan bahwa pengaruh PjBL terhadap kualitas
  rencana pembelajaran terutama berlangsung melalui pengembangan
  kompetensi integrasi STEM.
\item
  TPACK adalah mediator terkuat kedua (\(\beta\) tidak langsung
  \(= 0,309\), VAF \(= 35,0\%\)), mengindikasikan bahwa integrasi
  teknologi merupakan saluran penting yang memperkuat perencanaan
  pembelajaran melalui PjBL.
\item
  ESD juga merupakan mediator signifikan (\(\beta\) tidak langsung
  \(= 0,213\), VAF \(= 24,1\%\)), sehingga meskipun peningkatan
  absolutnya paling rendah pada analisis pre--post, ESD tetap menjadi
  jalur mediasi yang bermakna dalam model struktural.
\end{enumerate}

Tiga jalur mediasi yang signifikan ini memperluas temuan penelitian
sebelumnya yang masih menunjukkan pola mediasi campuran atau parsial.
Mediasi penuh mengindikasikan bahwa PjBL memengaruhi kualitas rencana
pembelajaran terutama melalui penguatan keterampilan yang spesifik,
bukan melalui mekanisme yang bersifat umum dan tidak terarah \\citep{hair2022item}. Jalur mediasi ESD juga penting: meskipun ESD membutuhkan
scaffolding tambahan untuk memaksimalkan peningkatan absolut
(sebagaimana terlihat pada analisis N-Gain), model struktural menegaskan
bahwa pengembangan kompetensi ESD merupakan bagian inti dari mekanisme
bagaimana PjBL meningkatkan kualitas rencana pembelajaran.

Dalam konteks pembelajaran guru, penjelasan berbasis mekanisme ini
sejalan dengan studi tentang perkembangan perencanaan calon guru, yang
menunjukkan bahwa dampak intervensi biasanya muncul terlebih dahulu
sebagai pergeseran kompetensi antara (misalnya pengetahuan perencanaan,
penalaran pedagogis, dan langkah reflektif dalam desain) sebelum tampak
pada indikator kualitas perencanaan secara holistik \\citep{karlstrom2021howdo,beckmann2023informaland}.

\subsection{Implikasi teoretis}\label{implikasi-teoretis}

Temuan sebelumnya berimplikasi pada pengembangan teori. Pertama,
penelitian ini memberikan uji empiris pertama terhadap model struktural
yang menempatkan TPACK, STEM, dan ESD sebagai mediator simultan antara
PjBL dan kualitas rencana pembelajaran integratif. Mediasi penuh (dengan
ketiga jalur yang signifikan) menunjukkan bahwa model kapasitas desain
guru perlu merinci jalur komposisional bagaimana intervensi memengaruhi
kompetensi desain, bukan memandang kapasitas desain sebagai luaran
tunggal dari pengalaman pedagogis \\citep{brown2009item}.

Kedua, penelitian ini menunjukkan bahwa ketiga dimensi integrasi
dipengaruhi secara signifikan oleh PjBL, sehingga menantang asumsi bahwa
sebagian dimensi (misalnya ESD) kurang dapat dikembangkan melalui PjBL.
Koefisien jalur yang kuat (seluruh \(\beta\)\textgreater0,6) dan effect size
yang besar (\(f^{2} > 0,5\)) menegaskan bahwa PjBL dapat menjadi
intervensi komprehensif untuk mengembangkan berbagai kompetensi
integrasi secara bersamaan.

Ketiga, penelitian ini memvalidasi perencanaan pembelajaran integratif
sebagai konstruk yang dibentuk oleh (bukan sekadar berkorelasi dengan)
kompetensi TPACK, STEM, dan ESD. Signifikansi konsisten dan effect sizes
besar pada ketiga jalur dimensional memperkuat dasar empiris bagi
konseptualisasi konstruk tingkat tinggi.

Temuan ini selaras dengan studi terbaru tentang keberlanjutan
lintas-kurikuler dalam pendidikan guru, yang menegaskan bahwa kualitas
perencanaan holistik muncul ketika logika teknologi, disiplin, dan
keberlanjutan dirancang secara simultan, bukan secara berurutan
\\citep{ozdemiryilmazer2025towardsa,vidal2025preservice}.

\subsection{Implikasi praktis}\label{implikasi-praktis}

Selain kontribusi teoretis, temuan penelitian ini memberikan rekomendasi
praktis bagi program pendidikan guru (Lembaga Pendidikan Tenaga
Kependidikan, LPTK). Pertama, PjBL perlu diadopsi sebagai strategi
pedagogis inti pada mata kuliah persiapan guru IPA karena terbukti
meningkatkan tiga kompetensi integrasi dengan effect size yang besar.
Hubungan struktural yang kuat (\(\beta\)\textgreater0,6 pada seluruh dimensi)
menegaskan bahwa PjBL efektif untuk mengembangkan kompetensi TPACK,
STEM, dan ESD.

Kedua, meskipun PjBL efektif pada semua dimensi, analisis N-Gain
menunjukkan peningkatan absolut ESD paling rendah. Karena itu, PjBL
perlu dilengkapi scaffolding ESD yang eksplisit untuk mengoptimalkan
pengembangan ESD, misalnya melalui pertanyaan pendorong berfokus
keberlanjutan, refleksi terstruktur terkait koneksi SDG, atau kriteria
desain ESD khusus dalam proses PjBL.

Ketiga, asesmen berbasis rubrik dalam penelitian ini menyediakan
instrumen praktis untuk menilai kualitas rencana pembelajaran integratif
calon guru. Rubrik 14 indikator (TPACK (7 item), STEM (4 item), dan ESD
(3 item)) dapat diadopsi atau disesuaikan untuk menilai kompetensi
desain multidimensi melalui tugas kinerja autentik, bukan ukuran laporan
diri.

Keempat, hasil mediasi menunjukkan bahwa ketiga dimensi integrasi
merupakan jalur yang diperlukan bagi PjBL dalam meningkatkan kualitas
rencana pembelajaran. Oleh karena itu, pendidik guru perlu memastikan
aktivitas PjBL mencakup ketiga dimensi tersebut, bukan hanya satu atau
dua dimensi.

Secara praktis dalam desain kurikulum, hal ini berarti brief proyek,
template perencanaan, dan rubrik umpan balik harus secara eksplisit
menuntut bukti keselarasan teknologi--pedagogi, koherensi STEM
interdisipliner, serta penalaran keberlanjutan pada setiap prototipe
pembelajaran \\citep{pitot2024establishinga,tellezacosta2023preservice,portilloblanco2025buildingan}.

\subsection{Kontribusi}\label{kontribusi}

Kontribusi utama penelitian ini adalah menyajikan uji empiris pertama
atas model struktural yang secara simultan menempatkan TPACK, STEM, dan
ESD sebagai mediator antara PjBL dan kualitas rencana pembelajaran
integratif. Temuan bahwa ketiga jalur tersebut berperan sebagai mediator
signifikan memperkuat kerangka \emph{Teacher Design Capacity} dengan
memperjelas jalur komposisional yang menjelaskan bagaimana intervensi
pedagogis diterjemahkan menjadi kompetensi desain. Berbeda dari studi
sebelumnya yang menyatakan bahwa beberapa dimensi (misalnya, ESD) kurang
responsif terhadap pengembangan berbasis PjBL, penelitian ini
menunjukkan bahwa PjBL berpengaruh signifikan terhadap ketiga dimensi
dengan \emph{effect size} besar; namun, analisis N-Gain mengindikasikan
bahwa ESD masih memerlukan \emph{scaffolding} tambahan untuk
mengoptimalkan peningkatan absolut.

\subsection{Keterbatasan}\label{keterbatasan}

Beberapa keterbatasan membatasi interpretasi dan generalisasi temuan.

Konstruk RPP Integratif diukur dengan indikator tunggal
(RPPInt\_total\_post), yaitu rata-rata aritmatika skor TPACK, STEM, dan
ESD. Pendekatan komposit ini berpotensi menghasilkan nilai \(R^2\) yang
tinggi secara mekanistis (0,977) pada konstruk RPP. Alternatif yang
lebih ketat adalah penilaian holistik independen terhadap kualitas RPP
oleh evaluator eksternal.

Partisipan berasal dari satu institusi pendidikan guru sehingga temuan
sulit digeneralisasi ke konteks institusional dan budaya lain. Ukuran
sampel (N = 95) memang melampaui syarat minimum PLS-SEM, tetapi masih
moderat untuk kompleksitas model. Replikasi pada berbagai institusi dan
sampel yang lebih besar diperlukan untuk memperkuat bukti.

Walaupun PjBL berpengaruh signifikan pada ketiga dimensi, analisis
N-Gain menunjukkan peningkatan ESD relatif lebih kecil secara absolut.
Intervensi yang lebih panjang dengan scaffolding ESD yang eksplisit
berpotensi meningkatkan kompetensi ESD secara lebih kuat.

Keterbatasan tersebut sejalan dengan literatur ESD pada pendidikan guru:
intervensi jangka pendek umumnya meningkatkan awareness dan niat
perencanaan, sedangkan action competence ESD yang lebih mendalam
biasanya memerlukan persiapan yang lebih panjang, iteratif, dan tertanam
dalam konteks \\citep{singhpillay2023preservice,vidal2025preservice}.

Meski demikian, penelitian ini tetap memberikan uji model integratif
yang koheren dan berbasis empiris. Konsistensi temuan pre--post (RQ1)
dan SEM (RQ2--RQ5) memperkuat keyakinan terhadap pola hasil, terutama
bahwa PjBL mempengaruhi ketiga dimensi integrasi secara signifikan
melalui mekanisme mediasi penuh.

\subsection{Arah masa depan}\label{arah-masa-depan}

Penelitian selanjutnya perlu mengatasi keterbatasan studi ini melalui
penyempurnaan metodologis. Pertama, replikasi multi-situs pada beragam
konteks institusional dan budaya diperlukan untuk meningkatkan
generalisabilitas. Kedua, intervensi yang lebih panjang disertai
scaffolding ESD terarah (misalnya pertanyaan pendorong berfokus
keberlanjutan dan refleksi SDG yang terstruktur) berpotensi menghasilkan
peningkatan absolut kompetensi ESD yang lebih besar.

Selain itu, studi metode campuran yang memadukan data kualitatif
(seperti wawancara dan observasi kelas) dapat memberi pemahaman yang
lebih mendalam mengenai mekanisme PjBL dalam membentuk pemikiran desain
integratif. Studi longitudinal yang menelusuri retensi kompetensi
integratif setelah periode intervensi juga akan memperkuat bukti tentang
kontribusi PjBL dalam mempersiapkan calon guru IPA.


\section{Conclusion}\label{conclusion}

Penelitian ini menyelidiki pengaruh \emph{Project-Based Learning}
terhadap kompetensi calon guru IPA dalam merencanakan pembelajaran
integratif. Pembelajaran integratif yang dimaksud, didefinisikan sebagai
integrasi simultan dimensi TPACK, STEM, dan ESD dalam desain rencana
pembelajaran. Analisis dilakukan menggunakan uji berpasangan, N-Gain,
dan PLS-SEM untuk menjawab lima pertanyaan penelitian (RQ1--RQ5) serta
menguji tiga hipotesis (H1--H3). Ringkasan temuan utama adalah sebagai
berikut.

RQ1: Keempat konstruk (TPACK, STEM, ESD, dan kualitas rencana
pembelajaran integratif) meningkat secara signifikan setelah intervensi
PjBL, dengan \emph{effect sizes} yang konsisten besar dan nilai N-Gain
pada kategori Sedang. TPACK dan STEM menunjukkan respons peningkatan
paling tinggi. Sebaliknya, ESD memiliki N-Gain paling rendah, tidak ada
partisipan yang mencapai kategori N-Gain Tinggi, dan 28,4\% partisipan
tetap berada pada kategori Rendah.

RQ2: Analisis PLS-SEM menunjukkan bahwa PjBL secara signifikan
memprediksi ketiga dimensi integrasi (TPACK, STEM, dan ESD), semuanya
dengan \emph{effect sizes} besar. Dengan demikian, H1 didukung
sepenuhnya.

RQ3: Dari ketiga dimensi, TPACK adalah yang paling kuat dipengaruhi oleh
PjBL, disusul STEM dan kemudian ESD. Urutan pengaruh (TPACK
> STEM > ESD) ditetapkan berdasarkan koefisien
jalur dan \emph{effect sizes}. Temuan ini menunjukkan bahwa karakter
PjBL yang berorientasi teknologi dan menekankan penciptaan artefak lebih
efektif dalam menguatkan \emph{technological pedagogical content
knowledge}.

RQ4: Ketiga dimensi integrasi berkontribusi signifikan terhadap kualitas
rencana pembelajaran integratif dengan \emph{effect sizes} besar, yaitu
STEM, TPACK, dan ESD. Oleh karena itu, H2 didukung, sekaligus memperkuat
konseptualisasi perencanaan pembelajaran integratif sebagai konstruk
tingkat tinggi.

RQ5: PjBL tidak berpengaruh langsung terhadap kualitas rencana
pembelajaran, tetapi berpengaruh melalui peningkatan tiga kompetensi
integrasi. Tiga jalur mediasi yang signifikan adalah melalui STEM,
TPACK, dan ESD. Secara keseluruhan, efek tidak langsung tergolong kuat,
sedangkan efek langsung tidak signifikan; sehingga H3 didukung
sepenuhnya.

Sebagai penutup, penelitian ini memberikan bukti yang robust bahwa PjBL
efektif dalam meningkatkan kompetensi calon guru IPA dalam merancang
pembelajaran integratif. Temuan mediasi penuh melalui jalur TPACK, STEM,
dan ESD tidak hanya memperdalam pemahaman teoretis tentang bagaimana
intervensi pedagogis berubah menjadi kompetensi desain, tetapi juga
memberikan arahan praktis bagi program pendidikan guru dalam membangun
keterampilan integrasi multidimensional pada calon guru IPA.

\textbf{Acknowledgements}

The author(s) express gratitude to the students who were involved in
this study.

\textbf{Funding Statement}

This research was funded by Universitas Negeri Semarang Budget
Implementation List (DIPA), Number: T/237/UN37/HK.02/2024.

\textbf{CRediT authorship contribution} \textbf{statement}

\textbf{Novi Ratna Dewi}: Conceptualization, Writing -- original draft,
Methodology, Investigation, Formal analysis, Writing -- review \&
editing, Supervision. \textbf{Rizki Nor Amelia}: Investigation, Data
curation, Writing -- review \& editing, Project administration.
\textbf{Septiko Aji}: Data curation, Visualization. \textbf{Ismail Okta
Kurniawan}: Data curation, Software.

\textbf{Declaration of generative AI and AI-assisted technologies in the
writing process}

During the preparation of this work, the author used ChatGPT (OpenAI) to
expand the search for relevant references from their existing
collection, DeepL to translate text from Indonesian to English, and
Grammarly to improve language and readability. After using this tool,
the author reviewed and edited the content as needed and takes full
responsibility for the content of the publication.



\bibliographystyle{apalike}
\bibliography{references}

\end{document}
